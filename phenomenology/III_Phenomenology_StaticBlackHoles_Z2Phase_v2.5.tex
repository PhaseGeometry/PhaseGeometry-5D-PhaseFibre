\documentclass[11pt]{article}

\usepackage[utf8]{inputenc}
\usepackage[english]{babel}
\usepackage{amsmath,amssymb}
\usepackage{geometry}
\geometry{a4paper, margin=2.5cm}

\title{%
  {\large PhaseGeometry Z$_2$ Phenomenology III\\[2pt]}
  {\Large\boldmath Static black holes with binary phase hair\\[4pt]}
}

\author{Aleksey Turchanov}

\date{November 2025}

\begin{document}

\maketitle

\begin{center}
\scriptsize
This note is part of the PhaseGeometry Z$_2$ Core Package v2.5 (strict).\\
Licensed under Creative Commons Attribution 4.0 International (CC BY 4.0).\\
Zenodo record: DOI\,10.5281/zenodo.17807433.
\end{center}

\vspace{0.5em}


\begin{abstract}
We extend the strict PhaseGeometry Z$_2$ framework --- in which a single real
scalar field with a Z$_2$-symmetric double-well potential unifies dark energy
and dark-matter--like haloes --- to static, spherically symmetric black holes.
The same order parameter $\phi(x)$ that generates an effective cosmological
constant $\Lambda_{\rm eff}=8\pi G_0 V(v)$ in the homogeneous broken phase and
supports halo-like defect profiles in the inhomogeneous regime is now coupled
to a black-hole geometry in full general relativity.

We work with a minimally coupled real scalar endowed with the standard quartic
potential
\(
 V(\phi) = \alpha \phi^2 + \frac{\beta}{2}\phi^4
\)
and interpret it as an order parameter of a binary phase medium with
$\phi\simeq\pm v$ in the broken phase. Within this strict Z$_2$ picture we
study two complementary regimes. In a weak-hair regime we linearise the scalar
equation around $\phi=v$ on a fixed Schwarzschild--de~Sitter background and
compute the associated excess energy density $\rho_\phi(r)$, which yields
a halo-like mass profile around the black hole. In a defect-like regime we
outline configurations in which the near-horizon region hosts a kink-like
transition of $\phi$ between the two minima $\pm v$, producing a
domain-wall-like shell that dresses the horizon.

The resulting ``phase-dressed'' black holes fit naturally into the broader
PhaseGeometry Z$_2$ programme: they sit in the same binary medium that already
accounts for $\Lambda_{\rm eff}$ and halo phenomenology. We discuss qualitative
connections to no-hair theorems and scalarized black holes, and close with a
road map of open problems and future directions for the Z$_2$ black-hole
sector.
\end{abstract}

%======================================================================
\section{Introduction and motivation}
\label{sec:intro}
%======================================================================

In standard general relativity (GR), static, asymptotically flat black holes
are remarkably simple objects. Under broad assumptions --- locality, canonical
matter fields, suitable energy conditions --- no-hair theorems state that
stationary black holes are completely characterised by a small set of
parameters (mass, charge, angular momentum). Scalar fields are typically
either expelled or forced into trivial configurations, leaving the Kerr--Newman
family as the generic endpoint.

At the same time, observations across cosmology and galaxy formation suggest
that the Universe is pervaded by a dark sector whose dominant manifestations
are:
\begin{itemize}[nosep]
  \item an almost homogeneous dark-energy-like component driving late-time
        acceleration;
  \item inhomogeneous, halo-like mass distributions that source galaxy
        rotation curves and cluster dynamics.
\end{itemize}
In the conventional $\Lambda$CDM parametrisation these two pieces are inserted
by hand: a bare cosmological constant $\Lambda$ in the Einstein equations and
a cold dark matter (CDM) fluid. Black holes then live inside this dark sector
but do not carry any structural information about it.

In the strict PhaseGeometry Z$_2$ programme the dark sector is instead realised
by a \emph{single} real scalar field with a Z$_2$-symmetric quartic potential.
The homogeneous broken phase of this field behaves as an effective cosmological
constant, while defects and inhomogeneities support halo-like profiles with
approximately $1/r^2$ energy-density tails over an intermediate radial range.
This provides a unified, strictly minimal picture: dark energy and
dark-matter--like haloes are different regimes of the \emph{same} binary phase
medium rather than two unrelated substances.

The goal of the present note is to extend this same Z$_2$ phase medium to
static black holes. We ask: what happens when a black hole is placed in the
binary vacuum background of PhaseGeometry Z$_2$? Can the scalar develop
non-trivial, static configurations around the horizon while remaining within a
strictly minimal, canonical setup? And if so, how do these configurations
relate to the halo profiles already explored in the Z$_2$ defect sector?

We will show, at the level of controlled analytical regimes, that:
\begin{enumerate}[nosep]
  \item in a weak-hair regime the scalar field acquires a small radial
        deviation from the homogeneous vacuum $v$ on a Schwarzschild--de~Sitter
        background, generating an excess energy density that behaves like a
        compact halo attached to the black hole;
  \item in a defect-like regime the binary structure of the potential allows
        kink-like configurations in which $\phi$ interpolates between $\pm v$
        in the near-horizon region, effectively dressing the black hole with a
        domain-wall-like shell;
  \item in both cases, the resulting structures are continuous deformations of
        the Z$_2$ halo phenomenology developed in the Phenomenology~II note,
        now transplanted into the strong-gravity regime.
\end{enumerate}

The rest of the paper is organised as follows. In
Sec.~\ref{sec:framework} we briefly recall the strict Z$_2$ PhaseGeometry
framework and fix notation. In Sec.~\ref{sec:setup} we introduce the static
spherically symmetric ansatz for the metric and scalar field and derive the
relevant field equations. Sections~\ref{sec:regimeI} and~\ref{sec:regimeII}
discuss, respectively, the weak-hair and defect-like regimes. In
Sec.~\ref{sec:comparison} we compare the resulting energy-density profiles with
Z$_2$ halo phenomenology. Finally, Sec.~\ref{sec:roadmap} provides a conceptual
road map and outlines open problems for the Z$_2$ black-hole sector.

%======================================================================
\section{Strict Z$_2$ PhaseGeometry framework in brief}
\label{sec:framework}
%======================================================================

\subsection{Action, potential and stress--energy}

We work with a single real scalar field $\phi$ minimally coupled to gravity,
with effective action
\begin{equation}
  S[g,\phi]
  = \int d^4x\,\sqrt{-g}\,\left[
      \frac{1}{2} g^{\mu\nu}\partial_\mu\phi\,\partial_\nu\phi
      - V(\phi)
      - \frac{1}{16\pi G_0} R
    \right]
  + S_{\rm ordinary}[g,\text{matter}],
  \label{eq:action}
\end{equation}
where $G_0$ is a fixed gravitational coupling and $S_{\rm ordinary}$ denotes
all non-scalar matter sectors (which we set to zero for most of this note).

The scalar potential is taken in the minimal Z$_2$-symmetric quartic form
\begin{equation}
  V(\phi) = \alpha\,\phi^2 + \frac{\beta}{2}\phi^4,
  \qquad \beta>0.
  \label{eq:potential}
\end{equation}
The point $\phi=0$ is distinguished and the sign of $\alpha$ selects the phase:
\begin{itemize}[nosep]
  \item for $\alpha>0$, the minimum lies at $\phi=0$ (unbroken symmetric phase);
  \item for $\alpha<0$, the potential develops two degenerate minima at
        \begin{equation}
          \phi = \pm v,
          \qquad
          v = \sqrt{-\alpha/\beta},
        \end{equation}
        corresponding to a broken binary phase.
\end{itemize}

The stress--energy tensor of the scalar field follows from the usual
definition,
\begin{equation}
  T^{(\phi)}_{\mu\nu}
  = \partial_\mu\phi\,\partial_\nu\phi
    - g_{\mu\nu}\left[
      \frac{1}{2} g^{\alpha\beta}\partial_\alpha\phi\,\partial_\beta\phi
      + V(\phi)
    \right],
  \label{eq:Tphi}
\end{equation}
and the Einstein equations read
\begin{equation}
  G_{\mu\nu}
  = 8\pi G_0\left(
      T^{(\phi)}_{\mu\nu}
      + T^{\rm(ordinary)}_{\mu\nu}
    \right).
  \label{eq:Einstein}
\end{equation}

\subsection{Vacuum and binary phase as a field}

In standard classical and quantum field theory, the electromagnetic and matter
fields are simply postulated to exist as continuous fields on space--time: even
when the field strengths vanish, one still assumes that a well-defined field
configuration is present at every point. The vacuum is not an empty container;
it is a particular \emph{state of a field}.

In the strict PhaseGeometry Z$_2$ programme we make an analogous, but more
economical move. Instead of postulating a collection of independent dark-sector
fields, we postulate a single real scalar \emph{order-parameter field}
$\phi(x)$ with a double-well, Z$_2$-symmetric potential. The mere assumption
that the Universe carries a global binary ``$+/-$'' structure already means
that space is filled by a binary medium: even when $\phi(x)$ is spatially
homogeneous and close to one of the minima $\phi\simeq\pm v$, there is a
well-defined field configuration at every point.

From this point of view, the vacuum is literally a binary phase background:
large regions in which $\phi$ sits near $+v$ or $-v$ provide an effective
vacuum energy density $V(\pm v)$, while kinks, domain walls and halo-like
profiles are just inhomogeneous configurations of the \emph{same} field. The
existence of a global Z$_2$ label is therefore not an extra decoration on top
of ordinary fields; in the strict PhaseGeometry Z$_2$ framework it \emph{is}
the content of a minimal order-parameter field $\phi(x)$ that underlies the
dark sector and branching structure of the medium.

In this sense, all macroscopic objects --- including matter distributions,
measuring devices and, in later notes, observers --- can in principle be
treated as metastable, redundantly encoded \emph{patterns} in the binary phase
medium. In the present note we will not exploit this interpretational layer in
detail, but it provides a useful backdrop for thinking about black holes as
structures sitting inside the same phase medium.

\subsection{Homogeneous broken phase and effective $\Lambda_{\rm eff}$}

In the broken phase ($\alpha<0$), a homogeneous configuration
\begin{equation}
  \phi(x) = \phi_{\rm vac} = \pm v,
  \qquad
  \partial_\mu\phi_{\rm vac} = 0,
\end{equation}
has stress--energy tensor
\begin{equation}
  T^{(\phi)}_{\mu\nu}[\phi_{\rm vac}]
  = - V(v)\,g_{\mu\nu},
\end{equation}
with
\begin{equation}
  V(v) = \alpha v^2 + \frac{\beta}{2}v^4
       = -\frac{\alpha^2}{2\beta}.
\end{equation}
This behaves as a cosmological-constant-like component with
\begin{equation}
  \rho_{\rm vac} = V(v),\qquad
  p_{\rm vac} = -\rho_{\rm vac},\qquad
  \Lambda_{\rm eff} = 8\pi G_0 V(v).
  \label{eq:Lambda_eff}
\end{equation}

In Phenomenology~I, the homogeneous broken phase was confronted with
cosmological age constraints and shown to be effectively indistinguishable from
a flat $\Lambda$CDM background for suitable parameter choices. In
Phenomenology~II, inhomogeneities in $\phi$ around $v$ produced positive
defect energy densities that behave as dark-matter--like haloes. In what
follows we reuse the same decomposition into a homogeneous vacuum and an excess
defect energy density, now in the context of black holes.

%======================================================================
\section{Static spherically symmetric GR + Z$_2$ scalar}
\label{sec:setup}
%======================================================================

\subsection{Metric ansatz and scalar configuration}

We consider a static, spherically symmetric space--time with line element
\begin{equation}
  ds^2
  = - e^{2\Phi(r)} f(r)\,dt^2
    + \frac{dr^2}{f(r)}
    + r^2 d\Omega^2,
  \label{eq:metric_ansatz}
\end{equation}
where $d\Omega^2$ is the metric on the unit 2-sphere, $\Phi(r)$ is a lapse
function and $f(r)$ encodes the mass profile and effective cosmological
constant. A convenient parametrisation of $f(r)$ is
\begin{equation}
  f(r) = 1 - \frac{2G_0 m(r)}{r} - \frac{\Lambda_{\rm eff}}{3} r^2,
  \label{eq:f_of_r}
\end{equation}
where $m(r)$ is a mass function that tends to a constant $M$ as $r\to\infty$
in the weak-hair regime but may receive contributions from the scalar field in
the interior.

We assume a static, spherically symmetric scalar configuration $\phi=\phi(r)$.
The non-vanishing components of $T^{(\phi)}_{\mu\nu}$ then read
\begin{align}
  T^{(\phi)t}{}_t
  &= -\rho_\phi(r)
   = -\left[
       \frac{1}{2} f(r)\,\phi'(r)^2 + V(\phi(r))
     \right],
     \label{eq:rho_phi}\\[4pt]
  T^{(\phi)r}{}_r
  &= p_r(r)
   = \frac{1}{2} f(r)\,\phi'(r)^2 - V(\phi(r)),\\[4pt]
  T^{(\phi)\theta}{}_\theta
  &= T^{(\phi)\varphi}{}_\varphi
   = p_t(r)
   = \frac{1}{2} f(r)\,\phi'(r)^2 - V(\phi(r)),
\end{align}
where prime denotes $d/dr$ and $(\rho_\phi,p_r,p_t)$ are the effective energy
density, radial pressure and tangential pressure of the scalar configuration.

\subsection{Einstein equations and scalar equation of motion}

With the ansatz \eqref{eq:metric_ansatz}--\eqref{eq:f_of_r}, the independent
Einstein equations reduce to
\begin{align}
  m'(r) &= 4\pi r^2 \rho_\phi(r),
  \label{eq:m_prime}\\[4pt]
  \Phi'(r)
  &= 4\pi G_0 r\,
     \frac{\rho_\phi(r)+p_r(r)}{f(r)}.
  \label{eq:Phi_prime}
\end{align}
The scalar equation of motion is
\begin{equation}
  \frac{1}{\sqrt{-g}} \partial_r\left(
     \sqrt{-g}\,g^{rr}\partial_r\phi
   \right)
  = \frac{dV}{d\phi},
\end{equation}
which, for the metric \eqref{eq:metric_ansatz}, becomes
\begin{equation}
  \phi''(r)
  + \left[
      \frac{2}{r} + \Phi'(r) + \frac{f'(r)}{f(r)}
    \right]\phi'(r)
  = \frac{1}{f(r)}\,\frac{dV}{d\phi}.
  \label{eq:scalar_eom}
\end{equation}

The system \eqref{eq:f_of_r}, \eqref{eq:m_prime}, \eqref{eq:Phi_prime} and
\eqref{eq:scalar_eom} defines a coupled Einstein--Klein--Gordon (EKG) problem
for $(m(r),\Phi(r),\phi(r))$.

\subsection{Vacuum background and defect energy density}

For $\phi(r)\equiv\phi_{\rm vac}=\pm v$ we recover the Schwarzschild--de~Sitter
solution,
\begin{equation}
  f_{\rm vac}(r) = 1 - \frac{2G_0 M}{r} - \frac{\Lambda_{\rm eff}}{3} r^2,
\end{equation}
with $\rho_\phi=V(v)$ and $m(r)=M+\frac{4\pi}{3}r^3 V(v)$; the latter term is
already accounted for in $\Lambda_{\rm eff}$. To isolate the contribution from
inhomogeneities, it is convenient to split
\begin{equation}
  \rho_\phi(r) = \rho_{\rm vac} + \rho_{\rm def}(r),
  \qquad
  \rho_{\rm vac} = V(v),
\end{equation}
and define the \emph{defect energy density}
\begin{equation}
  \rho_{\rm def}(r)
  \equiv \rho_\phi(r) - \rho_{\rm vac}
  = \frac{1}{2} f(r)\,\phi'(r)^2 + V(\phi(r)) - V(v).
  \label{eq:rho_def}
\end{equation}
The corresponding mass contribution is
\begin{equation}
  m_{\rm def}(r)
  = 4\pi \int_0^r \rho_{\rm def}(\tilde r)\,\tilde r^2\,d\tilde r,
\end{equation}
and we can write
\begin{equation}
  m(r) = M + \frac{4\pi}{3}r^3\rho_{\rm vac} + m_{\rm def}(r).
\end{equation}
In the weak-hair regime considered next, we will treat the metric as fixed
Schwarzschild--de~Sitter and compute $\rho_{\rm def}(r)$ on that background.

%======================================================================
\section{Regime I: weak binary phase hair}
\label{sec:regimeI}
%======================================================================

\subsection{Linearised scalar equation around the broken vacuum}

We first consider small deviations of $\phi(r)$ from the homogeneous broken
vacuum $v$ on a fixed Schwarzschild--de~Sitter background. We set
\begin{equation}
  \phi(r) = v + \delta\phi(r),
  \qquad
  |\delta\phi(r)|\ll v,
\end{equation}
and approximate the metric by
\begin{equation}
  ds^2
  = - f_{\rm vac}(r)\,dt^2 + \frac{dr^2}{f_{\rm vac}(r)} + r^2 d\Omega^2,
\end{equation}
with
\(
 f_{\rm vac}(r) = 1 - 2G_0 M/r - \Lambda_{\rm eff} r^2/3.
\)

Expanding the potential to quadratic order in $\delta\phi$,
\begin{equation}
  V(\phi)
  \simeq V(v) + \frac{1}{2} m_\phi^2 \delta\phi^2,
  \qquad
  m_\phi^2 \equiv V''(v) = -2\alpha > 0,
\end{equation}
the linearised equation for $\delta\phi(r)$ becomes
\begin{equation}
  \delta\phi''(r)
  + \left[
      \frac{2}{r} + \frac{f_{\rm vac}'(r)}{f_{\rm vac}(r)}
    \right]\delta\phi'(r)
  = \frac{m_\phi^2}{f_{\rm vac}(r)}\,\delta\phi(r).
  \label{eq:delta_phi_linear}
\end{equation}

This equation admits regular solutions at the event horizon $r=r_h$ and
decaying (or constant) solutions at large $r$. We impose regularity at the
horizon and require that $\delta\phi(r)\to 0$ as $r\to\infty$, corresponding
to the scalar relaxing to the broken vacuum at large distances.

\subsection{Energy density and mass profile}

In the linear regime, the defect energy density \eqref{eq:rho_def} simplifies
to
\begin{equation}
  \rho_{\rm def}(r)
  \simeq \frac{1}{2} f_{\rm vac}(r)\,\delta\phi'(r)^2
       + \frac{1}{2} m_\phi^2 \delta\phi(r)^2.
\end{equation}
Although a full analytic solution of \eqref{eq:delta_phi_linear} is not
available in closed form, its qualitative behaviour is clear:
\begin{itemize}[nosep]
  \item near the horizon, $\delta\phi(r)$ is regular and typically non-zero,
        leading to a finite $\rho_{\rm def}(r_h)$;
  \item at intermediate radii $r_h \ll r \ll m_\phi^{-1}$, the effective mass
        term is small and the solution behaves approximately as a massless
        mode on a Schwarzschild background, with $\delta\phi\sim A + B/r$;
  \item for $r\gg m_\phi^{-1}$, the mass term dominates and
        $\delta\phi(r)\propto e^{-m_\phi r}/r$, so $\rho_{\rm def}(r)$ decays
        exponentially.
\end{itemize}

In the intermediate region where $\delta\phi\sim B/r$ one finds
$\delta\phi'(r)\sim -B/r^2$ and hence
\begin{equation}
  \rho_{\rm def}(r)
  \sim \frac{B^2}{2}\,\frac{1}{r^4}
  \quad\text{(gradient-dominated regime)},
\end{equation}
while in mass-dominated regions
\(
 \rho_{\rm def}(r)\sim \tfrac12 m_\phi^2 \delta\phi^2
\)
tracks the exponential tail. The associated mass profile
\(
 m_{\rm def}(r) = 4\pi\int_0^r \rho_{\rm def}(\tilde r)\,\tilde r^2 d\tilde r
\)
then grows most efficiently in the near-horizon and intermediate region,
leading to a compact halo-like enhancement of the mass around the black hole.

A detailed numerical exploration of this regime is left for future work. For
the purposes of the present note, the key point is that even a small deviation
of $\phi$ from $v$ in the near-horizon region produces a well-defined excess
energy density that can be interpreted as \emph{weak binary phase hair} and an
attached scalar halo.

%======================================================================
\section{Regime II: defect-like phase structures near the horizon}
\label{sec:regimeII}
%======================================================================

\subsection{Kink-like configurations and domain-wall shells}

The quartic potential \eqref{eq:potential} admits topologically non-trivial
configurations in which $\phi$ interpolates between the two minima $\pm v$.
In flat space, the simplest example is a one-dimensional kink solution
$\phi(z)$ that connects $-v$ at $z\to-\infty$ to $+v$ at $z\to+\infty$. In the
present, spherically symmetric context, we can consider analogous configurations
in which $\phi(r)$ passes through $\phi\simeq 0$ in a narrow radial layer near
the horizon.

Qualitatively, such a configuration looks like:
\begin{equation}
  \phi(r)
  \simeq
  \begin{cases}
    -v, & r < r_{\rm shell},\\[2pt]
    \text{kink-like transition}, & r_{\rm shell}\lesssim r \lesssim r_{\rm shell}+\Delta r,\\[2pt]
    +v, & r > r_{\rm shell}+\Delta r,
  \end{cases}
\end{equation}
with $r_{\rm shell}$ of order the horizon radius $r_h$ and thickness
$\Delta r \ll r_h$ in the thin-shell limit. The energy density is then strongly
peaked in the transition region, forming a domain-wall-like shell that dresses
the black hole.

In a thin-wall approximation one can define an effective surface tension
\begin{equation}
  \sigma
  \sim \int_{\rm shell} \left[
          \frac12 f(r)\,\phi'(r)^2 + V(\phi(r)) - V(v)
        \right] dr,
\end{equation}
and treat the shell as a gravitating layer in the spirit of Israel junction
conditions. A full treatment requires solving the coupled EKG system with
appropriate boundary conditions at the horizon and across the shell; here we
only sketch the structure of the solutions.

\subsection{Near-horizon structure and regularity}

Regularity at the event horizon imposes constraints on $\phi(r)$ and its
derivatives. In Eddington--Finkelstein coordinates the horizon is a smooth
null surface, and the scalar field must remain finite there. For kink-like
configurations this means that the transition region cannot coincide exactly
with the horizon but must straddle it, with $\phi$ finite and differentiable
across $r=r_h$.

The resulting picture is that of a black hole embedded in a Z$_2$ medium where
the binary phase structure is locally rearranged in the near-horizon region.
From the outside, the geometry is modified by the additional mass and stress of
the shell; from the inside, the scalar configuration may encode non-trivial
phase information that is hidden behind the horizon. Exploring the full space
of such solutions and their stability properties is an important open problem.

%======================================================================
\section{Comparison with Z$_2$ halo phenomenology}
\label{sec:comparison}
%======================================================================

In Phenomenology~II, halo-like configurations of the Z$_2$ field were studied
in a weak-gravity, approximately Newtonian regime. There, static defects and
smooth inhomogeneities in $\phi$ produced energy-density profiles $\rho_\phi(r)$
with extended $1/r^2$ tails over an intermediate radial range, naturally
supporting flat rotation curves.

The black-hole configurations considered here can be viewed as the strong-field
counterparts of those haloes:
\begin{itemize}[nosep]
  \item in the weak-hair regime, the black hole sits at the centre of a
        compact scalar halo whose energy density is concentrated near the
        horizon and decays away from it;
  \item in the defect-like regime, the black hole is dressed by a domain-wall
        shell where the binary phase changes branch, analogous to a spherical
        defect embedded in the medium.
\end{itemize}

In both cases, the underlying mechanism is the same: the Z$_2$ order parameter
$\phi$ deviates from the homogeneous vacuum $v$ and the excess energy density
$\rho_{\rm def}(r)$ acts as an effective dark-matter--like component. The
differences are quantitative rather than qualitative and are driven by the
strong curvature near the horizon and the boundary conditions required by
regularity.

A natural next step is to connect the phenomenology of phase-dressed black
holes to astrophysical observables:
\begin{itemize}[nosep]
  \item modifications of the near-horizon geometry and their impact on black
        hole shadows and lensing;
  \item possible shifts in quasi-normal mode spectra due to the scalar halo or
        shell;
  \item the interplay between large-scale Z$_2$ haloes and local
        phase-dressed black holes residing inside them.
\end{itemize}
These questions sit at the intersection of black-hole physics and the unified
dark sector of PhaseGeometry Z$_2$.

%======================================================================
\section{Road map and conceptual remarks}
\label{sec:roadmap}
%======================================================================

This section plays the rôle of a conceptual road map for the Z$_2$ black-hole
sector and loosely replaces a separate ``BH Road Map'' note. We summarise the
main messages and outline concrete directions for future work.

\subsection{Key take--home points}

\begin{itemize}[nosep]
  \item The \emph{same} real scalar with a broken binary (Z$_2$) phase that
        already generates an effective $\Lambda_{\rm eff}$ and supports
        dark-matter--like haloes is now used, unchanged, to dress static black
        holes.
  \item Black holes are no longer ``holes in empty spacetime'' but structures
        embedded in, and interacting with, a \emph{binary phase medium} that
        carries both dark energy and halo structure.
  \item In this medium, a static black hole naturally acquires weak binary
        phase hair and, in more nonlinear regimes, defect-like shells whose
        energy density can be compared directly to phenomenological halo
        profiles.
\end{itemize}

\subsection{From Z$_2$ dark energy and haloes to Z$_2$ black holes}

Within the broader PhaseGeometry Z$_2$ series, the black-hole sector is the
next logical layer after:
\begin{itemize}[nosep]
  \item the homogeneous broken phase (Phenomenology~I), which fixes
        $\Lambda_{\rm eff}$ and the background cosmology;
  \item the static defect/halo sector (Phenomenology~II), which uses
        inhomogeneities in $\phi$ to generate halo-like mass profiles.
\end{itemize}
Placing a black hole into this same medium is then not a new model but a
consistency check: can the already-specified Z$_2$ medium support static
black-hole configurations that fit naturally into the picture, without adding
new fields or couplings?

\subsection{From ``isolated'' black holes to phase-dressed black holes}

In most GR discussions, black holes are treated as isolated objects on
asymptotically flat or de~Sitter backgrounds. In PhaseGeometry Z$_2$ they are
instead immersed in a concrete phase medium. This invites a shift of language:
\begin{itemize}[nosep]
  \item rather than thinking of a black hole as an empty region of spacetime
        surrounded by vacuum, we think of it as a region where the phase medium
        is strongly curved and possibly reconfigured;
  \item binary phase hair and defect shells are then natural excitations of the
        same medium, not exotic add-ons.
\end{itemize}

\subsection{Relation to no-hair theorems and scalarized black holes}

The configurations discussed here do not violate the spirit of classic no-hair
theorems, but they do exploit ways in which their assumptions can be relaxed:
\begin{itemize}[nosep]
  \item the background is asymptotically de~Sitter rather than strictly flat,
        with $\Lambda_{\rm eff}$ inherited from the broken phase;
  \item the scalar potential is symmetry breaking and supports domain-wall-like
        defects already present in the cosmological solution;
  \item the focus is on controlled deformations of a pre-existing halo/medium
        within which the black hole sits.
\end{itemize}
The resulting objects are distinct from the usual ``scalarized black holes'' in
scalar--tensor theories, where new non-minimal couplings are introduced and
hair typically appears only beyond certain thresholds. Here the scalar is the
same order parameter that also sources dark energy and haloes; the main new
ingredient is the strong-field environment provided by the horizon.

\subsection{What happens to light and matter?}

A full analysis of light propagation and matter dynamics in phase-dressed
backgrounds is beyond the scope of this note, but several qualitative remarks
are immediate:
\begin{itemize}[nosep]
  \item photons and test particles follow geodesics of the dressed metric, so
        any modification of $f(r)$ and $\Phi(r)$ by the scalar configuration
        can leave imprints on lensing, shadows and orbital dynamics;
  \item in the defect-like regime, the domain-wall shell provides a localized
        region of high energy density and pressure that may affect accretion
        flows and the structure of any surrounding disk;
  \item fluctuations of the scalar field around its background configuration
        could, in principle, participate in quasi-normal modes and ringdown
        signals, offering another potential observational window.
\end{itemize}

\subsection{Priority road map for the Z$_2$ black-hole sector}

For practical work it is useful to list a few concrete tasks, roughly ordered
by priority:

\medskip
\noindent\textbf{(***)
Fully coupled static solutions.} Solve the coupled EKG system
\eqref{eq:m_prime}--\eqref{eq:scalar_eom} numerically for both weak-hair and
defect-like regimes, imposing regularity at the horizon and appropriate
asymptotics. Map out the space of solutions and their dependence on
$(\alpha,\beta,G_0)$.

\medskip
\noindent\textbf{(**)
Rotating phase-dressed black holes.} Extend the analysis to stationary,
axisymmetric backgrounds (Kerr--de~Sitter with a Z$_2$ scalar). Determine
whether binary phase hair and defect-like structures persist in the rotating
case and how they affect frame dragging and ergoregions.

\medskip
\noindent\textbf{(**)
Stability and quasi-normal modes.} Study linearised perturbations of the
phase-dressed solutions and compute quasi-normal modes, looking for distinctive
signatures of the Z$_2$ medium in gravitational-wave signals.

\medskip
\noindent\textbf{(*)
Dynamical formation scenarios.} Explore, at least in toy models, whether
realistic collapse or accretion scenarios in a Z$_2$ medium naturally lead to
phase-dressed black holes and on what timescales the scalar halo or shell
forms and relaxes.

\medskip
\noindent\textbf{(*)
Connections to the broader ``unity field'' picture.} In the wider
PhaseGeometry programme, the Z$_2$ medium is one layer of a multi-tiered
framework that also includes phase-fibre and device-oriented sectors.
Integrating the black-hole results into that larger picture --- for example,
via analogies with superconducting devices or 5D phase-fibre constructions ---
is an open conceptual direction.

\bigskip

The present Phenomenology~III note should therefore be read as a structural and
conceptual bridge: it translates the existing Z$_2$ cosmological and halo
framework into the strong-gravity regime of black holes, without altering the
underlying strict Z$_2$ physics. The technical regimes outlined here are
deliberately simple, providing a starting point for more detailed analytical
and numerical studies.



%=========================================================
\section*{Next steps and Core Package structure}
%=========================================================

This note is part of the PhaseGeometry Z$_2$ Core Package v2.5 (strict). Together
with the other notes listed below it defines a minimal binary-phase framework
for dark energy, dark-matter–like haloes, black holes and quantum branching in
a single Z$_2$ medium. The Core Package is archived as a bundled record under
Zenodo DOI\,10.5281/zenodo.17807433.

The current structure of the Core Package is:

\begin{itemize}
  \item \textbf{Foundations I -- Technical Passport: Minimal Binary Phase Model for Dark Sector and Gravity}\\
        Defines the strict Z$_2$ action $S[g,\phi]$, potential $V(\phi)$, phase
        structure and basic ontology of the binary phase medium.
  \item \textbf{Foundations II -- Lambda from Broken Phase}\\
        Derives how the homogeneous broken phase $\phi\simeq\pm v$ acts as an
        effective cosmological constant $\Lambda_{\rm eff}=8\pi G_0 V(v)$.
  \item \textbf{Foundations III -- Dark Matter from Defects}\\
        Develops the defect/inhomogeneous sector and defines the excess energy
        density $\rho_\phi(r)$ as a dark-matter–like component.
  \item \textbf{Foundations IV -- Phase medium, observer and branching histories}\\
        Interprets the Z$_2$ field as a phase medium hosting observers and
        classical branches in an Everett–Zurek picture.
  \item \textbf{Foundations V -- Decoherence and Quantum Darwinism in a binary phase medium}\\
        Implements decoherence and Quantum Darwinism explicitly in finite Z$_2$
        chains, with redundant records stored in environmental fragments.
  \item \textbf{Phenomenology I -- Background cosmology and the age of the Universe}\\
        Confronts the homogeneous broken phase with FRW background evolution,
        distance–redshift relations and cosmic age constraints.
  \item \textbf{Phenomenology II -- Dark-matter–like haloes from defects}\\
        Studies static halo profiles supported by Z$_2$ defects and compares
        them with rotation-curve phenomenology.
  \item \textbf{Phenomenology III -- Static black holes with binary phase hair}\\
        Embeds black holes into the same binary phase medium and explores weak
        scalar hair and defect-like shells in the strong-gravity regime.
\end{itemize}

Across these layers, the same strict Z$_2$ order parameter $\phi$ underlies:
(i) the effective cosmological constant from the homogeneous broken phase,
(ii) dark-matter–like haloes from defects and inhomogeneities,
(iii) phase-dressed black holes, and
(iv) quantum branching and classical records realised as patterns in the binary medium.

The present note should be read as one component of this unified picture. It is
designed to be technically self-contained, but its full meaning emerges when
combined with the other items in the PhaseGeometry Z$_2$ Core Package v2.5 (strict).


\end{document}

% PhaseGeometry_PhenomenologyII_HaloesAndRotationCurves.tex
\documentclass[11pt]{article}

\usepackage[T1]{fontenc}
\usepackage[utf8]{inputenc}
\usepackage[english]{babel}

\usepackage[a4paper,margin=1in]{geometry}
\usepackage{amsmath,amssymb}
\usepackage{bm}
\usepackage{graphicx}
\usepackage{float}        % for [H]
\usepackage{placeins}     % for \FloatBarrier
\usepackage{hyperref}

% Larger, custom-styled abstract
\renewenvironment{abstract}
  {\par\bigskip
   \begin{center}
     \bfseries\Large Abstract
   \end{center}
   \begin{quote}
   \large
  }
  {\end{quote}\par\bigskip}

\title{%
  {\large PhaseGeometry Z$_2$ Phenomenology II\\[2pt]}
  {\Large\boldmath Dark-matter-like haloes from defects\\[4pt]}
}

\author{Aleksey Turchanov}
\date{December 2025}

\begin{document}

\maketitle

\begin{center}
\scriptsize
This note is part of the PhaseGeometry Z$_2$ Core Package v2.5 (strict).\\
Licensed under Creative Commons Attribution 4.0 International (CC BY 4.0).\\
Zenodo record: DOI\,10.5281/zenodo.17807433.
\end{center}

\vspace{0.5em}






\begin{abstract}
Within the minimal PhaseGeometry framework --- a single real order parameter
$\phi$ with a binary $Z_2$ symmetry and quartic potential
$V(\phi) = \alpha \phi^2 + \frac{\beta}{2}\phi^4$, coupled minimally to gravity via
the action $S[g,\phi]$ --- the homogeneous broken phase generates an effective
cosmological constant $\Lambda_{\rm eff}$, while inhomogeneities and defects in the
broken phase contribute as a dark-matter-like component. Building on the
\emph{Technical Passport}, the \emph{LambdaFromBrokenPhase} note and the
\emph{DarkMatterFromDefects} foundations, we develop a phenomenological description
of static, spherically symmetric defect-supported haloes. We derive the effective
halo energy density $\rho_\phi(r)$ and enclosed mass profile $M_\phi(r)$ for
configurations $\phi=\phi(r)$, and show explicitly how an intermediate radial range
with $\rho_\phi(r)\propto r^{-2}$ can emerge, leading to quasi-flat rotation curves
$v_c(r)\simeq \text{const}$. We then propose a simple analytic ansatz for $\phi(r)$
that realizes this behaviour and outline how to match the resulting analytic rotation
curves to spiral galaxy data in a way that remains faithful to the underlying
PhaseGeometry structure. The defect contribution is clearly separated from the
homogeneous $\Lambda_{\rm eff}$ background at the level of the Einstein equations,
so that both dark-energy-like and dark-matter-like effects arise from the same
order parameter. We illustrate the construction by comparing with three SPARC
rotation curves (DDO154, NGC2403 and NGC3198) using a minimal two-parameter
mapping between the dimensionless defect halo and physical units.
\end{abstract}

\pagebreak
\tableofcontents
\pagebreak
%======================================================================
\section{Introduction}
%======================================================================

The PhaseGeometry framework postulates a single pre-geometric field whose effective
order parameter $\phi(x)$ carries a binary $Z_2$ symmetry, $\phi\to-\phi$, about a
distinguished state $\phi=0$. Different phases and configurations of this order
parameter are interpreted as the origin of dark energy, dark matter and gravity
within one unified structure, rather than as independent sectors added by hand.

In the minimal realization, the order parameter is a real scalar field with quartic
potential
\begin{equation}
  V(\phi) = \alpha \phi^2 + \frac{\beta}{2}\phi^4, 
  \qquad \beta>0,
\end{equation}
coupled to an emergent metric $g_{\mu\nu}$ via the action $S[g,\phi]$. For
$\alpha<0$ the potential admits two degenerate minima at $\phi=\pm v$ with
\begin{equation}
  v=\sqrt{-\alpha/\beta},
\end{equation}
corresponding to two symmetry-broken branches of the same binary structure.

In the homogeneous broken phase, $\phi(x)\simeq \pm v$ with negligible gradients,
the potential energy $V(v)$ acts as a vacuum energy density
$\rho_{\rm vac}=V(v)$ and induces an effective cosmological constant
$\Lambda_{\rm eff}=8\pi G_0\rho_{\rm vac}$. Inhomogeneities and defects in the
broken phase, where $\phi$ deviates from $\pm v$ and develops spatial gradients,
contribute additional positive energy density that couples gravitationally but does
not couple directly to light. This contribution is interpreted as a
dark-matter-like component arising from the same order parameter that sources
$\Lambda_{\rm eff}$.

Within this picture the dark sector is no longer represented by two unrelated
ingredients --- a bare cosmological constant and a separate cold dark matter
fluid --- but instead emerges from a single $Z_2$-symmetric order parameter.
The homogeneous broken phase accounts for the effective cosmological constant,
while defect structures in the same phase play the role of dark-matter-like
haloes on galactic and larger scales.

\paragraph{This work.}
The present paper develops the phenomenology of such defect-supported haloes on
galactic scales. Our goal is twofold. First, starting from the minimal
PhaseGeometry setup we derive the energy density and mass profiles associated
with static, spherically symmetric configurations $\phi=\phi(r)$ and show that
an intermediate radial regime with $\rho_\phi(r)\propto 1/r^2$ arises naturally,
leading to quasi-flat rotation curves. Second, we confront this mechanism
directly with data by constructing a simple analytic ansatz for $\phi(r)$ and
applying it to SPARC rotation curves \cite{Lelli:2016}. The underlying
dimensionless defect profile is fixed once and for all; for each galaxy only two
phenomenological scaling parameters, a characteristic radius $R_0$ and velocity
scale $V_0$, are fitted, while the baryonic contribution is taken from the SPARC
mass models without re-adjusting their mass-to-light ratios.

In more detail, we first introduce a minimal analytic ansatz for the defect profile
that faithfully reproduces the qualitative behavior of the exact PhaseGeometry
solutions in the halo regime. We then use this ansatz to build a defect-induced
circular velocity profile $V_\phi(R)$ and study its properties, emphasizing the
emergent $1/r^2$ density tail and the associated flat rotation curves. Finally,
we illustrate the framework by fitting the combined baryonic+defect model to
three benchmark SPARC galaxies and by constructing a simple summary statistic
for a small heterogeneous SPARC subsample, thereby demonstrating that the same
fixed defect profile with galaxy-dependent $(R_0,V_0)$ can reproduce the
correct order of magnitude and shape of galactic rotation curves over more than
a decade in $V_{\rm flat}$.

\paragraph{Structure of the paper.}
The paper is organised as follows. In Sec.~2 we briefly review the observational
halo problem and the standard CDM interpretation. In Sec.~3 we summarise the
minimal PhaseGeometry framework, including the effective cosmological constant
from the homogeneous broken phase. In Sec.~4 we introduce static, spherically
symmetric defect configurations, and in Sec.~5 we derive the corresponding
defect energy density and mass profiles, highlighting the emergence of an
intermediate $\rho_\phi\propto r^{-2}$ regime. In Sec.~6 we construct analytic
rotation curves based on a simple defect ansatz, and in Sec.~7 we confront the
resulting baryonic+defect model with SPARC rotation curve data. Sec.~8 contains
our discussion and outlook. Appendix~A summarises the numerical implementation
and collects illustrative profiles and rotation curves.
\\
\\
\\


%======================================================================
\section{Galaxy rotation curves and the halo problem}
%======================================================================

\subsection{Observed rotation curves}

Neutral hydrogen and stellar kinematics in spiral galaxies reveal rotation curves
$v_c(r)$ that rise roughly linearly in the inner regions, flatten over a wide radial
range, and only mildly decline (if at all) at large radii.
In a purely baryonic, Newtonian picture with luminous mass concentrated in the
stellar disc and bulge, one would expect $v_c(r)\propto r$ in the very inner region
and a Keplerian decline $v_c(r)\propto r^{-1/2}$ outside the bulk of the baryonic
mass. The observed persistence of nearly flat rotation curves out to tens of
kiloparsecs is the classic \emph{halo problem}.

This pattern is usually encoded phenomenologically in terms of a spherically
symmetric halo density profile $\rho_{\rm DM}(r)$ that behaves roughly as
\begin{equation}
  \rho_{\rm DM}(r) \propto
  \begin{cases}
    \text{const}, & r \ll r_{\rm core},\\[2pt]
    r^{-2}, & r_{\rm core} \ll r \ll r_{\rm vir},\\[2pt]
    \text{fast fall-off}, & r\gg r_{\rm vir},
  \end{cases}
\end{equation}
so that the enclosed mass grows as $M(r)\propto r$ in the intermediate region and
$v_c(r)\simeq \text{const}$ there.

\subsection{Standard CDM picture (very briefly)}

In the standard cold dark matter (CDM) paradigm, the halo is composed of
non-relativistic particles that interact weakly with ordinary matter and with
themselves, clustering gravitationally. Cosmological $N$-body simulations in this
framework typically lead to halo density profiles such as NFW or Einasto, which can
reproduce many qualitative features of observed rotation curves when combined with a
baryonic disc and bulge
\cite{Navarro:1996, SofueRubin:2001, Rubin:1980}.

However, CDM treats dark matter as an independent component, with its own particle
ontology and parameter space, separate from dark energy and gravity. The
PhaseGeometry programme instead explores the possibility that dark energy, dark
matter and gravity are different manifestations of the same underlying
binary-symmetry structure, so that what is usually called ``dark matter'' is not a
new species but the energy stored in defects and inhomogeneities of the same order
parameter that generates $\Lambda_{\rm eff}$.

\subsection{Motivation for defect-supported haloes in PhaseGeometry}

Within the PhaseGeometry framework, the homogeneous broken phase naturally gives
rise to an effective cosmological constant $\Lambda_{\rm eff}$, while the
inhomogeneous broken phase --- with domains, defects and kink-like transitions
between $\phi\simeq -v$ and $\phi\simeq +v$ --- carries additional energy density
through both potential and gradient contributions. This energy:
\begin{itemize}
  \item is positive (or at least non-negative) for field configurations close to the
        minima,
  \item couples only gravitationally to ordinary matter and light,
  \item can be distributed in extended halo-like configurations,
  \item and is controlled by the same potential $V(\phi)$ and gravitational coupling
        $G_0$ that determine $\Lambda_{\rm eff}$.
\end{itemize}
It is therefore natural to ask whether static or quasi-static defect configurations
can support halo mass profiles that reproduce the main phenomenological features of
spiral galaxy rotation curves, at least at the level of a toy model, without
introducing any new fields or couplings beyond those already present in $S[g,\phi]$.
\\
\\
\\
%======================================================================
\section{Brief overview of the PhaseGeometry framework}
%======================================================================

\subsection{Minimal recap of the action $S[g,\phi]$ and potential $V(\phi)$}

The minimal PhaseGeometry ``technical passport'' specifies the effective action for
the order parameter and metric as
\begin{equation}
  S[g,\phi] = \int d^4x\,\sqrt{-g}\,\left[
    \frac{1}{2} g^{\mu\nu}\partial_\mu\phi\,\partial_\nu\phi
    - V(\phi)
    - \frac{1}{16\pi G_0} R
  \right],
  \label{eq:action}
\end{equation}
with quartic potential
\begin{equation}
  V(\phi) = \alpha \phi^2 + \frac{\beta}{2}\phi^4,
  \qquad \beta>0.
  \label{eq:potential}
\end{equation}
The symmetry is a binary
\begin{equation}
  \phi \to -\phi,
\end{equation}
with $\phi=0$ representing a distinguished symmetric state.

Varying the action \eqref{eq:action} with respect to $g^{\mu\nu}$ yields the Einstein
equations
\begin{equation}
  G_{\mu\nu} = 8\pi G_0\left(T^{(\phi)}_{\mu\nu} +
  T^{(\rm ordinary)}_{\mu\nu}\right),
\end{equation}
where the stress--energy tensor of $\phi$ is
\begin{equation}
  T^{(\phi)}_{\mu\nu}
  = \partial_\mu\phi\,\partial_\nu\phi
    - g_{\mu\nu}\left(
    \frac{1}{2} g^{\alpha\beta}\partial_\alpha\phi\,\partial_\beta\phi
    + V(\phi)
  \right).
  \label{eq:Tphi}
\end{equation}

In this Phenomenology~II note we adopt the strict PhaseGeometry Z$_2$
viewpoint: the scalar field $\phi(x)$ entering \eqref{eq:action} is
interpreted as a real order--parameter field for a binary medium with a
global ``$+/-$'' structure. Even when $\phi(x)$ is spatially homogeneous
and close to one of the symmetry-broken minima, space is filled by a
well-defined field configuration at every point. Large domains in which
$\phi$ sits near a broken minimum realise a binary vacuum background,
while inhomogeneous departures from this background will be interpreted
below as defect configurations sourcing dark-matter-like haloes.

\subsection{Phases: $\phi\simeq 0$ and $\phi\simeq \pm v$}

The potential \eqref{eq:potential} has the following phases:
\begin{itemize}
  \item For $\alpha>0$, the minimum is at $\phi=0$; the $Z_2$ symmetry is unbroken,
        and this regime is interpreted as a symmetric or pre-geometric phase.
  \item For $\alpha<0$, the potential develops two degenerate minima at
        \begin{equation}
          \phi = \pm v, \qquad v = \sqrt{-\alpha/\beta},
        \end{equation}
        corresponding to a symmetry-broken phase in which one of the two branches is
        selected.
\end{itemize}

On cosmological scales, the Universe is assumed to reside in a homogeneous broken
phase, $\phi(x)\simeq \phi_{\rm vac} = \pm v$ with $\partial_\mu\phi_{\rm vac}\simeq 0$.
On smaller (galactic and cluster) scales, inhomogeneities and defects can develop,
with domains of $\phi\simeq +v$ and $\phi\simeq -v$ separated by transition regions
where $\phi$ passes through $\phi\simeq 0$.

\subsection{Effective $\Lambda_{\rm eff}$ from $V(v)$}

For a homogeneous broken-phase configuration
\begin{equation}
  \phi(x) = \phi_{\rm vac} = \pm v, \qquad \partial_\mu\phi_{\rm vac} = 0,
\end{equation}
the stress--energy tensor \eqref{eq:Tphi} reduces to
\begin{equation}
  T^{(\phi)}_{\mu\nu}[\phi_{\rm vac}]
  = - V(v)\,g_{\mu\nu},
\end{equation}
which is equivalent to a perfect fluid with energy density
\begin{equation}
  \rho_{\rm vac} \equiv V(v),
\end{equation}
and pressure $p_{\rm vac}=-\rho_{\rm vac}$, i.e.\ an equation of state $w=-1$.

Inserting this into the Einstein equations and moving the vacuum term to the
left-hand side,
\begin{equation}
  G_{\mu\nu} + \Lambda_{\rm eff} g_{\mu\nu}
  = 8\pi G_0\,T^{(\rm ordinary)}_{\mu\nu},
\end{equation}
defines an effective cosmological constant
\begin{equation}
  \Lambda_{\rm eff} = 8\pi G_0\,\rho_{\rm vac}
  = 8\pi G_0\,V(v).
  \label{eq:Lambdaeff}
\end{equation}
In this minimal setup, no explicit cosmological constant term is introduced in the
gravitational sector; $\Lambda_{\rm eff}$ is entirely determined by the parameters
$(\alpha,\beta)$ and the choice of branch.

\subsection{Order-of-magnitude consistency with $\Lambda_{\rm eff}$}

Since we are using a single $Z_2$ scalar both to generate an effective
cosmological constant and to support galactic haloes, it is important to check
that there is no obvious order-of-magnitude conflict between these two roles.

Consider a simple quartic double-well potential
\[
  V(\phi) = \frac{\lambda}{4}\,(\phi^2 - v^2)^2 + \text{const},
\]
with the constant chosen so that the vacuum energy in the broken minimum
$\phi = \pm v$ matches the observed dark-energy density,
\[
  \rho_\Lambda \simeq V(v) \sim \frac{\lambda v^4}{4}.
\]
In this setup the mass of small excitations around the minimum is
\[
  m_\phi^2 \simeq 2 \lambda v^2.
\]

Reproducing the observed vacuum energy $\rho_\Lambda \sim (2.3~\text{meV})^4$
with a sub-Planckian vacuum expectation value,
for example $v \sim 10^{-3} M_{\rm Pl}$, implies an extremely small quartic
coupling,
\[
  \lambda \sim \frac{4\rho_\Lambda}{v^4} \sim 10^{-110},
\]
and an ultralight scalar mass,
\[
  m_\phi \sim \sqrt{2\lambda}\,v \sim 10^{-30}~\text{eV},
\]
corresponding to a Compton wavelength of order Mpc.

Independently of this, the defect-supported halo profiles considered in this
paper yield approximately flat rotation curves with an asymptotic velocity
of the form
\[
  v_{\rm flat} \sim \mathcal{O}(1)\,\frac{v}{M_{\rm Pl}}\,c,
\]
where the $\mathcal{O}(1)$ factor depends only on the detailed defect profile.
Taking again $v \sim 10^{-3} M_{\rm Pl}$ one obtains
\[
  v_{\rm flat} \sim 10^{-3} c,
\]
which is consistent with typical galactic values
$\sim 100$–$300~\text{km/s}$ without any additional tuning of particle-physics
parameters beyond those already required by $\rho_\Lambda$.

Thus, at the level of order-of-magnitude estimates, there exists a broad
parameter window in which a single $Z_2$ scalar field with a quartic
double-well potential can simultaneously reproduce the observed vacuum
energy density and the characteristic scale of galactic flat rotation
velocities. A full joint exploration of the $(\lambda, v, G_0)$ (or
equivalently $(\alpha,\beta,G_0)$) parameter space, combining background
and halo constraints in a systematic way, is left for future work.
Here we restrict ourselves to showing that the defect sector can reproduce
realistic halo profiles for phenomenologically reasonable macroscopic
scales $(R_0, V_0)$.

%======================================================================
\section{Static spherically symmetric defect configurations}
%======================================================================


\subsection{Ansatz $\phi=\phi(r)$ and metric}

To model halo configurations, we consider static, spherically symmetric solutions
of the coupled Einstein--scalar system. The most general static, spherically
symmetric metric can be written as
\begin{equation}
  ds^2 = - e^{2\Phi(r)} dt^2 + e^{2\Lambda(r)} dr^2 + r^2 d\Omega^2,
  \label{eq:metric}
\end{equation}
and we take the order parameter to depend only on the radial coordinate,
\begin{equation}
  \phi = \phi(r), \qquad \partial_t\phi = 0.
  \label{eq:phi-ansatz}
\end{equation}

We impose the broken-phase boundary condition at large radii,
\begin{equation}
  \phi(r)\to +v, \quad \phi'(r)\to 0 \quad \text{as } r\to\infty,
\end{equation}
and allow the field to interpolate between different values near the centre, e.g.\ a
kink-like profile with $\phi(0)\simeq -v$ and $\phi(\infty)=+v$.

\subsection{Field equation for $\phi(r)$}

Varying the action \eqref{eq:action} with respect to $\phi$ yields the scalar equation
of motion
\begin{equation}
  \Box\phi - \frac{dV}{d\phi} = 0.
\end{equation}
For the metric \eqref{eq:metric} and ansatz \eqref{eq:phi-ansatz}, this becomes
\begin{equation}
  \phi''(r) + \left(\Phi'(r)-\Lambda'(r) + \frac{2}{r}\right)\phi'(r)
  = \frac{dV}{d\phi}(\phi)
  = 2\alpha \phi + 2\beta \phi^3.
  \label{eq:full-spherical-eom}
\end{equation}

In the weak-field, quasi-Newtonian limit appropriate for galactic scales,
$|\Phi(r)|\ll1$ and $|\Lambda(r)|\ll1$, and their derivatives can be neglected in the
scalar equation at leading order. Under this approximation, \eqref{eq:full-spherical-eom}
reduces to the flat-space static spherically symmetric Klein--Gordon equation,
\begin{equation}
  \phi''(r) + \frac{2}{r}\phi'(r)
  = 2\alpha \phi + 2\beta \phi^3.
  \label{eq:flat-eom}
\end{equation}

\subsection{Boundary conditions: kink-like/domain-wall profiles}

The configurations of interest are static, finite-energy solutions of \eqref{eq:flat-eom}
with the following qualitative properties:
\begin{itemize}
  \item Regularity at the origin:
    \begin{equation}
      \phi'(0) = 0,\qquad |\phi(0)|<\infty.
    \end{equation}
  \item Broken-phase asymptotics:
    \begin{equation}
      \phi(r)\to +v, \qquad \phi'(r)\to 0
      \quad \text{as } r\to\infty.
    \end{equation}
  \item Monotonicity:
    \begin{equation}
      \phi(r) \text{ is monotone in $r$},
    \end{equation}
    interpolating between a central value $\phi_c\equiv\phi(0)$ and the outer broken
    phase. A simple example is a kink-like profile $\phi(0)\simeq -v$,
    $\phi(\infty)=+v$.
\end{itemize}

The full non-linear equation \eqref{eq:flat-eom} with these boundary conditions
generally does not admit a simple closed-form solution; numerical treatment is
expected to be necessary. For the purposes of this phenomenological paper we will
instead use controlled approximations and simple analytic ansatzes that capture the
relevant radial regimes.

\subsection{Dimensionless form}

It is often convenient to rescale to dimensionless variables. Let
\begin{equation}
  m_\phi^2 \equiv V''(v) > 0,
\end{equation}
be the mass-squared of small fluctuations around the broken minimum. In terms of the
quartic potential, $m_\phi^2=2\alpha+6\beta v^2$, which is positive for $\alpha<0$
and $\beta>0$.
Introduce dimensionless variables
\begin{equation}
  x \equiv m_\phi r, \qquad
  \chi(x) \equiv \frac{\phi(r)}{v},
\end{equation}
so that $x\sim 1$ corresponds to $r\sim m_\phi^{-1}$. In terms of $(x,\chi)$ the
equation \eqref{eq:flat-eom} can be written as
\begin{equation}
  \chi''(x) + \frac{2}{x}\,\chi'(x)
  = \frac{1}{m_\phi^2 v}\,\frac{dV}{d\phi}\bigl(\phi=v\chi\bigr)
  = \frac{2\alpha}{m_\phi^2}\,\chi
    + \frac{2\beta v^2}{m_\phi^2}\,\chi^3.
  \label{eq:chi-eom-explicit}
\end{equation}
This makes it clear that $m_\phi^{-1}$ sets the characteristic length scale beyond
which deviations from the broken minimum are exponentially suppressed.
\\
\\
\\
%======================================================================
\section{Effective halo energy density and mass profile}
%======================================================================

\subsection{Defect energy density relative to the homogeneous broken phase}

For a static configuration in a weakly curved background, the energy density measured
by static observers can be read from $T^{(\phi)0}{}_{0}$:
\begin{equation}
  T^{(\phi)0}{}_{0}[\phi(r)]
  \simeq \frac{1}{2}(\nabla\phi)^2 + V(\phi(r)),
\end{equation}
where $(\nabla\phi)^2 = \delta^{ij}\partial_i\phi\partial_j\phi$. For spherical
symmetry,
\begin{equation}
  (\nabla\phi)^2 = \left(\frac{d\phi}{dr}\right)^2
  \equiv \phi'(r)^2.
\end{equation}
In the homogeneous broken phase, $\phi=\phi_{\rm vac}=\pm v$ and
\begin{equation}
  T^{(\phi)0}{}_{0}[\phi_{\rm vac}] = V(v) \equiv \rho_{\rm vac}.
\end{equation}

We define the \emph{defect energy density} as the excess relative to the homogeneous
broken phase,
\begin{equation}
  \rho_\phi(r)
  \equiv T^{(\phi)0}{}_{0}[\phi(r)] - T^{(\phi)0}{}_{0}[\phi_{\rm vac}]
  \simeq \frac{1}{2}\phi'(r)^2 + V(\phi(r)) - V(v).
  \label{eq:rho-def}
\end{equation}
By construction $\rho_\phi(r)\to 0$ as $r\to\infty$ if $\phi(r)\to v$ and
$\phi'(r)\to 0$. In regions where $\phi$ deviates from $v$ and exhibits gradients,
$\rho_\phi(r)$ is non-negative for configurations close to the minimum, since
$V(\phi)-V(v)\ge 0$ and the gradient term is positive.

It is useful to decompose the full stress--energy tensor as
\begin{equation}
  T^{(\phi)}_{\mu\nu}
  = T^{(\rm vac)}_{\mu\nu} + \tau^{(\phi)}_{\mu\nu},
\end{equation}
with
\begin{equation}
  T^{(\rm vac)}_{\mu\nu} \equiv -V(v)\,g_{\mu\nu}, \qquad
  \tau^{(\phi)}_{\mu\nu}
  \equiv \partial_\mu\phi\,\partial_\nu\phi
  - g_{\mu\nu}\left(
    \frac{1}{2} g^{\alpha\beta}\partial_\alpha\phi\,\partial_\beta\phi
    + V(\phi) - V(v)
  \right).
\end{equation}
The Einstein equations then take the form
\begin{equation}
  G_{\mu\nu} + \Lambda_{\rm eff} g_{\mu\nu}
  = 8\pi G_0\left(
    T^{(\rm ordinary)}_{\mu\nu}
    + \tau^{(\phi)}_{\mu\nu}
  \right),
  \label{eq:Einstein-split}
\end{equation}
with $\Lambda_{\rm eff}$ given by \eqref{eq:Lambdaeff}. In this representation,
$\Lambda_{\rm eff}$ is purely homogeneous, while $\tau^{(\phi)}_{\mu\nu}$ encodes the
inhomogeneous, dark-matter-like contribution.

\subsection{Enclosed mass profile $M_\phi(r)$}

In the weak-field limit, the effective mass associated with the defect energy density
inside radius $r$ is given by
\begin{equation}
  M_\phi(r) \equiv 4\pi\int_0^r dr'\,r'^2\,\rho_\phi(r'),
  \label{eq:Mphi-def}
\end{equation}
with $\rho_\phi(r)$ defined in \eqref{eq:rho-def}. This $M_\phi(r)$ is the quantity
that enters the quasi-Newtonian rotation curve for test particles.

If ordinary baryonic matter with density $\rho_{\rm bar}(r)$ is also present, the
total enclosed mass is
\begin{equation}
  M_{\rm tot}(r) = M_{\rm bar}(r) + M_\phi(r),
  \qquad
  M_{\rm bar}(r) = 4\pi\int_0^r dr'\,r'^2\,\rho_{\rm bar}(r'),
\end{equation}
and the circular velocity satisfies
\begin{equation}
  v_c^2(r) \simeq \frac{G_0 M_{\rm tot}(r)}{r}.
  \label{eq:vc-general}
\end{equation}
In the present paper we focus primarily on the defect contribution $M_\phi(r)$,
treating the baryonic component schematically, and leave a full disc+halo treatment
to future work.

\subsection{Linearized outer profile and emergence of $\rho_\phi\propto r^{-2}$}

Away from the core region where $\phi$ departs strongly from $v$ and may pass through
$\phi\simeq 0$, the field is expected to be close to the broken minimum. For radii
$r\gtrsim r_{\rm core}$ we therefore write
\begin{equation}
  \phi(r) = v + \delta\phi(r), \qquad |\delta\phi(r)|\ll v,
  \label{eq:delta-phi-def}
\end{equation}
where $r_{\rm core}$ is the characteristic radius where the field finishes transitioning
into the chosen branch.

Expanding the potential to quadratic order,
\begin{equation}
  V(\phi) = V(v) + \frac{1}{2}V''(v)\,\delta\phi^2 + \mathcal{O}(\delta\phi^3)
  = V(v) + \frac{1}{2}m_\phi^2 \delta\phi^2 + \cdots,
\end{equation}
with
\begin{equation}
  m_\phi^2 \equiv V''(v) > 0.
\end{equation}
Linearizing the equation of motion \eqref{eq:flat-eom} in $\delta\phi$ yields
\begin{equation}
  \delta\phi''(r) + \frac{2}{r}\delta\phi'(r) - m_\phi^2\,\delta\phi(r) = 0.
  \label{eq:delta-phi-eom}
\end{equation}
The static, spherically symmetric solution regular at infinity is
\begin{equation}
  \delta\phi(r) = -\frac{A}{r}\,e^{-m_\phi r},
  \label{eq:delta-phi-sol}
\end{equation}
where $A$ is an amplitude determined by the inner non-linear core and the global
defect configuration. Differentiating,
\begin{equation}
  \delta\phi'(r)
  = -A\,\frac{e^{-m_\phi r}}{r}\left(m_\phi + \frac{1}{r}\right).
\end{equation}

In the linear regime, the defect energy density \eqref{eq:rho-def} becomes
\begin{equation}
  \rho_\phi(r) \simeq \frac{1}{2}\left(\delta\phi'(r)\right)^2
  + \frac{1}{2}m_\phi^2 \left(\delta\phi(r)\right)^2.
\end{equation}
Using \eqref{eq:delta-phi-sol}, one finds
\begin{align}
  \left(\delta\phi(r)\right)^2
  &= \frac{A^2}{r^2}e^{-2m_\phi r},\\
  \left(\delta\phi'(r)\right)^2
  &= A^2\,\frac{e^{-2m_\phi r}}{r^2}
  \left(m_\phi^2 + \frac{2m_\phi}{r} + \frac{1}{r^2}\right),
\end{align}
and hence
\begin{equation}
  \rho_\phi(r) \simeq \frac{A^2}{2} e^{-2m_\phi r}
  \left(
    \frac{2m_\phi^2}{r^2}
    + \frac{2m_\phi}{r^3}
    + \frac{1}{r^4}
  \right).
  \label{eq:rho-linear-exact}
\end{equation}

This expression exhibits three qualitatively distinct radial regimes:
\begin{itemize}
  \item \textbf{Core} ($r\lesssim r_{\rm core}$): the linearization fails; $\phi$ is far
        from $v$ and the full non-linear equation must be solved. For phenomenology
        this region can be approximated as having roughly constant density
        $\rho_\phi \approx \rho_{\rm core}$.
  \item \textbf{Intermediate halo} ($r_{\rm core}\ll r\ll m_\phi^{-1}$): the exponential
        factor $e^{-2m_\phi r}\approx 1$, and the leading term in
        \eqref{eq:rho-linear-exact} is
        \begin{equation}
          \rho_\phi(r) \approx
          \frac{A^2 m_\phi^2}{r^2},
          \qquad r_{\rm core}\ll r\ll m_\phi^{-1}.
          \label{eq:rho-intermediate}
        \end{equation}
        The $r^{-3}$ and $r^{-4}$ contributions are subleading.
  \item \textbf{Outer region} ($r\gg m_\phi^{-1}$): the exponential suppression becomes
        dominant and
        \begin{equation}
          \rho_\phi(r) \sim
          \frac{A^2 m_\phi^2}{r^2} e^{-2m_\phi r},
          \qquad r\gg m_\phi^{-1},
        \end{equation}
        so that the total mass associated with the defect converges.
\end{itemize}

\subsection{Mass profile and quasi-flat rotation curves}

In the intermediate regime \eqref{eq:rho-intermediate}, the enclosed mass \eqref{eq:Mphi-def}
is approximately
\begin{equation}
  M_\phi(r)
  \simeq 4\pi\int_0^r dr'\,r'^2\,\rho_\phi(r')
  \approx 4\pi A^2 m_\phi^2 \int_0^r dr' 
  = 4\pi A^2 m_\phi^2\,r
  \quad (r_{\rm core}\ll r\ll m_\phi^{-1}),
  \label{eq:Mphi-linear}
\end{equation}
up to an additive constant determined by the core region. The corresponding circular
velocity is
\begin{equation}
  v_c^2(r) \simeq \frac{G_0 M_\phi(r)}{r}
  \approx 4\pi G_0 A^2 m_\phi^2
  \equiv v_{\rm flat}^2,
\end{equation}
so that
\begin{equation}
  v_c(r) \approx v_{\rm flat}
  = \sqrt{4\pi G_0}\,A m_\phi,
  \qquad r_{\rm core}\ll r\ll m_\phi^{-1}.
  \label{eq:vflat-linear}
\end{equation}
Thus a generic linearized defect profile around the broken minimum naturally
produces a density $\rho_\phi\propto r^{-2}$ over a finite radial range and hence an
approximately flat rotation curve.

For $r\gg m_\phi^{-1}$, the exponential suppression in \eqref{eq:rho-linear-exact}
causes $M_\phi(r)$ to saturate to a finite value $M_\phi(\infty)$ and the rotation
curve reverts to a Keplerian decline $v_c(r)\propto r^{-1/2}$. The qualitative
behaviour is:
\begin{itemize}
  \item inner region ($r\lesssim r_{\rm core}$): $v_c(r)$ rises with $r$ (roughly
        solid-body-like), corresponding to a roughly constant core density;
  \item intermediate halo ($r_{\rm core}\ll r\ll m_\phi^{-1}$): $M_\phi(r)\propto r$
        and $v_c(r)\approx \text{const}$;
  \item outer region ($r\gg m_\phi^{-1}$): $M_\phi(r)$ saturates, $v_c(r)$ declines as
        $r^{-1/2}$.
\end{itemize}

\subsection{Separation from the homogeneous $\Lambda_{\rm eff}$ background}

In the decomposition \eqref{eq:Einstein-split}, the homogeneous vacuum term
$\Lambda_{\rm eff}g_{\mu\nu}$ affects the large-scale cosmological background but does
not contribute directly to the local Newtonian potential on galactic scales, where
the Poisson equation reduces to
\begin{equation}
  \nabla^2\Phi_N \simeq 4\pi G_0\left(
    \rho_{\rm ordinary} + \rho_\phi
  \right).
\end{equation}
In this sense, the defect contribution $\rho_\phi$ plays the role of an effective
dark-matter density, while $\Lambda_{\rm eff}$ sets the background cosmology. Both
arise from the same underlying order parameter and potential, but their
phenomenological roles are clearly separated.
\\
\\
\\
%======================================================================
\section{Illustrative rotation curves and phenomenological implications}
%======================================================================

\subsection{A simple analytic ansatz for $\phi(r)$}

The linearized solution \eqref{eq:delta-phi-sol} captures the outer behaviour but is
singular at $r=0$ and does not model the core. For phenomenological applications it
is useful to introduce a simple analytic ansatz that:
\begin{itemize}
  \item is regular at $r=0$,
  \item approaches $\phi\to v$ as $r\to\infty$,
  \item reproduces $\rho_\phi(r)\propto 1/r^2$ in an intermediate radial range,
  \item and contains a small number of parameters with clear physical interpretation.
\end{itemize}

One convenient choice is
\begin{equation}
  \phi(r) = v - \Delta\phi\,
  \frac{r_c}{r_c + r}\,e^{-r/R_h},
  \label{eq:toy-phi-ansatz}
\end{equation}
where:
\begin{itemize}
  \item $r_c$ is a core radius controlling the size of the inner region,
  \item $R_h$ is a halo scale roughly corresponding to $m_\phi^{-1}$,
  \item $\Delta\phi$ is the amplitude of the deviation from $v$ near the core.
\end{itemize}
For a kink-like configuration interpolating between $\phi\simeq -v$ at $r=0$ and
$\phi\simeq +v$ at large $r$ one would typically take $\Delta\phi\sim 2v$, but in the
present phenomenological treatment we can keep $\Delta\phi$ as a free parameter.

This ansatz is regular at $r=0$,
\begin{equation}
  \phi(0) = v - \Delta\phi, \qquad \phi'(0)=0,
\end{equation}
and tends to the broken minimum as $r\to\infty$,
\begin{equation}
  \phi(r)\to v, \qquad \phi'(r)\to 0.
\end{equation}

\subsection{Energy density and mass profile for the toy ansatz}

Differentiating \eqref{eq:toy-phi-ansatz},
\begin{align}
  \phi'(r)
  &= -\Delta\phi\,\frac{d}{dr}\left[
    \frac{r_c}{r_c + r}\,e^{-r/R_h}
  \right]\nonumber\\
  &= -\Delta\phi\,e^{-r/R_h}\left[
    \frac{r_c}{(r_c+r)^2}
    - \frac{r_c}{R_h(r_c + r)}
  \right].
  \label{eq:phi-prime-ansatz}
\end{align}
The defect energy density \eqref{eq:rho-def} for this ansatz is then
\begin{equation}
  \rho_\phi(r)
  = \frac{1}{2}\phi'(r)^2
    + V\big(\phi(r)\big) - V(v),
  \label{eq:rho-ansatz-exact}
\end{equation}
with $\phi(r)$ and $\phi'(r)$ given by
\eqref{eq:toy-phi-ansatz}--\eqref{eq:phi-prime-ansatz}. This expression can be
evaluated analytically but is somewhat cumbersome; for practical purposes it is
natural to compute it numerically given parameter values $(\alpha,\beta,G_0,r_c,R_h,
\Delta\phi)$.

For analytic insight, we focus on the intermediate regime
\begin{equation}
  r_c \ll r \ll R_h.
\end{equation}
In this range one has $e^{-r/R_h}\approx 1$ and $r_c+r\approx r$. The ansatz reduces
to
\begin{equation}
  \phi(r) \approx v - \Delta\phi\,\frac{r_c}{r}, \qquad
  \phi'(r) \approx \Delta\phi\,\frac{r_c}{r^2}.
\end{equation}
The gradient contribution to the energy density is then
\begin{equation}
  \frac{1}{2}\phi'(r)^2
  \approx \frac{(\Delta\phi)^2 r_c^2}{2\,r^4},
\end{equation}
while the potential contribution can be approximated by expanding around $v$,
\begin{align}
  V\big(\phi(r)\big) - V(v)
  &\approx \frac{1}{2}m_\phi^2\,\delta\phi(r)^2
  = \frac{1}{2}m_\phi^2\left(\Delta\phi\,\frac{r_c}{r}\right)^2
  = \frac{(\Delta\phi)^2 m_\phi^2 r_c^2}{2\,r^2}.
\end{align}
In this intermediate regime we therefore have
\begin{equation}
  \rho_\phi(r)
  \approx \frac{(\Delta\phi)^2 m_\phi^2 r_c^2}{2\,r^2}
  + \mathcal{O}(r^{-4}),
  \qquad r_c\ll r\ll R_h,
  \label{eq:rho-ansatz-approx}
\end{equation}
so the potential term dominates and the defect energy density scales as $r^{-2}$ as
desired.

The enclosed mass in this regime is
\begin{align}
  M_\phi(r)
  &\approx 4\pi\int_0^r dr'\,r'^2
  \left(\frac{(\Delta\phi)^2 m_\phi^2 r_c^2}{2\,r'^2}\right)
  = 4\pi\left[\frac{(\Delta\phi)^2 m_\phi^2 r_c^2}{2}\right]
  \int_0^r dr'\nonumber\\
  &\approx 2\pi (\Delta\phi)^2 m_\phi^2 r_c^2\,r
  \quad (r_c\ll r\ll R_h),
  \label{eq:Mphi-ansatz-approx}
\end{align}
up to a constant set by the core. The corresponding rotation curve is
\begin{equation}
  v_c^2(r) \simeq \frac{G_0 M_\phi(r)}{r}
  \approx 2\pi G_0 (\Delta\phi)^2 m_\phi^2 r_c^2,
  \qquad r_c\ll r\ll R_h,
\end{equation}
so that
\begin{equation}
  v_c(r)\approx v_{\rm flat}^{\rm (toy)}
  = \sqrt{2\pi G_0}\,|\Delta\phi|\,m_\phi r_c.
  \label{eq:vflat-ansatz}
\end{equation}
This gives a clear interpretation of the parameters:
\begin{itemize}
  \item $r_c$ controls the core size and enters linearly in $v_{\rm flat}$,
  \item $m_\phi^{-1}$ (or equivalently $R_h$) sets the outer scale of the halo and the
        radial extent of the flat part,
  \item $\Delta\phi$ controls the amplitude of the deviation from $v$ and thus the
        overall mass in the halo; roughly $\Delta\phi\sim 2v$ for a full
        $-v\to+v$ transition.
\end{itemize}

For $r\gg R_h$, the exponential factor in \eqref{eq:toy-phi-ansatz} ensures that
$\phi(r)$ relaxes rapidly to $v$, $\rho_\phi(r)$ decays faster than $r^{-2}$, the
enclosed mass $M_\phi(r)$ saturates to a finite constant
$M_\phi(\infty)$, and the rotation curve transitions to a Keplerian fall-off
$v_c(r)\propto r^{-1/2}$.

\subsection{Illustrative rotation curves}

Given the toy ansatz \eqref{eq:toy-phi-ansatz}, the full rotation curve in the
quasi-Newtonian limit is obtained from
\begin{equation}
  v_c^2(r) = \frac{G_0}{r}
  \left[
    M_{\rm bar}(r) + 4\pi\int_0^r dr'\,r'^2\,\rho_\phi(r')
  \right],
  \label{eq:vc-ansatz-full}
\end{equation}
with $\rho_\phi(r)$ from \eqref{eq:rho-ansatz-exact}. For the baryonic component, one
may adopt any standard parametric form (e.g.\ an exponential disc plus bulge) as input.

In practice, for a first phenomenological exploration, it is useful to separate the
rotation curve into baryonic and defect contributions,
\begin{equation}
  v_c^2(r) = v_{\rm bar}^2(r) + v_\phi^2(r),
\end{equation}
with
\begin{equation}
  v_\phi^2(r) \equiv \frac{G_0 M_\phi(r)}{r},
\end{equation}
and to compute $v_\phi(r)$ numerically from \eqref{eq:Mphi-def} using the analytic
expressions for $\phi(r)$ and $\rho_\phi(r)$. The approximate flat value in the
intermediate regime is given analytically by \eqref{eq:vflat-ansatz}.
\\
\\
\\
%======================================================================
\section{Phenomenological comparison with SPARC rotation curves}
%======================================================================

\subsection{SPARC phenomenology and baryonic mass models}

The SPARC database \cite{Lelli:2016} provides high-quality rotation curves and
mass models for 175 disc galaxies, with homogeneous 3.6\,\(\mu\)m photometry and
HI kinematics. For each galaxy the tabulated quantities include the observed
rotation curve $V_{\rm obs}(R)$ with uncertainties, and the contributions from gas,
stellar disc and bulge, $V_{\rm gas}(R)$, $V_{\rm disk}(R)$ and $V_{\rm bul}(R)$,
computed for a fiducial mass-to-light ratio.

In this section we use the SPARC mass decomposition as given, without attempting to
re-fit the baryonic mass-to-light ratios. The baryonic contribution to the circular
velocity is simply
\begin{equation}
  V_{\rm bar}^2(R) = V_{\rm gas}^2(R) + V_{\rm disk}^2(R) + V_{\rm bul}^2(R).
\end{equation}
The residual $V_{\rm obs}^2(R) - V_{\rm bar}^2(R)$ in the usual dark-matter
phenomenology is here interpreted as the contribution from a PhaseGeometry
defect halo.

For a pilot study we select three representative galaxies:
\begin{itemize}
  \item DDO154: a gas-rich dwarf irregular galaxy with slowly rising, still
        rising rotation curve out to $\sim6$\,kpc.
  \item NGC2403: a late-type spiral with a classic flat rotation curve extending
        beyond the stellar disc.
  \item NGC3198: a more massive, high-surface-brightness spiral with an extended
        flat outer region.
\end{itemize}
These three objects span a range of masses and baryonic morphologies while having
well-measured SPARC rotation curves.

\subsection{Mapping the defect halo to physical units}

The defect construction of Secs.~4--6 is naturally formulated in dimensionless
variables. Let $x$ denote the dimensionless radius and $\hat M_\phi(x)$ the
dimensionless enclosed mass obtained from the ansatz \eqref{eq:toy-phi-ansatz} and
the integral \eqref{eq:Mphi-def}, i.e.\ $\hat M_\phi(x)$ is computed with
$G_0=1$ and an arbitrary reference scale.

To compare with a given galaxy we introduce two phenomenological scaling parameters:
a characteristic radius $R_0$ and a characteristic circular velocity $V_0$. Physical
radii are related to the dimensionless variable by
\begin{equation}
  x = \frac{R}{R_0},
\end{equation}
and the defect contribution to the circular velocity is written as
\begin{equation}
  V_\phi^2(R) = V_0^2\,\hat v_\phi^2\!\left(\frac{R}{R_0}\right),
\end{equation}
where
\begin{equation}
  \hat v_\phi^2(x) \equiv \frac{\hat M_\phi(x)}{x}
\end{equation}
is entirely determined by the dimensionless defect ansatz and is the same for all
galaxies. The physical information about a particular galaxy is then encoded only
in the two scaling parameters $(R_0,V_0)$.

The total model rotation curve is
\begin{equation}
  V_{\rm model}^2(R) = V_{\rm bar}^2(R) + V_\phi^2(R)
  = V_{\rm bar}^2(R) + V_0^2\,\hat v_\phi^2(R/R_0).
  \label{eq:Vmodel-def}
\end{equation}
In practice we fix the shape of the defect halo profile from the PhaseGeometry
ansatz and, for each galaxy, fit the two scaling parameters $(R_0,V_0)$ by
minimizing a standard $\chi^2$ against the observed SPARC rotation curve, in
direct analogy with how NFW or pseudo--isothermal halo parameters are typically
fitted to SPARC mass models.

\subsection{Results for three representative galaxies}

Figures~\ref{fig:DDO154}--\ref{fig:NGC3198} show the resulting rotation curves for
the three galaxies in our SPARC subsample. In each panel the blue points with error
bars are the observed SPARC rotation curves, the orange line is the baryonic
contribution from the SPARC mass model, the green line is the PhaseGeometry defect
halo contribution $v_\phi(R)$ computed from the scalar ansatz, and the red line is
the quadrature sum $v_{\rm model}(R)$ from Eq.~\eqref{eq:Vmodel-def}. In all three
cases the underlying dimensionless defect profile $\hat v_\phi(x)$ is the same;
only the scaling parameters $(R_0, V_0)$ differ between galaxies. The best--fit
values of these scaling parameters and the corresponding reduced chi-squared
values for the three benchmark galaxies are summarised in
Table~\ref{tab:fit_params_3gal}.

\begin{figure}[t]
  \centering
  \includegraphics[width=0.7\textwidth]{D00154_1.png}
  \caption{DDO154: observed SPARC rotation curve (blue points with error bars)
  compared to the baryonic contribution (orange), the PhaseGeometry defect halo
  computed from the scalar profile (green), and the total model
  $V_{\rm model}(R)$ (red). The dwarf irregular DDO154 is strongly
  dark-matter dominated; the defect halo reproduces both the overall amplitude and
  the gentle rise of the rotation curve with only two effective scaling parameters
  $(R_0,V_0)$, while all baryonic quantities are fixed by SPARC.}
  \label{fig:DDO154}
\end{figure}

For DDO154 (Fig.~\ref{fig:DDO154}) the baryonic contribution from gas and a very
low-mass stellar disc never exceeds $\sim20$\,km\,s$^{-1}$, whereas the observed
curve rises to $\sim45$\,km\,s$^{-1}$ and remains roughly flat out to the last
measured point. The PhaseGeometry defect halo alone supplies the required
additional support in the outer regions, while the combination of baryons and
defects yields a smooth total rotation curve without invoking any galaxy-specific
dark-matter parameters.

\begin{figure}[t]
  \centering
  \includegraphics[width=0.7\textwidth]{NGC2403.png}
  \caption{NGC2403: rotation curve decomposition as in
  Fig.~\ref{fig:DDO154} (DDO154). Here the luminous baryons contribute substantially
  to the inner rotation curve, and the defect halo is subdominant at small radii but
  dominates the outer parts and maintains an almost flat profile beyond
  $R\simeq5$\,kpc.}
  \label{fig:NGC2403}
\end{figure}

In NGC2403 (Fig.~\ref{fig:NGC2403}) the baryons dominate the gravitational potential
inside the inner few kiloparsecs, where most of the rapid rise in the rotation
curve occurs. The defect halo contribution becomes significant beyond the stellar
disc scale length, where the PhaseGeometry density behaves approximately as
$1/r^2$ and the corresponding $V_\phi(R)$ is nearly flat. With a simple choice of
$(R_0,V_0)$ the combined curve tracks the observed rotation curve from the inner
rising part to the extended flat regime.

\begin{figure}[t]
  \centering
  \includegraphics[width=0.7\textwidth]{NGC3198.png}
  \caption{NGC3198: rotation curve decomposition as in
  Fig.~\ref{fig:DDO154} (DDO154). This more massive, high-surface-brightness galaxy
  shows a classic flat rotation curve out to $\sim45$\,kpc. The defect halo
  produces an approximately flat contribution in the outer parts, while the baryons
  dominate the inner region.}
  \label{fig:NGC3198}
\end{figure}
NGC3198 (Fig.~\ref{fig:NGC3198}) is a classic test case for halo modelling. In our
PhaseGeometry interpretation the inner rise of the rotation curve is again well
accounted for by the disc and bulge, while the defect halo provides the additional
support needed to maintain the flat outer profile. Despite the simplicity of the
mapping --- a single defect ansatz with galaxy-dependent $(R_0,V_0)$ per galaxy ---
the model qualitatively reproduces the main features of these standard rotation
curve benchmarks and suggests that defect-supported haloes can play the role
usually attributed to CDM haloes.

\begin{table}[t]
  \centering
  \caption{Best--fit PhaseGeometry defect--halo parameters for the three
  benchmark SPARC galaxies. The fit uses the full rotation curve with the
  baryonic contribution fixed by the SPARC mass model.}
  \label{tab:fit_params_3gal}
  \begin{tabular}{lccc}
    \hline
    Galaxy   & $R_0$ [kpc] & $V_0$ [km\,s$^{-1}$] & $\chi^2_{\rm red}$ \\
    \hline
    DDO154   & 1.15 & 11.5 & 3.7  \\
    NGC2403  & 3.27 & 28.4 & 23.4 \\
    NGC3198  & 7.93 & 31.9 & 8.3  \\
    \hline
  \end{tabular}
\end{table}

It is worth stressing that the reduced chi-squared values in
Table~\ref{tab:fit_params_3gal} are significantly larger than unity.
This is not unexpected in the present setup: we keep the baryonic
mass models fixed to the fiducial SPARC mass-to-light ratios and
allow only two free parameters $(R_0,V_0)$ per galaxy for the defect
halo. Our aim here is therefore not a fully optimized multi-parameter
fit, but rather to demonstrate that a single defect profile with
galaxy-dependent $(R_0,V_0)$ can reproduce the correct order of
magnitude and overall shape of the rotation curves. In this sense
$\chi^2_{\rm red}>1$ should be interpreted as a limitation of the minimal
two-parameter phenomenology, rather than as a direct inconsistency of
the PhaseGeometry halo with the data.

\subsection{Summary plot: $V_{\rm flat}$ from dwarfs to massive spirals}
\label{subsec:Vflat-summary}


Beyond the three representative galaxies shown in
Figs.~\ref{fig:DDO154}--\ref{fig:NGC3198} it is useful to check whether
the same defect halo profile can qualitatively accommodate systems over a
broader dynamical range. As a first step we consider a small subsample of
twelve SPARC galaxies, spanning from low-mass dwarfs to massive
high-surface-brightness spirals, and for each of them perform the same
two-parameter fit of $(R_0,V_0)$ described above.

For each galaxy we define an observed ``flat'' velocity $V_{\rm flat}^{\rm obs}$
as the maximum value of the SPARC rotation curve $V_{\rm obs}(R)$, and a
model flat velocity $V_{\rm flat}^{\rm model}$ as the maximum of the defect
contribution $V_\phi(R)$ for the best-fit parameters.
Figure~\ref{fig:Vflat-stat} compares these two quantities for the twelve
galaxies in our pilot sample.

\begin{figure}[t]
  \centering
  \includegraphics[width=0.7\textwidth]{Vflat_stat.png}
  \caption{PhaseGeometry defect halo: comparison between the model
  flat velocity $V_{\rm flat}^{\rm model}$ and the observed flat velocity
  $V_{\rm flat}^{\rm obs}$ for a set of twelve SPARC galaxies, ranging from
  the dwarf DDO154 to massive spirals such as NGC2403, NGC3198, NGC5055,
  UGC12506 and NGC7814. The dashed line shows the one-to-one relation
  $V_{\rm flat}^{\rm model}=V_{\rm flat}^{\rm obs}$. Most galaxies lie
  near the diagonal within a factor of order unity, indicating that the
  same defect halo profile with galaxy-dependent $(R_0,V_0)$ can reproduce
  the correct order-of-magnitude flat velocity over more than a decade in
  $V_{\rm flat}$. The outlier NGC5055 sits near the horizontal axis,
  reflecting the fact that its SPARC rotation curve is already almost
  entirely accounted for by the baryonic disc, so that the fitted defect
  halo amplitude is correspondingly small.}
  \label{fig:Vflat-stat}
\end{figure}

Even though this is still a small and deliberately heterogeneous sample,
the summary plot suggests that a single PhaseGeometry defect profile,
scaled only by $(R_0,V_0)$, is compatible with the observed flat
velocities of galaxies spanning from gas-rich dwarfs to high-mass
high-surface-brightness spirals. A more systematic statistical analysis,
including formal uncertainties on $(R_0,V_0)$ and a larger SPARC
subsample, is left for future work.

\FloatBarrier

%======================================================================
\section{Discussion and outlook}
%======================================================================

\subsection{Limitations and open issues}

The analysis presented here is intentionally minimal and carries several important
limitations:

\begin{itemize}
  \item \textbf{Spherical symmetry and stationarity.}
        We restricted attention to a single, static, spherically symmetric defect
        configuration. Real galaxies have discs, bars, triaxial and time-dependent
        haloes; realistic defect networks may involve domain walls, filaments and
        more complex geometries.

  \item \textbf{Approximate Newtonian regime.}
        The scalar field profile was obtained in a fixed, weakly curved background
        (essentially flat space), and only later used to source the Newtonian
        potential. A fully self-consistent treatment requires solving the coupled
        Einstein--scalar system for $(\Phi(r),\Lambda(r),\phi(r))$ simultaneously.

  \item \textbf{Linearization and ansatz.}
        The $r^{-2}$ behaviour in the intermediate regime arises from a linearized
        analysis around the broken minimum and from a particular analytic ansatz.
        The inner core, where $\phi$ passes through $\phi\simeq 0$ and the
        non-linearities of the potential are crucial, has been treated only
        schematically.

  \item \textbf{Neglected baryonic feedback.}
        The baryonic component has been included only as an external source of the
        Newtonian potential. The influence of baryons on the defect configuration
        (e.g.\ through their gravitational field) has been neglected, as has any
        backreaction of $\phi$ on baryonic dynamics beyond gravity.

  \item \textbf{Formation and stability of defects.}
        We have assumed the existence of static, long-lived defect configurations
        with the desired properties, without analyzing their formation in a
        cosmological phase transition, their stability under perturbations, or
        their evolution within an expanding Universe. In particular, one must
        show that a population of halo-like configurations can form without
        overproducing large-scale domain walls that would conflict with
        observations.

  \item \textbf{Cosmological perturbations and large-scale structure.}
        The present work does not yet embed the PhaseGeometry scalar (and its
        defects) into the standard theory of cosmological perturbations.
        A crucial open question is whether a unified scalar--defect dark
        sector can reproduce the observed large-scale structure (LSS) and
        CMB anisotropy spectra at the same level of accuracy as $\Lambda$CDM,
        or whether it predicts distinctive signatures in the matter power
        spectrum and growth rate.

  \item \textbf{Halo hierarchy, clustering and internal profiles.}
        Standard cold dark matter successfully accounts for the hierarchical
        build-up of structure, the halo mass function and, to first
        approximation, typical concentration--mass relations. Here we match
        individual defect-supported configurations to observed rotation curves,
        but do not yet address the statistics and clustering of haloes in a
        realistic population. A full comparison requires simulations of
        scalar-field defect networks in a cosmological background, tracking
        halo formation, merging and profile evolution from dwarfs to clusters.

  \item \textbf{Strong-gravity regime and lensing.}
        The behaviour of the PhaseGeometry scalar and its defects in strongly
        curved spacetimes, for example near black holes, remains to be analysed.
        This includes the structure and stability of black holes with binary
        phase hair, and the impact of defect-supported haloes on strong and
        weak gravitational lensing. Since lensing is one of the cleanest probes
        of the dark sector, it provides a direct and powerful test of the
        defect--halo picture.

  \item \textbf{Parameter-space exploration and tuning.}
        Although the underlying model is minimal (a single real scalar with a
        $\mathbb{Z}_2$-symmetric quartic potential), obtaining simultaneously
        (i) a vacuum energy compatible with the observed late-time acceleration
        and (ii) a scalar mass scale $m_\phi^{-1}$ of order galactic radii may
        require non-trivial correlations between the parameters $(\alpha,\beta)$.
        A systematic exploration of the allowed parameter space, combining the
        background constraints from PhaseGeometry Phenomenology~I with halo and
        structure-formation constraints from the present analysis and future
        simulations, is essential to quantify how much tuning, if any, the
        unified dark-sector picture actually requires.
\end{itemize}

\subsection{Next steps: towards quantitative PhaseGeometry halo phenomenology}

Within the PhaseGeometry programme, several concrete extensions suggest themselves:

\begin{itemize}
  \item \textbf{Non-linear numerical solutions.}
        Solve the full non-linear equation \eqref{eq:flat-eom}, and ultimately the
        coupled Einstein--scalar system with metric \eqref{eq:metric}, for
        representative defect configurations. This would provide first-principles
        profiles for $\phi(r)$, $\rho_\phi(r)$ and $v_c(r)$ beyond the linear and
        ansatz approximations used here.

  \item \textbf{Defect networks and coarse-graining.}
        Extend from single defects to realistic networks of domains and walls,
        including their statistical distribution in a given galaxy or halo.
        Develop a coarse-grained description of the effective $\rho_\phi$ on
        galactic and cluster scales, and study how such networks cluster and
        merge in an expanding Universe.

  \item \textbf{Full disc+halo modelling.}
        Incorporate realistic baryonic discs and bulges, including their
        gravitational backreaction on the defect configuration, and confront the
        combined rotation curves with high-quality data sets such as
        SPARC \cite{Lelli:2016}.

  \item \textbf{Lensing and dynamical tests.}
        Go beyond rotation curves to include strong and weak gravitational lensing,
        satellite dynamics and cluster-scale observables, providing multi-probe
        constraints on defect-supported haloes and directly testing whether the
        inferred mass distribution matches that required by observations.

  \item \textbf{Cosmological perturbations and LSS.}
        Embed the PhaseGeometry scalar and its defect sector in the standard
        cosmological perturbation framework, compute the matter power spectrum,
        growth rate and CMB anisotropies, and compare them with current and
        future data. This step is crucial to assess whether the unified
        scalar--defect picture can serve as a genuine alternative to $\Lambda$CDM
        at the level of precision cosmology.

  \item \textbf{Parameter constraints and consistency.}
        Explore the allowed regions of the $(\alpha,\beta,G_0)$ parameter space
        where (i) $\Lambda_{\rm eff}$ from \eqref{eq:Lambdaeff} matches
        cosmological measurements, (ii) the defect sector can support haloes with
        realistic rotation curves, and (iii) cosmological perturbations and LSS
        observables remain consistent with current bounds.

  \item \textbf{Microscopic interpretation.}
        Relate the macroscopic defects and kink-like configurations in $\phi$ to
        more microscopic structures of the underlying pre-geometric field(s), in
        a way that preserves the minimal binary-symmetry picture and clarifies
        the origin and stability of the haloes within a broader PhaseGeometry
        framework.
\end{itemize}

The central message of this paper is that, within the minimal PhaseGeometry
framework and without introducing any new fields or couplings beyond those
already present in $S[g,\phi]$, defect configurations in the inhomogeneous
broken phase naturally give rise to halo-like mass distributions with
$\rho_\phi(r)\propto 1/r^2$ over an intermediate range of radii and hence to
quasi-flat rotation curves. In other words, defects of a single
$\mathbb{Z}_2$-symmetric scalar can already play, at the level of a simple
phenomenological model, the same qualitative role that is usually ascribed to
standard cold-dark-matter haloes --- without postulating a separate cold dark
matter fluid.

Taken together with the homogeneous background analysis of
PhaseGeometry Phenomenology~I, this suggests that a single binary phase medium
can, at least at a coarse-grained level, account for both the effective
cosmological constant and dark-matter-like haloes. The framework does not yet
rival the empirical maturity of the $\Lambda$CDM paradigm, but it provides a
concrete, geometrically motivated alternative whose predictions can be sharpened
and tested. Whether PhaseGeometry will ultimately survive detailed comparison
with data is an open question, but the combination of Phenomenology~I and II
already offers a coherent target for future analytical work, numerical
simulations and observational probes of the dark sector.

\section*{Acknowledgements}

The author is grateful to several colleagues and early readers for helpful
comments and questions on scalar-field halo models and the broader
PhaseGeometry programme. Parts of this work were developed in intensive
dialogue with large language models used as technical writing and calculation
assistants; responsibility for the physical content and conclusions rests
entirely with the author.



%======================================================================
\appendix
%======================================================================

\section{Numerical implementation and illustrative figures}
\label{app:numerics}

In this appendix we summarize the numerical setup used to generate illustrative toy
profiles and rotation curves, and we collect the corresponding figures. All
numerical examples are based on the toy ansatz \eqref{eq:toy-phi-ansatz} with
parameters chosen for clarity rather than as fits to specific galaxies.

\subsection{Dimensionless units and parameter choice}

We work in dimensionless units with $G_0=1$, lengths measured in units of an
arbitrary scale $r_0$, and masses in units of $M_0$. A physical interpretation can
be restored by reintroducing $r_0$ and $M_0$ at the end.

For the scalar sector we take
\begin{equation}
  \alpha=-1,\qquad \beta=1,
\end{equation}
so that the broken-phase vacuum expectation value is
\begin{equation}
  v = \sqrt{-\alpha/\beta} = 1.
\end{equation}
The mass of small fluctuations around the broken minimum is
\begin{equation}
  m_\phi^2 = V''(v) = 2\alpha + 6\beta v^2 = 4,
  \qquad m_\phi=2.
\end{equation}
For the toy ansatz \eqref{eq:toy-phi-ansatz} we choose
\begin{equation}
  \Delta\phi = 2v,\qquad
  r_c = 1,\qquad
  R_h = \frac{20}{m_\phi} = 10.
\end{equation}
Thus $r_c$ sets the core size, while $R_h\sim 1/m_\phi$ sets the outer halo scale.
The rotation curves are computed on a radial grid $r\in[0,r_{\max}]$ with
$r_{\max}=30$ in these units.

\subsection{Defect energy density and mass profile}

The defect energy density is defined as the excess over the homogeneous broken phase,
\begin{equation}
  \rho_\phi(r) =
  \frac{1}{2}\left(\frac{d\phi}{dr}\right)^2
  + V(\phi(r)) - V(v),
  \label{eq:app-rho-def}
\end{equation}
with $\phi(r)$ given by \eqref{eq:toy-phi-ansatz} and
$V(v) = \alpha v^2 + \frac{\beta}{2}v^4$. The derivative of the ansatz is computed
analytically and implemented directly in the numerical code.

Assuming spherical symmetry, the enclosed mass profile is
\begin{equation}
  M_\phi(r) = 4\pi\int_0^r dr'\,r'^2\,\rho_\phi(r'),
  \label{eq:app-Mphi}
\end{equation}
In the numerical implementation this integral is evaluated on the one-dimensional
grid $r_i$ using a standard trapezoidal rule (for example, a cumulative trapezoid
integration, e.g.\ the SciPy function \texttt{cumulative\_trapezoid} in Python).

\subsection{Rotation curves}

In the quasi-Newtonian regime the circular velocity associated with the defect
component is
\begin{equation}
  v_\phi^2(r) = \frac{G_0 M_\phi(r)}{r},
  \label{eq:app-vc-def}
\end{equation}
which in the dimensionless units used here reduces to $v_\phi^2(r)=M_\phi(r)/r$.
For the toy ``galaxy'' example we also include a simple baryonic component with an
effective enclosed mass $M_{\rm bar}(r)$ modelled as a spherically-symmetric
exponential profile that asymptotes to a total mass $M_{\rm disk}$,
\begin{equation}
  M_{\rm bar}(r) \to M_{\rm disk} \quad \text{as } r\to\infty,
\end{equation}
so that
\begin{equation}
  v_{\rm bar}^2(r) = \frac{G_0 M_{\rm bar}(r)}{r}.
\end{equation}
The total rotation curve is then
\begin{equation}
  v_c^2(r) = v_{\rm bar}^2(r) + v_\phi^2(r).
\end{equation}
In the illustrative example we take $M_{\rm disk}=50$ and a disc scale radius
$R_{\rm d}=3$ in the same units.

\subsection{Illustrative toy profiles}

All profiles in this subsection are computed in the dimensionless setup described
above, with $(\alpha,\beta)=(-1,1)$, $v=1$, $\Delta\phi=2v$, $r_c=1$, $R_h=10$ and
$G_0=1$. The scalar field profile $\phi(r)$, the corresponding defect energy
density $\rho_\phi(r)$, the enclosed mass $M_\phi(r)$ and the defect-only
rotation curve $v_\phi(r)$ are shown separately, followed by a toy ``galaxy''
example where the defect halo is combined with a simple baryonic component.

\begin{figure}[H]
  \centering
  \includegraphics[width=0.6\textwidth]{1_Profile_Phi.png}
  \caption{Illustrative PhaseGeometry defect configuration: scalar field profile
  $\phi(r)$ as a function of radius. Parameters are
  $(\alpha,\beta) = (-1,1)$, $\Delta\phi=2v$, $r_c=1$ and $R_h=10$.
  Here and in Figs.~\ref{fig:rho-profile}--\ref{fig:toy-galaxy} all quantities
  are shown in dimensionless units with $G_0=1$ and $v=1$. The field interpolates
  smoothly between $\phi(0)\simeq -v$ and $\phi(r)\to +v$ at large radii.}
  \label{fig:phi-profile}
\end{figure}

\begin{figure}[H]
  \centering
  \includegraphics[width=0.6\textwidth]{2_Defect_energy_density.png}
  \caption{Defect energy density $\rho_\phi(r)$ corresponding to the profile in
  Fig.~\ref{fig:phi-profile}, plotted on a logarithmic vertical axis. In an
  intermediate range of radii the profile exhibits an approximate $r^{-2}$
  scaling, leading to $M_\phi(r)\propto r$ and a quasi-flat contribution to the
  rotation curve. At large radii the density is exponentially suppressed.}
  \label{fig:rho-profile}
\end{figure}

\begin{figure}[H]
  \centering
  \includegraphics[width=0.6\textwidth]{3_mass_profile_M_phi.png}
  \caption{Enclosed mass $M_\phi(r)$ for the illustrative defect halo. The mass
  grows steeply in the inner core, approximately linearly in the intermediate halo
  region where $\rho_\phi(r)\propto r^{-2}$, and saturates at large radii when
  the density becomes exponentially suppressed.}
  \label{fig:M-profile}
\end{figure}

\begin{figure}[H]
  \centering
  \includegraphics[width=0.6\textwidth]{4_Rotation_curvev_phi.png}
  \caption{Defect contribution to the rotation curve, $v_\phi(r)$, obtained
  from $v_\phi^2(r)=G_0 M_\phi(r)/r$ with $G_0=1$. The curve rises roughly
  linearly in the core, is quasi-flat over the range where $M_\phi(r)\propto r$,
  and declines slowly once the enclosed defect mass saturates.}
  \label{fig:vc-profile}
\end{figure}

\begin{figure}[H]
  \centering
  \includegraphics[width=0.7\textwidth]{5_Toy_galaxy.png}
  \caption{Toy rotation curve combining a PhaseGeometry defect halo with a
  simple baryonic component. The defect contribution $v_\phi(r)$, the baryonic
  contribution $v_{\rm bar}(r)$ and the total $v_c(r)$ are shown separately.
  Parameters are chosen for clarity and do not represent a fit to any specific
  galaxy. The example demonstrates how a defect-supported halo can generate a
  broad quasi-flat segment of $v_c(r)$ similar to that observed in spiral
  galaxies.}
  \label{fig:toy-galaxy}
\end{figure}

\subsection{Verification of the $1/r^2$ regime}

To make the $1/r^2$ behaviour more explicit it is useful to examine the
dimensionless defect energy density $\rho_\phi(x)$ and the combination
$\rho_\phi(x)x^2$ in log--log coordinates, where $x$ denotes the dimensionless
radius. The following figures use the same scalar parameters as above but focus on
the intermediate halo region.

\begin{figure}[H]
  \centering
  \includegraphics[width=0.6\textwidth]{D00154_2.png}
  \caption{Defect halo energy density $\rho_\phi(x)$ (solid blue) compared with
  a best-fit profile $A/x^2$ (dashed orange) in log--log coordinates. Over an
  extended radial range the two curves overlap closely, demonstrating that the
  defect energy density follows $\rho_\phi\propto 1/x^2$ to good approximation
  before the exponential outer cutoff sets in.}
  \label{fig:rho-fit}
\end{figure}

\begin{figure}[H]
  \centering
  \includegraphics[width=0.6\textwidth]{D00154_3.png}
  \caption{Defect halo profile $\rho_\phi(x)x^2$ as a function of dimensionless
  radius $x$. The broad plateau indicates the radial range where
  $\rho_\phi(x)\propto 1/x^2$, corresponding to a defect halo with
  $M_\phi(x)\propto x$ and a quasi-flat rotation curve.}
  \label{fig:rho-r2-plateau}
\end{figure}

\FloatBarrier
\pagebreak
%======================================================================
% References
%======================================================================


%=========================================================
\section*{Next steps and Core Package structure}
%=========================================================

This note is part of the PhaseGeometry Z$_2$ Core Package v2.5 (strict). Together
with the other notes listed below it defines a minimal binary-phase framework
for dark energy, dark-matter–like haloes, black holes and quantum branching in
a single Z$_2$ medium. The Core Package is archived as a bundled record under
Zenodo DOI\,10.5281/zenodo.17807433.

The current structure of the Core Package is:

\begin{itemize}
  \item \textbf{Foundations I -- Technical Passport: Minimal Binary Phase Model for Dark Sector and Gravity}\\
        Defines the strict Z$_2$ action $S[g,\phi]$, potential $V(\phi)$, phase
        structure and basic ontology of the binary phase medium.
  \item \textbf{Foundations II -- Lambda from Broken Phase}\\
        Derives how the homogeneous broken phase $\phi\simeq\pm v$ acts as an
        effective cosmological constant $\Lambda_{\rm eff}=8\pi G_0 V(v)$.
  \item \textbf{Foundations III -- Dark Matter from Defects}\\
        Develops the defect/inhomogeneous sector and defines the excess energy
        density $\rho_\phi(r)$ as a dark-matter–like component.
  \item \textbf{Foundations IV -- Phase medium, observer and branching histories}\\
        Interprets the Z$_2$ field as a phase medium hosting observers and
        classical branches in an Everett–Zurek picture.
  \item \textbf{Foundations V -- Decoherence and Quantum Darwinism in a binary phase medium}\\
        Implements decoherence and Quantum Darwinism explicitly in finite Z$_2$
        chains, with redundant records stored in environmental fragments.
  \item \textbf{Phenomenology I -- Background cosmology and the age of the Universe}\\
        Confronts the homogeneous broken phase with FRW background evolution,
        distance–redshift relations and cosmic age constraints.
  \item \textbf{Phenomenology II -- Dark-matter–like haloes from defects}\\
        Studies static halo profiles supported by Z$_2$ defects and compares
        them with rotation-curve phenomenology.
  \item \textbf{Phenomenology III -- Static black holes with binary phase hair}\\
        Embeds black holes into the same binary phase medium and explores weak
        scalar hair and defect-like shells in the strong-gravity regime.
\end{itemize}

Across these layers, the same strict Z$_2$ order parameter $\phi$ underlies:
(i) the effective cosmological constant from the homogeneous broken phase,
(ii) dark-matter–like haloes from defects and inhomogeneities,
(iii) phase-dressed black holes, and
(iv) quantum branching and classical records realised as patterns in the binary medium.

The present note should be read as one component of this unified picture. It is
designed to be technically self-contained, but its full meaning emerges when
combined with the other items in the PhaseGeometry Z$_2$ Core Package v2.5 (strict).



\begin{thebibliography}{99}

\bibitem{passport}
A.~Turchanov,
\emph{PhaseGeometry Z$_2$ Foundations I -- Technical Passport},
PhaseGeometry Z$_2$ Core Package v2.5 (strict),
Zenodo (2025), DOI:10.5281/zenodo.17807433.

\bibitem{Rubin:1980}
V.~C.~Rubin, N.~Thonnard and W.~K.~Ford Jr.,
``Rotational properties of 21 SC galaxies with a large range of luminosities and
radii, from NGC 4605 (R = 4 kpc) to UGC 2885 (R = 122 kpc),''
\emph{Astrophys.\ J.} \textbf{238}, 471 (1980).

\bibitem{SofueRubin:2001}
Y.~Sofue and V.~Rubin,
``Rotation curves of spiral galaxies,''
\emph{Ann.\ Rev.\ Astron.\ Astrophys.} \textbf{39}, 137 (2001).

\bibitem{Navarro:1996}
J.~F.~Navarro, C.~S.~Frenk and S.~D.~M.~White,
``The Structure of cold dark matter haloes,''
\emph{Astrophys.\ J.} \textbf{462}, 563 (1996).

\bibitem{Lelli:2016}
F.~Lelli, S.~S.~McGaugh and J.~M.~Schombert,
``SPARC: Mass Models for 175 Disk Galaxies with Spitzer Photometry and Accurate
Rotation Curves,''
\emph{Astron.\ J.} \textbf{152}, 157 (2016).

\end{thebibliography}

\end{document}

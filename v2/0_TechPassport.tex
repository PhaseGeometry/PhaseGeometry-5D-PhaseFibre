\documentclass[11pt,a4paper]{article}

\usepackage[utf8]{inputenc}
\usepackage[T1]{fontenc}
\usepackage[english]{babel}

\usepackage{amsmath,amssymb}
\usepackage{geometry}
\usepackage{hyperref}
\geometry{margin=1in}
\numberwithin{equation}{section}

\title{PhaseGeometry Phase-Fibre Foundations 0:\\[4pt]
Phase as a compact fifth dimension for gravity, electromagnetism and phase clocks\\[4pt]
Technical Passport}

\author{Aleksey Turchanov}
\date{November 2025}

\begin{document}

\maketitle

\begin{center}
\small
Licensed under Creative Commons Attribution 4.0 International (CC BY 4.0).\\[2pt]
Part of the PhaseGeometry Phase-Fibre programme; archived in the Core Package\\
DOI: \texttt{10.5281/zenodo.17783322}.
\end{center}

\vspace{1em}

\begin{abstract}
We formulate a minimal five-dimensional phase-fibre framework in which weak gravity,
classical electromagnetism and phase-based clocks are described within a single
geometric structure. The central structural move is to \emph{promote phase to a
genuine fifth dimension}: a compact internal coordinate $\phi$ living on a U(1)
fibre above ordinary four-dimensional spacetime. In other words, physical systems
are described on $M_5 = M_4 \times S^1$, where $M_4$ carries the usual spacetime
metric and $S^1$ carries the phase coordinate.

The construction is deliberately conservative. We take standard four-dimensional
general relativity and Maxwell electrodynamics and repackage them into a
five-dimensional Kaluza--Klein metric with a compact phase fibre. No new
propagating degrees of freedom are introduced; any Einstein--Maxwell solution
can be lifted to this phase-fibre picture. Gauge transformations of the
electromagnetic potential are encoded as reparametrisations of the phase
coordinate on the fibre.

On top of the background fields we introduce a simple worldline description of a
``phase clock'' --- a test trajectory in spacetime carrying an internal phase
variable on the fibre. The covariant phase rate is constrained to a constant
reference frequency in proper time. This provides a compact way to discuss
phase-based clocks (including Josephson oscillators) and gravitational redshift
in the same geometric language as weak gravity and electromagnetism.

The goal of this Technical Passport is not to derive every result in detail, but
to fix a stable set of assumptions, equations and conventions for the
\emph{``phase as fifth dimension''} framework that subsequent notes in the
Phase-Fibre series can refer to. The framework does not modify the tested
content of classical GR+Maxwell; it reorganises it so that phase becomes a
central geometric ingredient rather than an auxiliary label.
\end{abstract}

%======================================================================
\section*{0. Purpose of this document}
%======================================================================

This document specifies a minimal five-dimensional phase-fibre framework
designed to:

\begin{itemize}
  \item \textbf{Promote phase to a full geometric coordinate.}
        Make explicit the move from a purely four-dimensional
        spacetime picture to a \emph{$4D + \phi$} picture, where phase
        becomes a compact fifth dimension: a U(1) fibre with coordinate
        $\phi$ above each spacetime point.
  \item Package standard weak-field gravity and classical electromagnetism
        into a single geometric structure, without introducing extra
        light fields beyond Einstein--Maxwell.
  \item Provide a clean description of phase-based clocks
        (e.g.\ Josephson oscillators, interferometers) in curved
        spacetime and electromagnetic backgrounds, with their internal
        phase treated as motion along the fifth dimension.
  \item Make the link between Newtonian gravity, redshift of clocks and
        phase evolution transparent and calculable in one common
        language.
  \item Serve as a ``core'' reference: later Phase-Fibre notes
        (on clock configurations, horizons, haloes, cosmology, etc.)
        will use exactly this framework.
\end{itemize}

The philosophy is deliberately conservative. We remain entirely within
classical four-dimensional general relativity (GR) and Maxwell theory.
The five-dimensional viewpoint is used as a structural rephrasing whose
\emph{key step} is:

\begin{quote}
Treat phase as a genuine fifth dimension, modelled by a compact
U(1) fibre with coordinate $\phi$ over $M_4$, and encode
electromagnetism and phase clocks in this fibre geometry.
\end{quote}

Concretely:

\begin{itemize}
  \item gravity is encoded in the spacetime metric $g_{\mu\nu}(x)$;
  \item electromagnetism is the connection on the compact U(1) phase
        fibre, via the gauge field $A_\mu(x)$;
  \item phase clocks follow timelike worldlines and carry an internal
        phase coordinate $\phi(\tau)$ on this fibre, i.e.\ the
        coordinate along the fifth dimension.
\end{itemize}

All calculations in follow-up work can be performed in ordinary
four-dimensional notation. The phase-fibre picture simply provides a
compact ``source code'' that keeps the assumptions and normalisations
under control and keeps the ``phase as fifth dimension'' idea explicit.

%======================================================================
\section{Ontology: base, fibre and fields}
%======================================================================

\subsection{Spacetime base}

We start from a four-dimensional spacetime manifold $M_4$ with Lorentzian
metric $g_{\mu\nu}(x)$ of signature $(+,-,-,-)$. Greek indices
$\mu,\nu,\dots$ run over $0,1,2,3$, with $x^0 = ct$.

In the weak-field regime we often write
\begin{equation}
  g_{00}(x) \simeq 1 + \frac{2\Phi(x)}{c^2}, \qquad
  g_{0i}(x) \simeq 0, \qquad
  g_{ij}(x) \simeq -\delta_{ij},
  \label{eq:weak_field_metric}
\end{equation}
where $\Phi(x)$ is the Newtonian gravitational potential.

\subsection{Phase fibre: phase as a geometric fifth dimension}

Over each spacetime point $x^\mu$ we attach a compact U(1) phase fibre
with coordinate $\phi$, identified modulo $2\pi$,
\begin{equation}
  \phi \sim \phi + 2\pi.
\end{equation}
The fibre is a circle $S^1$ of fixed radius $R$ in internal space. The
total five-dimensional manifold is the product
\begin{equation}
  M_5 = M_4 \times S^1,
\end{equation}
described locally by coordinates $X^A = (x^\mu,\phi)$. Capital indices
$A,B,\dots$ run over $0,1,2,3,5$.

This embodies the main conceptual step of the Phase-Fibre line:

\begin{quote}
Phase is not just an auxiliary label in the equations; it is a
\emph{geometric coordinate} on a compact fifth dimension, on the same
footing (mathematically) as the usual spacetime coordinates.
\end{quote}

The geometric role of the fibre is twofold:
\begin{itemize}
  \item it provides a natural home for electromagnetic gauge transformations
        as shifts of the phase coordinate;
  \item it hosts an internal phase variable for macroscopic clocks, so
        that their phase evolution can be treated as motion along the
        fifth dimension.
\end{itemize}

\subsection{Fields and sources}

On $M_4$ we consider the usual gravitational and electromagnetic fields:
\begin{itemize}
  \item a spacetime metric $g_{\mu\nu}(x)$;
  \item a U(1) gauge field $A_\mu(x)$ with field strength
        $F_{\mu\nu} = \partial_\mu A_\nu - \partial_\nu A_\mu$;
  \item matter sources with energy--momentum tensor $T_{\mu\nu}$ and
        electromagnetic current $J^\mu$.
\end{itemize}

The phase-fibre framework is agnostic about the microscopic origin of the
matter sector: it can be any effective fluid or field model consistent
with Einstein--Maxwell theory.

\subsection{Notation for phase variables}

In the broader PhaseGeometry programme several types of phase variables
appear. We will use the following conventions:

\begin{itemize}
  \item $\theta(x)$ denotes the condensate phase of a superconducting
        order parameter in device-oriented notes.
  \item $\phi$ denotes the compact phase-fibre coordinate on $S^1$.
        Along a worldline $x^\mu(\tau)$ we write $\phi(\tau)$ for the
        corresponding phase variable carried by a clock, i.e.\ the
        coordinate along the fifth dimension.
  \item $\varphi(x)$ (or, in some notes, $\phi_{\rm N}(x)$) denotes a
        phase-field degree of freedom in the Phase-Field Newtonian
        sector, used to model effective gravitational sources and
        defects.
\end{itemize}

In the present Technical Passport we only work with the phase-fibre
coordinate and its worldline version; the condensate and phase-field
variables appear in other notes but obey the same basic conventions
outlined here.

%======================================================================
\section{5D phase-fibre metric and effective action}
%======================================================================

\subsection{Kaluza--Klein metric with phase fibre}

On $M_5$ we adopt the Kaluza--Klein-type metric ansatz
\begin{equation}
  ds_5^2 = G_{AB} dX^A dX^B
  = g_{\mu\nu}(x) dx^\mu dx^\nu
  + R^2 \bigl(d\phi + k A_\mu(x) dx^\mu\bigr)^2,
  \label{eq:5d_metric}
\end{equation}
where $R$ and $k$ are constants chosen so that $kA_\mu dx^\mu$ is
dimensionless. The one-form
\begin{equation}
  \mathcal{A} \equiv d\phi + k A_\mu(x) dx^\mu
\end{equation}
is interpreted as a connection on the U(1) phase fibre. In this form,
electromagnetism is literally encoded as geometry of the fifth
(phase) dimension.

Under a local phase redefinition
\begin{equation}
  \phi(x) \rightarrow \phi'(x) = \phi(x) + \chi(x),
\end{equation}
we require the five-dimensional metric to remain invariant. This is
achieved if the four-dimensional gauge field transforms as
\begin{equation}
  A_\mu(x) \rightarrow A'_\mu(x)
  = A_\mu(x) - \frac{1}{k}\,\partial_\mu \chi(x),
  \label{eq:gauge_transformation}
\end{equation}
which is exactly the standard U(1) gauge transformation. Thus Maxwell
gauge invariance is geometrically encoded as freedom to shift the phase
coordinate along the fibre: a direct realisation of the
``phase as fifth dimension'' idea.

\subsection{Five-dimensional Einstein--Hilbert action}

We consider the five-dimensional Einstein--Hilbert action
\begin{equation}
  S_5 = \frac{1}{16\pi G_5}
        \int d^5X \,\sqrt{-G}\, R_5,
  \label{eq:S5}
\end{equation}
where $G = \det G_{AB}$, $R_5$ is the five-dimensional Ricci scalar and
$G_5$ is the five-dimensional Newton constant.

For the metric ansatz \eqref{eq:5d_metric} with all fields independent of
$\phi$ one finds the standard Kaluza--Klein identity
\begin{equation}
  R_5 = R_4 - \frac{1}{4}(kR)^2 F_{\mu\nu} F^{\mu\nu},
  \label{eq:R5_decomposition}
\end{equation}
where $R_4$ is the four-dimensional Ricci scalar of $g_{\mu\nu}$.

The determinant factorises as
\begin{equation}
  \sqrt{-G} = R\,\sqrt{-g},
\end{equation}
so that integration over the compact fibre (with $\phi \in[0,2\pi)$)
yields
\begin{equation}
  S_5 = \frac{2\pi R}{16\pi G_5}
        \int d^4x\,\sqrt{-g}\,
        \left(R_4 - \frac{1}{4}(kR)^2 F_{\mu\nu}F^{\mu\nu}\right).
  \label{eq:S5_reduced}
\end{equation}

\subsection{Dimensional reduction to Einstein--Maxwell}

We now identify the effective four-dimensional Newton constant $G_4$ via
\begin{equation}
  \frac{1}{16\pi G_4} \equiv \frac{2\pi R}{16\pi G_5},
\end{equation}
and choose the combination $(kR)^2$ such that the Maxwell term acquires
its standard prefactor $1/(4\mu_0)$:
\begin{equation}
  \frac{2\pi R}{16\pi G_5} (kR)^2 = \frac{1}{\mu_0}.
\end{equation}
With this normalisation, the gravitational and electromagnetic parts of
\eqref{eq:S5_reduced} reduce to the familiar four-dimensional action
\begin{equation}
  S_{\rm grav+EM}
  = \frac{1}{16\pi G_4} \int d^4x\,\sqrt{-g}\,R_4
    - \frac{1}{4\mu_0} \int d^4x\,\sqrt{-g}\,F_{\mu\nu}F^{\mu\nu}.
  \label{eq:S_Einstein_Maxwell}
\end{equation}

Matter fields couple in the usual way through an action
\begin{equation}
  S_{\rm matter} = \int d^4x\,\sqrt{-g}\,\mathcal{L}_{\rm matter}
                   [g_{\mu\nu}, A_\mu, \dots],
\end{equation}
giving an energy--momentum tensor $T_{\mu\nu}$ and current $J^\mu$.

Varying the total four-dimensional action
$S = S_{\rm grav+EM} + S_{\rm matter}$ with respect to $g_{\mu\nu}$ and
$A_\mu$ yields the standard Einstein and Maxwell equations,
\begin{align}
  G_{\mu\nu} &= 8\pi G_4\,T_{\mu\nu}, \label{eq:Einstein_eq}\\[3pt]
  \nabla_\mu F^{\mu\nu} &= \mu_0 J^\nu. \label{eq:Maxwell_eq}
\end{align}

Thus, at the level of background fields the phase-fibre framework is
exactly equivalent to ordinary Einstein--Maxwell theory: every solution of
\eqref{eq:Einstein_eq}--\eqref{eq:Maxwell_eq} lifts to a five-dimensional
solution of \eqref{eq:S5} with metric \eqref{eq:5d_metric}, and no new
propagating modes are introduced. The novelty lies not in new dynamics,
but in treating phase as an explicit fifth dimension that geometrises
both electromagnetism and phase clocks.

%======================================================================
\section{Phase clocks as test worldlines}
%======================================================================

We now introduce a minimal description of phase-based clocks as test
worldlines carrying an internal phase coordinate on the fibre. The goal
is to obtain a clear evolution equation for the phase in a given
background $(g_{\mu\nu},A_\mu)$ and to connect it to gravitational
redshift and other phase phenomena.

\subsection{Worldline and proper time}

A point-like clock follows a timelike worldline $x^\mu(\lambda)$ in
spacetime. For a given metric $g_{\mu\nu}$ the line element is
\begin{equation}
  ds^2 = g_{\mu\nu} dx^\mu dx^\nu.
\end{equation}
For timelike curves with $ds^2>0$ we define proper time $\tau$ along the
worldline by
\begin{equation}
  d\tau = \frac{1}{c}\sqrt{ds^2}
        = \frac{1}{c}\sqrt{g_{\mu\nu} u^\mu u^\nu}\,d\lambda,
\end{equation}
where $u^\mu \equiv dx^\mu / d\lambda$ is the tangent vector.
Choosing $\lambda=\tau$ we obtain the four-velocity
\begin{equation}
  u^\mu \equiv \frac{dx^\mu}{d\tau}, \qquad
  g_{\mu\nu} u^\mu u^\nu = c^2.
\end{equation}

\subsection{Phase variable and covariant phase rate}

On top of the spacetime trajectory we introduce an internal phase
variable $\phi(\tau)$ taking values on the fibre,
\begin{equation}
  \phi(\tau) \sim \phi(\tau) + 2\pi.
\end{equation}
Geometrically, this is the coordinate along the fifth (phase) dimension
for that particular worldline.

Its covariant rate along the worldline is defined by
\begin{equation}
  \frac{D\phi}{D\tau}
  \equiv \frac{d\phi}{d\tau} + k A_\mu(x(\tau))\,u^\mu(\tau).
  \label{eq:covariant_phase_rate}
\end{equation}
This combination is invariant under the gauge transformation
\eqref{eq:gauge_transformation} and corresponds to the phase velocity
measured with respect to proper time, including the minimal coupling to
the electromagnetic potential.

\subsection{Clock action and ideal phase constraint}

We model a phase-based clock by the worldline action
\begin{equation}
  S_{\rm clock}
  = -m c^2 \int d\tau
    + \frac{I}{2} \int d\tau\,
      \left[\frac{D\phi}{D\tau} - \omega_0\right]^2,
  \label{eq:S_clock}
\end{equation}
where:
\begin{itemize}
  \item $m$ is the rest mass of the clock (assumed small enough that it
        can be treated as a test body);
  \item $I$ is an effective ``moment of inertia'' in phase space
        controlling how tightly the phase rate is constrained;
  \item $\omega_0$ is a constant reference angular frequency representing
        the intrinsic phase rate of the clock in proper time.
\end{itemize}

The first term in \eqref{eq:S_clock} is the usual proper-time action for
a point particle. The second term penalises deviations of the covariant
phase rate from $\omega_0$. In the formal limit $I\to\infty$ the
constraint is enforced exactly,
\begin{equation}
  \frac{D\phi}{D\tau} = \omega_0,
  \label{eq:ideal_clock_constraint}
\end{equation}
and the clock runs at a fixed phase rate along its worldline.

Throughout this Technical Passport we work in the test-clock limit:
the backreaction of $S_{\rm clock}$ on the background fields
$g_{\mu\nu}$ and $A_\mu$ is neglected. The worldline follows a timelike
geodesic (or prescribed trajectory), while the phase evolution is
governed by \eqref{eq:covariant_phase_rate} and
\eqref{eq:ideal_clock_constraint}. In the five-dimensional picture, this
is simply a curve in $M_5$ with a constrained component of its velocity
along the fifth (phase) direction.

%======================================================================
\section{Newtonian limit and gravitational redshift}
%======================================================================

To make contact with familiar weak-field physics we consider a static
gravitational potential $\Phi(\mathbf{x})$ and, for simplicity, set
$A_\mu = 0$.

\subsection{Static clocks at different heights}

In the weak-field metric \eqref{eq:weak_field_metric} a static clock at
position $\mathbf{x}$ has worldline
\begin{equation}
  x^\mu(\tau) = \bigl(ct(\tau),\mathbf{x}\bigr), \qquad
  \frac{dx^i}{d\tau}=0.
\end{equation}
The proper time increment is
\begin{equation}
  d\tau = \sqrt{g_{00}(\mathbf{x})}\,dt
        \simeq \left(1 + \frac{\Phi(\mathbf{x})}{c^2}\right) dt,
\end{equation}
where $t$ is the coordinate time.

For an ideal phase clock obeying
\eqref{eq:ideal_clock_constraint} with $A_\mu=0$ we have
\begin{equation}
  \frac{d\phi}{d\tau} = \omega_0,
\end{equation}
so that
\begin{equation}
  \frac{d\phi}{dt}
  = \omega_0 \sqrt{g_{00}(\mathbf{x})}.
\end{equation}
The frequency of the clock as measured in coordinate time is therefore
\begin{equation}
  f(\mathbf{x}) \equiv \frac{1}{2\pi}\frac{d\phi}{dt}
  = \frac{\omega_0}{2\pi}\sqrt{g_{00}(\mathbf{x})}.
  \label{eq:local_frequency}
\end{equation}

Consider two identical clocks held at positions $\mathbf{x}_1$ and
$\mathbf{x}_2$ in the same static gravitational field. Their frequency
ratio is
\begin{equation}
  \frac{f_2}{f_1}
  = \sqrt{\frac{g_{00}(\mathbf{x}_2)}{g_{00}(\mathbf{x}_1)}}
  \simeq 1
    + \frac{\Phi(\mathbf{x}_2)-\Phi(\mathbf{x}_1)}{c^2},
  \label{eq:redshift_ratio}
\end{equation}
which is the standard gravitational redshift formula in the
weak-field limit. In the ``phase as fifth dimension'' language this
simply means that identical clocks have the same intrinsic phase
velocity $D\phi/D\tau=\omega_0$ along the fifth direction, but the
projection of this motion onto coordinate time $t$ is modulated by
$g_{00}(\mathbf{x})$.

\subsection{Outlook towards more general configurations}

The same geometric picture can be applied to more general situations:
\begin{itemize}
  \item moving clocks in static or time-dependent potentials;
  \item clocks coupled to non-trivial electromagnetic backgrounds,
        where the $kA_\mu u^\mu$ term in \eqref{eq:covariant_phase_rate}
        contributes to the phase evolution;
  \item interferometric setups where phase differences between
        alternative worldlines encode both gravitational and
        electromagnetic effects.
\end{itemize}
The details depend on the specific background $(g_{\mu\nu},A_\mu)$,
but the underlying building blocks remain those fixed in this Passport.

%======================================================================
\section{Discussion and outlook}
%======================================================================

\subsection{Phase as a conservative fifth dimension}

The phase-fibre framework offers a conservative way to talk about
gravity, electromagnetism and phase-based clocks within one geometric
structure, while keeping the central idea explicit:

\begin{quote}
Phase is treated as a compact fifth dimension, not as an informal label.
\end{quote}

More concretely:

\begin{itemize}
  \item The background sector is exactly Einstein--Maxwell theory.
        No additional light fields or exotic dynamics are introduced.
  \item The U(1) gauge symmetry of electromagnetism is realised as
        fibre reparametrisation along the phase coordinate $\phi$, which
        keeps track of phase in a geometrically transparent way.
  \item Phase clocks are modelled as test worldlines with a covariant
        phase rate fixed in proper time. Gravitational redshift and
        other phase effects are then read off from the same metric
        components that describe Newtonian gravity.
\end{itemize}

In this sense the framework can be viewed as a ``classical unification''
of geometry and phase: it merges the description of gravitational
potentials, electromagnetic potentials and phase evolution without
leaving the domain of standard classical physics, and makes the
$4D + \phi$ structure explicit.

\subsection{Practical advantages}

For subsequent work the main practical advantages are:

\begin{itemize}
  \item \textbf{Single language for many phenomena.} Weak gravity,
        redshift of clocks, electromagnetic couplings and phase
        interference effects can all be written in terms of the same
        few objects $(g_{\mu\nu},A_\mu,\phi)$.
  \item \textbf{Clean separation of background and clocks.} The
        background obeys the usual Einstein--Maxwell equations; clocks
        are test systems whose phase evolution is governed by
        \eqref{eq:covariant_phase_rate} and
        \eqref{eq:ideal_clock_constraint}. This makes it
        straightforward to analyse different experimental
        configurations by ``plugging in'' the relevant background
        solution.
  \item \textbf{Transparent Newtonian limit.} Newtonian gravity and
        gravitational redshift emerge directly from $g_{00}$, making it
        easy to connect to laboratory and astrophysical regimes.
\end{itemize}

\subsection{Step towards unified field descriptions}

Although the present framework is entirely classical and does not
attempt to quantise the fields, it can be seen as a step towards more
unified descriptions where phase plays a central role:

\begin{itemize}
  \item At the level of classical field theory, the phase-fibre picture
        encourages treating phase (and its gauge structure) on the same
        footing as spacetime geometry, as an explicit fifth dimension.
  \item It provides a natural stage for studying horizons, black holes,
        haloes, cosmology and interferometric experiments in terms of
        phase clocks and phase transport, to be developed in follow-up
        notes.
  \item It is compatible with a wide range of microscopic models for
        matter and condensed-matter inspired clocks (e.g.\
        superconducting devices), which can be embedded at the level of
        the matter Lagrangian.
\end{itemize}

The intent of this Technical Passport is to stabilise the core
assumptions and notation. Subsequent works in the PhaseGeometry
Phase-Fibre series (on specific clock configurations, Newtonian
phase-field haloes, phase-fibre black-hole analogues and related topics)
will build on this framework without modifying its basic structure.

%======================================================================
\section*{Core equations (quick reference)}
%======================================================================

For convenience we collect here the key relations fixed in this
Technical Passport:

\begin{itemize}
  \item \textbf{5D phase-fibre metric with phase as fifth dimension:}
        Eq.~\eqref{eq:5d_metric}.
  \item \textbf{Four-dimensional Einstein--Maxwell action:}
        Eq.~\eqref{eq:S_Einstein_Maxwell}, obtained from the 5D
        Einstein--Hilbert action \eqref{eq:S5} via dimensional
        reduction on the compact phase fibre $S^1$.
  \item \textbf{Covariant phase rate and ideal phase clock:}
        Eq.~\eqref{eq:covariant_phase_rate} together with the
        constraint \eqref{eq:ideal_clock_constraint}.
  \item \textbf{Local clock frequency and weak-field redshift:}
        Eq.~\eqref{eq:local_frequency} with
        $g_{00}(x) \simeq 1 + 2\Phi(x)/c^2$, and the corresponding
        frequency ratio \eqref{eq:redshift_ratio},
        \[
          \frac{f_2}{f_1}
          \simeq 1 + \frac{\Phi(\mathbf{x}_2)-\Phi(\mathbf{x}_1)}{c^2},
        \]
        so that in the limit $|\Phi|/c^2 \ll 1$ one has
        $\Delta f/f \simeq \Delta\Phi/c^2$.
\end{itemize}

These equations define the ``core'' of the Phase-Fibre framework ---
the minimal $4D + \phi$ structure with phase as a compact fifth
dimension --- used throughout the subsequent notes in the series.

%======================================================================
\begin{thebibliography}{99}
%======================================================================

\bibitem{PhaseFibreSeries}
A.~Turchanov,
\newblock \emph{PhaseGeometry Phase-Fibre Core Package v1.0},
\newblock Zenodo (2025),
DOI:\,\texttt{10.5281/zenodo.17783322}.

\end{thebibliography}

\end{document}

\documentclass[11pt,a4paper]{article}

\usepackage[utf8]{inputenc}
\usepackage[T1]{fontenc}
\usepackage[english]{babel}

\usepackage{amsmath,amssymb}
\usepackage{geometry}
\usepackage{graphicx}
\usepackage{subcaption}
\usepackage{hyperref}

\geometry{margin=1in}
\numberwithin{equation}{section}

\title{PhaseGeometry Phase-Fibre Foundations IV:\\[4pt]
Newtonian black-hole analogues from phase-fibre defects}

\author{Aleksey Turchanov}

\date{November 2025}

\begin{document}

\maketitle

\begin{center}
\small
Licensed under Creative Commons Attribution 4.0 International (CC BY 4.0).\\[2pt]
Part of the PhaseGeometry Phase-Fibre programme; archived in the Core Package\\
DOI: \texttt{10.5281/zenodo.17783322}.
\end{center}

\vspace{1em}

%======================================================================
\begin{abstract}
%======================================================================

We explore phase-fibre defects in a simple five-dimensional PhaseGeometry
setup and show how they generate black-hole-like gravitational fields in
the four-dimensional Newtonian limit. Starting from a 5D metric with a
compact phase fibre, we derive an effective phase-field description in 4D
where the energy density is sourced by gradients of a scalar phase field,
$\rho_{\rm eff} \propto |\nabla\varphi|^2$. We then construct localised,
radially symmetric ``phase bubbles'' and compute their effective mass
and Newtonian potential. Far away from the defect the potential approaches
the standard $-GM/r$ form, while the phase-gradient energy is concentrated
in a thin shell surrounding a ``hole'' in the phase fibre. In this picture
a black hole is not introduced as a fundamental singular object, but
appears as a localised defect of a phase fibre: its boundary carries
tension and the integrated phase energy plays the rôle of mass in the
Newtonian gravitational field.

The present note is part of the PhaseGeometry Phase-Fibre Foundations
programme. Foundations~0 (Technical Passport) fixes the 5D metric and
notation; Foundations~I (Conceptual Framework) presents the intuitive
picture of a compact phase fibre over 4D spacetime and its relation to
gravity, electromagnetism and phase clocks; Foundations~II provides the
technical derivation of the Einstein--Maxwell sector and simple clock
configurations; Foundations~III derives halo-like rotation curves from
extended phase textures. Here we show that the same minimal phase-field
setup also supports deep, compact defects whose exterior field is
indistinguishable from that of a point mass down to radii close to the
nominal Schwarzschild scale. In this sense the construction provides a
simple, fully transparent toy example of a ``black-hole analogue'' built
from a phase field within the unified Phase-Fibre language.
\end{abstract}

\newpage
\tableofcontents

%======================================================================
\section{Introduction}
%======================================================================

In conventional general relativity, a static black hole is described as a
region of spacetime bounded by an event horizon, with the exterior geometry
determined by a few macroscopic parameters. Many effective and analogue
models rephrase this textbook picture: black holes as bubbles of alternative
vacuum, gravastars with tension-supported shells, or defects in condensed
matter and optical media. In all these cases an underlying field or medium
is responsible for mimicking the gravitational behaviour of a black hole.

PhaseGeometry offers a different starting point. In the Phase-Fibre view,
gravity is organised by \emph{phase degrees of freedom} living over a
four-dimensional base spacetime. In the simplest Newtonian phase-field
limit of the PhaseGeometry programme (developed in the companion
Foundations~III note on Newtonian halo analogues), the gravitational
potential is sourced by gradients of a scalar phase field
$\varphi(\mathbf{x})$ through an effective energy density
$\rho_{\rm eff} \propto |\nabla\varphi|^2$. Extended phase textures with
$\rho_{\rm eff}(r)\propto 1/r^2$ naturally lead to halo-like rotation
curves with flat intermediate regions.

From this perspective it is natural to ask:

\medskip
\emph{What happens when the phase structure itself develops a localised
defect? Can a defect of a phase fibre behave gravitationally like a black
hole, already at the Newtonian level?}
\medskip

This note answers the question in a deliberately modest and transparent
setting. We introduce a five-dimensional phase-fibre ansatz in which ordinary
spacetime is a 4D base manifold and above each point there is a compact U(1)
phase fibre. We then construct a localised phase-fibre defect --- a
\emph{phase bubble} --- whose gradients generate a thin shell of phase
energy. In the Newtonian limit this shell produces a standard $1/r$
potential at large distances. A black-hole-like regime is approached when
the bubble radius becomes comparable to its Schwarzschild radius,
$r_s = 2GM_{\rm eff}/c^2$.

From a broader perspective, the present note sits at the intersection of
several existing ideas. On the one hand, there is a large literature on
compact objects supported by scalar or more exotic fields --- from boson
stars and scalar-field stars to gravastar-like and other ``exotic compact
objects'' (ECOs).\footnote{For reviews see e.g.\
F.~E.~Schunck and E.~W.~Mielke, Class.\ Quant.\ Grav.\ \textbf{20}, R301 (2003);
V.~Cardoso and P.~Pani, Living Rev.\ Relativ.\ \textbf{22}, 4 (2019).}
On the other hand, phenomenological dark-matter halo models often use simple
radial density profiles (isothermal, NFW, Burkert) to reproduce flat rotation
curves.\footnote{For standard halo profiles see e.g.\
J.~F.~Navarro, C.~S.~Frenk and S.~D.~M.~White, Astrophys.\ J.\ \textbf{462}, 563 (1996);
A.~Burkert, Astrophys.\ J.\ Lett.\ \textbf{447}, L25 (1995).}
The construction developed here is more modest and more minimal: we work
entirely in the Newtonian limit of general relativity, with a single
static phase field whose gradient energy plays the rôle of an effective
mass density. Nevertheless, this minimal setup already supports both
halo-like rotation curves (as in Foundations~III) and deep, compact defects
whose external field is indistinguishable from that of a Schwarzschild
mass down to radii close to the nominal horizon. In that sense the present
work can be viewed as a simple, fully transparent example of an exotic
compact object built from a phase field, but formulated in a language
tailored to the Phase-Fibre programme.

Within the Phase-Fibre line of the PhaseGeometry programme this note
sits above the core Foundations~0/I/II level. The accompanying
\emph{Foundations~0: Technical Passport} fixes the ontology and notation
of the five-dimensional phase-fibre framework in a compact, equation-driven
form. \emph{Foundations~I: Conceptual Framework} emphasises the physical
picture, intuitive cartoons and links to superconducting / Josephson systems.
\emph{Foundations~II: Field equations and phase clocks} develops the explicit
5D/4D action, derives the Einstein--Maxwell sector by dimensional reduction,
and works out simple redshift and Sagnac-type configurations for phase clocks.
\emph{Foundations~III: Newtonian halo analogues} shows how extended phase
textures reproduce halo-like rotation curves. The present Foundations~IV
extends this chain to black-hole analogues: we keep the same Newtonian
phase-field language, but focus on deep, compact defects.

The structure of the paper is as follows. In
Sec.~\ref{sec:phase_fibre_setup} we recall the phase-fibre setup and its
Newtonian phase-field limit. Section~\ref{sec:phase_bubble} constructs
a spherically symmetric phase bubble and computes its effective mass.
In Sec.~\ref{sec:bh_regime} we analyse the conditions under which the
resulting potential is practically indistinguishable from that of a point
mass down to radii close to $r_s$, and we discuss the corresponding
Newtonian redshift. Section~\ref{sec:visualisation_embedding} embeds the
construction into the full Phase-Fibre picture and uses simple contour
plots to visualise the configuration. Sec.~\ref{sec:discussion} closes
with a discussion of the advantages, limitations and relation to existing
work, and outlines directions for further development.

\subsection*{How to read this note}

This note is meant to be read as a conceptual story with a minimal amount
of formalism. The technical ingredients are deliberately simple and are
used only to support the physical picture. The key points to keep in mind
are:

\begin{itemize}
  \item In the Newtonian phase-fibre language the effective mass density
        is not put in by hand but built from a real phase field,
        $\rho_{\rm eff} \propto |\nabla\varphi|^2$. Phase gradients carry
        tension and gravitate.
  \item A ``black hole'' in this picture is represented as a \emph{phase
        cavity}: a hole in the phase fibre surrounded by a thin shell of
        phase-gradient tension. The integrated phase energy of this shell
        plays the rôle of the mass $M_{\rm eff}$.
  \item Outside the shell the Newtonian potential is indistinguishable
        from that of a point mass, $\Phi(r) \simeq -GM_{\rm eff}/r$.
        What changes is the internal structure: the centre is regular and
        the source is extended rather than singular.
  \item A simple tanh-bubble profile for $\varphi(r)$ already captures
        the whole cartoon: a hole in the phase fibre, a ring of
        phase-gradient energy, and the resulting $1/r$ potential. This
        is the minimal bridge between the Phase-Fibre language and
        familiar black-hole intuition, and it prepares the ground for
        fully relativistic, horizon-bearing configurations in later work.
\end{itemize}

%======================================================================
\section{Phase-fibre setup and Newtonian phase-field limit}
\label{sec:phase_fibre_setup}
%======================================================================

In this section we briefly recall the phase-fibre construction and the
Newtonian phase-field limit that we will use throughout. The goal is not
to re-derive the results of Foundations~0--II, but to fix notation and
provide a minimal self-contained background.

%----------------------------------------------------------------------
\subsection{Five-dimensional phase-fibre metric}
%----------------------------------------------------------------------

We consider a five-dimensional manifold $\mathcal{M}_5$ which is a U(1)
fibre bundle over four-dimensional spacetime $\mathcal{M}_4$. Local
coordinates on $\mathcal{M}_5$ are written as
\begin{equation}
  X^A = (x^\mu,\phi), \qquad \mu = 0,1,2,3,
\end{equation}
where $\phi$ is a periodic coordinate on the internal circle $S^1$.

The phase-fibre metric is taken in the Kaluza--Klein form
\begin{equation}
  ds_5^2
  = G_{AB} dX^A dX^B
  = g_{\mu\nu}(x)\,dx^\mu dx^\nu
    + R^2\bigl(d\phi + kA_\mu(x)\,dx^\mu\bigr)^2.
  \label{eq:5d_metric_PF_IV}
\end{equation}
Here $g_{\mu\nu}(x)$ is the four-dimensional spacetime metric and
$A_\mu(x)$ is a four-dimensional U(1) gauge field. The constants $R$
and $k$ set the scale of the fibre radius and the coupling of the fibre
coordinate to $A_\mu$; their combination is fixed in Foundations~0 by
matching to the four-dimensional Newton constant and electromagnetic
coupling.

Assuming that $g_{\mu\nu}$ and $A_\mu$ are independent of $\phi$, the
five-dimensional Einstein--Hilbert action reduces to the standard
Einstein--Maxwell action in four dimensions. In the present note we
work in the Newtonian limit of this setup, focusing on static, weak
gravitational fields sourced by an effective energy density associated
with phase gradients.

%----------------------------------------------------------------------
\subsection{Effective phase-field energy density}
%----------------------------------------------------------------------

In the Newtonian phase-field limit the central object is a scalar phase
field $\varphi(\mathbf{x})$ on three-dimensional space. The simplest
effective energy density associated with gradients of $\varphi$ is
\begin{equation}
  \rho_{\rm eff}(\mathbf{x})
  = \frac{\kappa}{c^2}\,|\nabla\varphi(\mathbf{x})|^2,
  \label{eq:rho_eff_def}
\end{equation}
where $\kappa$ is a phenomenological stiffness parameter. The factor
$1/c^2$ ensures that $\rho_{\rm eff}$ has the dimensions of mass density.

The total effective mass contained in a static configuration is
\begin{equation}
  M_{\rm eff}
  = \int d^3x\,\rho_{\rm eff}(\mathbf{x})
  = \frac{\kappa}{c^2}
    \int d^3x\,|\nabla\varphi(\mathbf{x})|^2.
  \label{eq:Meff_def}
\end{equation}
The Newtonian gravitational potential $\Phi(\mathbf{x})$ then satisfies
the usual Poisson equation
\begin{equation}
  \nabla^2\Phi(\mathbf{x})
  = 4\pi G \rho_{\rm eff}(\mathbf{x}).
  \label{eq:Poisson_rho_eff}
\end{equation}
At large distances from a localised configuration one recovers
\begin{equation}
  \Phi(r) \simeq -\frac{GM_{\rm eff}}{r}.
\end{equation}

This effective description can be viewed as the Newtonian shadow of
the full 5D phase-fibre setup: the phase field is a remnant of the
fibre coordinate, and the stiffness $\kappa$ encodes the relevant
combination of 5D parameters. For the purposes of this note we take
Eq.~\eqref{eq:rho_eff_def} as a phenomenological starting point.

%======================================================================
\section{Spherically symmetric phase-fibre bubbles}
\label{sec:phase_bubble}
%======================================================================

We now construct a minimal spherically symmetric phase defect and compute
its effective mass and potential. The picture is that of a ``bubble'' of
alternative phase: the phase field takes one asymptotic value inside the
bubble and another outside, with a smooth transition across a thin shell.

%----------------------------------------------------------------------
\subsection{Smooth phase profile}
%----------------------------------------------------------------------

For simplicity we consider a purely radial phase profile $\varphi(r)$
interpolating between two constants,
\begin{equation}
  \varphi(r \to 0) \to \varphi_{\rm in},
  \qquad
  \varphi(r \to \infty) \to \varphi_{\rm out}.
\end{equation}
A convenient smooth profile centred at radius $R$ with thickness
$\Delta$ is
\begin{equation}
  \varphi(r)
  = \varphi_{\rm in}
    + \frac{\Delta\varphi}{2}
      \left[
        1 + \tanh\!\left(\frac{r - R}{\Delta}\right)
      \right],
  \qquad
  \Delta\varphi \equiv \varphi_{\rm out} - \varphi_{\rm in}.
  \label{eq:phi_profile}
\end{equation}
Its radial derivative is
\begin{equation}
  \varphi'(r)
  = \frac{d\varphi}{dr}
  = \frac{\Delta\varphi}{2\Delta}\,
    \operatorname{sech}^2\!\left(\frac{r - R}{\Delta}\right).
  \label{eq:phi_prime_profile}
\end{equation}
The gradient is sharply localised in a shell of width $\sim\Delta$
around $r=R$; inside and outside the bubble the phase is nearly constant
and the gradient energy is negligible.

Substituting~\eqref{eq:phi_prime_profile} into
Eq.~\eqref{eq:rho_eff_def}, the effective energy density becomes
\begin{equation}
  \rho_{\rm eff}(r)
  = \frac{\kappa}{c^2}
    \left[\varphi'(r)\right]^2
  = \frac{\kappa}{c^2}
    \left(\frac{\Delta\varphi}{2\Delta}\right)^2
    \operatorname{sech}^4\!\left(\frac{r - R}{\Delta}\right).
  \label{eq:rho_eff_shell}
\end{equation}
This is a thin shell of phase energy wrapped around a nearly phase-flat
interior.

%----------------------------------------------------------------------
\subsection{Effective mass in the thin-shell limit}
%----------------------------------------------------------------------

The total effective mass of the phase bubble is
\begin{equation}
  M_{\rm eff}
  = 4\pi \int_0^\infty
      \rho_{\rm eff}(r)\,r^2\,dr.
  \label{eq:Meff_shell_def}
\end{equation}
In the thin-shell regime $\Delta \ll R$ we may approximate $r^2 \simeq R^2$
inside the narrow transition region and change variables to
$u = (r - R)/\Delta$. Using~\eqref{eq:rho_eff_shell} and
$dr = \Delta\,du$, we obtain
\begin{align}
  M_{\rm eff}
  &\simeq 4\pi R^2
    \frac{\kappa}{c^2}
    \left(\frac{\Delta\varphi}{2\Delta}\right)^2
    \Delta
    \int_{-\infty}^{+\infty}
      \operatorname{sech}^4 u\,du
  \nonumber\\[4pt]
  &= 4\pi R^2
     \frac{\kappa}{c^2}
     \left(\frac{\Delta\varphi}{2\Delta}\right)^2
     \Delta \cdot \frac{4}{3},
  \label{eq:Meff_shell_calc}
\end{align}
where we have used the standard integral
\begin{equation}
  \int_{-\infty}^{+\infty} \operatorname{sech}^4 u\,du
  = \frac{4}{3}.
\end{equation}
Thus
\begin{equation}
  M_{\rm eff}
  \simeq \frac{4\pi}{3}\,
         \frac{\kappa}{c^2}\,
         R^2\,\frac{(\Delta\varphi)^2}{\Delta}.
  \label{eq:Meff_shell_result}
\end{equation}

Several points are worth noting:
\begin{itemize}
  \item $M_{\rm eff}$ scales with the surface area $4\pi R^2$ of the bubble,
        as expected for a thin shell.
  \item The dependence on $\Delta$ is $M_{\rm eff} \propto 1/\Delta$:
        thinner shells store more gradient energy for a fixed jump
        $\Delta\varphi$.
  \item The combination $\kappa(\Delta\varphi)^2/\Delta$ plays the rôle
        of a surface tension scale for the phase bubble.
\end{itemize}
In particular, for given $\kappa$ and $\Delta\varphi$, one can adjust
$\Delta$ and $R$ to obtain a desired effective mass $M_{\rm eff}$.

%======================================================================
\section{Black-hole-like regime in the Newtonian picture}
\label{sec:bh_regime}
%======================================================================

We now ask under what conditions the phase bubble constructed above behaves,
for external observers, like a Newtonian point mass that is ``almost'' a
black hole. Since we remain in the Newtonian framework, there is no true
event horizon; instead, we look for a regime where the potential is
indistinguishable from $-GM_{\rm eff}/r$ down to radii comparable to the
Schwarzschild radius associated with $M_{\rm eff}$.

%----------------------------------------------------------------------
\subsection{Exterior potential and matching scale}
%----------------------------------------------------------------------

Let us denote the Schwarzschild radius associated with $M_{\rm eff}$ by
\begin{equation}
  r_s = \frac{2GM_{\rm eff}}{c^2}.
  \label{eq:rs_def}
\end{equation}
Outside the shell ($r \gg R + \Delta$) the potential is
\begin{equation}
  \Phi(r) \simeq -\frac{GM_{\rm eff}}{r},
  \qquad
  r \gtrsim R + \Delta.
\end{equation}
A black-hole-like regime is approached when the bubble radius $R$ is
only mildly larger than $r_s$,
\begin{equation}
  R \sim \mathcal{O}(1)\,r_s.
\end{equation}
In that case the potential near the shell is deep,
\begin{equation}
  |\Phi(R)| \simeq \frac{GM_{\rm eff}}{R}
  \sim \frac{c^2}{2},
\end{equation}
and the local escape velocity,
\begin{equation}
  v_{\rm esc}(R)
  = \sqrt{\frac{2GM_{\rm eff}}{R}},
\end{equation}
approaches $c$ from below. For an external observer restricted to
radii $r \gtrsim R$ the configuration behaves like a compact mass
very close to the black-hole threshold.

%----------------------------------------------------------------------
\subsection{Newtonian redshift of phase clocks}
%----------------------------------------------------------------------

Although true horizons and full relativistic redshift lie beyond the
Newtonian approximation, we can already see how a strong redshift of
phase clocks arises in this regime. In the weak-field limit the metric
component $g_{00}$ is
\begin{equation}
  g_{00}(\mathbf{x})
  \simeq 1 + \frac{2\Phi(\mathbf{x})}{c^2},
\end{equation}
and a phase clock at rest at radius $r$ ticks with coordinate-time
frequency
\begin{equation}
  f(r)
  \propto \sqrt{g_{00}(r)}
  \simeq 1 + \frac{\Phi(r)}{c^2}.
\end{equation}
For two identical clocks at radii $r_1$ and $r_2$ the fractional
frequency shift is
\begin{equation}
  \frac{\Delta f}{f}
  \simeq \frac{\Phi(r_2) - \Phi(r_1)}{c^2}.
\end{equation}
If $r_1$ lies close to $R \sim r_s$ while $r_2$ is far away, the
difference $\Phi(r_2) - \Phi(r_1)$ can approach $-\mathcal{O}(c^2)$,
leading to a large redshift factor. In this sense, even at the Newtonian
level, the phase bubble realises a strongly redshifted region, consistent
with its interpretation as a black-hole analogue.

%======================================================================
\section{Visualisation and embedding into the Phase-Fibre programme}
\label{sec:visualisation_embedding}
%======================================================================

So far we have treated the phase bubble as a purely Newtonian phase
configuration. In this section we briefly discuss how it embeds into
the full Phase-Fibre picture and how it can be visualised.

%----------------------------------------------------------------------
\subsection{Contour plots: phase, energy density and potential}
%----------------------------------------------------------------------

A particularly simple way to visualise the configuration is to plot
contours of the phase field $\varphi(r)$, the effective density
$\rho_{\rm eff}(r)$ and the Newtonian potential $\Phi(r)$ in a
meridional $(r,\theta)$ plane. In such plots:
\begin{itemize}
  \item the phase $\varphi$ is nearly uniform inside and outside
        the bubble, with a sharp transition at $r \simeq R$;
  \item the effective density $\rho_{\rm eff}$ is concentrated in a
        thin shell around $r = R$, representing the phase tension;
  \item the potential $\Phi$ is shallow at large radii and becomes
        deep near the shell, approaching the familiar $-GM_{\rm eff}/r$
        profile.
\end{itemize}

\begin{figure}[t]
  \centering

  \begin{subfigure}{0.32\textwidth}
    \centering
    \includegraphics[width=\linewidth]{5D_IV_1.png}
    \caption{Phase field lines around a circular cavity.}
  \end{subfigure}
  \hfill
  \begin{subfigure}{0.32\textwidth}
    \centering
    \includegraphics[width=\linewidth]{5D_IV_2.png}
    \caption{Field lines and circular equipotentials.}
  \end{subfigure}
  \hfill
  \begin{subfigure}{0.32\textwidth}
    \centering
    \includegraphics[width=\linewidth]{5D_IV_3.png}
    \caption{Ring of phase-gradient energy $|\nabla\varphi|^2$.}
  \end{subfigure}

  \caption{Phase cavity in the Newtonian phase-field model.
  \textbf{Left:} schematic phase cavity in the $(x,y)$ plane. The central
  white disk is a circular defect where the phase fibre is removed, while the
  teal solid and dashed curves show orthogonal phase coordinates that solve
  the Laplace equation in the exterior region. \textbf{Middle:} the same
  phase field lines together with circular equipotentials (orange) representing
  a spherically symmetric Newtonian potential sourced by the cavity.
  \textbf{Right:} radial ring of phase-gradient energy
  $|\nabla\varphi|^2 \propto \mathrm{sech}^4[(r-R)/\Delta]$ generated by a
  tanh-bubble profile $\varphi(r)=\varphi_0\tanh[(r-R)/\Delta]$, localizing
  the effective mass in a thin shell around $r\simeq R$. Together, these
  panels summarize our cartoon: a hole in the phase fibre, a ring of phase
  tension, and the resulting $1/r$ potential.}
  \label{fig:phase_bubble_contours}
\end{figure}

%----------------------------------------------------------------------
\subsection{Embedding into the 5D Phase-Fibre language}
%----------------------------------------------------------------------

From the full Phase-Fibre viewpoint, the scalar field $\varphi$ in the
Newtonian limit is a low-energy remnant of the compact fibre coordinate
$\phi$ in the 5D metric~\eqref{eq:5d_metric_PF_IV}. A phase bubble
corresponds to a configuration where the ``preferred'' phase along the
fibre changes from $\varphi_{\rm in}$ to $\varphi_{\rm out}$ across
a shell, and the fibre itself may shrink or deform in the core region.
The effective density~\eqref{eq:rho_eff_def} then encodes the stiffness
or tension associated with gradients along the fibre.

This interpretation ties the present construction back to the earlier
Foundations notes:
\begin{itemize}
  \item Foundations~0 and~I fix the 5D metric and the conceptual picture
        of a compact phase fibre over spacetime.
  \item Foundations~II shows how phase clocks move along the fibre and
        how their frequencies are affected by $g_{00}$ and $A_\mu$.
  \item Foundations~III demonstrates that extended phase textures can
        reproduce halo-like rotation curves.
  \item The present Foundations~IV shows that localised phase defects
        generate black-hole-like Newtonian potentials, completing the
        minimal diagram
        \[
          \text{phase fibre} \;\Rightarrow\;
          \text{phase field} \;\Rightarrow\;
          \{\text{haloes, black-hole analogues}\}.
        \]
\end{itemize}
Taken together, these notes provide a unified geometric language in
which weak gravity, dark-matter-like haloes and black-hole-like sources
are different manifestations of the same underlying phase structure.

%======================================================================
\section{Discussion and outlook}
\label{sec:discussion}
%======================================================================

In this note we have shown that a very simple phase-field setup in the
Phase-Fibre framework already supports configurations that behave, for
external observers, like Newtonian black-hole analogues. A single static
phase profile $\varphi(r)$, together with the effective density
\eqref{eq:rho_eff_def} and the Poisson equation \eqref{eq:Poisson_rho_eff},
is sufficient to construct a phase bubble whose exterior potential is
$-GM_{\rm eff}/r$ and whose core is regular.

%----------------------------------------------------------------------
\subsection{Take-home message}
%----------------------------------------------------------------------

The key points can be summarised as follows:
\begin{itemize}
  \item A smooth phase-bubble profile
        \eqref{eq:phi_profile}--\eqref{eq:rho_eff_shell}
        generates a thin shell of phase-gradient energy around radius $R$.
  \item In the thin-shell limit the effective mass is given by
        Eq.~\eqref{eq:Meff_shell_result}, scaling with the surface area
        $4\pi R^2$ and inversely with the shell thickness $\Delta$.
  \item For $R$ only slightly larger than the Schwarzschild radius
        $r_s = 2GM_{\rm eff}/c^2$ the bubble behaves as a compact mass
        whose escape velocity approaches $c$ near the shell and whose
        potential produces strong Newtonian redshifts.
  \item From the Phase-Fibre viewpoint, the configuration is simply a
        localised defect of the phase fibre whose boundary carries
        tension; the integrated phase energy plays the role of mass.
\end{itemize}

In the broader PhaseGeometry programme, this provides a concrete example
of how the same phase degrees of freedom that govern haloes in
Foundations~III can also generate black-hole-like sources, all within a
single geometric language based on phase fibres.

%----------------------------------------------------------------------
\subsection{What we do not claim}
%----------------------------------------------------------------------

It is equally important to state explicitly what we \emph{do not} claim:

\begin{itemize}
  \item We do not claim to have constructed a new fundamental theory of
        black holes. The present model is a classical, Newtonian toy
        construction built on top of standard gravity.
  \item We do not introduce exotic matter or modify general relativity.
        The only new ingredient is an effective phase field whose
        gradient energy sources the usual Poisson equation.
  \item We do not model horizons, Hawking radiation or genuinely quantum
        aspects of black-hole physics. The analysis remains in the
        weak-field, Newtonian regime, even when the potential is taken
        to be deep.
  \item We do not claim specific new observational signatures that would
        distinguish these phase bubbles from standard compact objects.
        The construction is intended as a proof-of-principle within the
        PhaseGeometry paradigm, not as a fully realistic model of
        astrophysical black holes.
\end{itemize}

Within these boundaries, the phase-bubble setup provides a clean and
controllable playground for phase-based black-hole analogues.

%----------------------------------------------------------------------
\subsection{Relation to existing work}
\label{subsec:relation_existing}
%----------------------------------------------------------------------

It is useful to place the present construction within the broader
landscape of compact-object and halo models. Conceptually, the closest
relatives are scalar-field compact objects (such as boson stars) and
more general exotic compact objects (ECOs), where the gravitational
field is sourced not by ordinary matter but by non-trivial field
configurations. In many such models the exterior spacetime can be very
close to Schwarzschild, while the interior structure differs
dramatically from a black hole, sometimes avoiding an event horizon
entirely.

The approach taken here is deliberately more conservative and more
minimal. We stay in the Newtonian limit of general relativity, work
with a static phase field whose gradient energy defines an effective
mass density, and use only standard Poisson gravity. Within this simple
framework we find that:
(i) extended phase defects can reproduce halo-like circular-velocity
profiles with a constant-density core, a flat intermediate region and
a Keplerian fall-off (as shown in Foundations~III); and
(ii) sufficiently deep and compact defects behave, for external observers,
like point masses with an effective Schwarzschild radius
$r_s = 2GM_{\rm eff}/c^2$, leading to strong redshifts and escape
velocities approaching the speed of light.

In that sense the present note provides a fully transparent toy example
of an exotic compact object built from a phase field: the exterior
potential is exactly the Newtonian $-GM_{\rm eff}/r$, while the internal
structure is encoded in the phase texture and its gradient energy. At
the same time, the link to the Phase-Fibre programme keeps the
construction anchored in a unified geometric language where
$g_{\mu\nu}$, $A_\mu$ and the phase fibre are treated on the same
footing. Readers familiar with standard halo profiles and scalar compact
objects may view this work as a minimal phase-field analogue, tailored
to a Newtonian regime and to the needs of the PhaseGeometry series
rather than to full-fledged model building in relativistic astrophysics.

%----------------------------------------------------------------------
\subsection{Future directions}
%----------------------------------------------------------------------

Several natural extensions suggest themselves:
\begin{itemize}
  \item Derive the phase profile $\varphi(r)$ and the corresponding
        fibre geometry from a concrete 5D action, rather than treating
        them phenomenologically.
  \item Go beyond the Newtonian approximation and study the backreaction
        of phase-fibre defects on the full 4D metric, including the
        possible formation of horizons and quasi-static relativistic
        solutions.
  \item Investigate time-dependent phase bubbles, mergers and
        oscillations as toy models for gravitational collapse and
        ringdown in the phase-fibre language.
  \item Explore multi-defect configurations and phase-fibre lattices as
        possible analogues of multi-black-hole systems and
        dark-matter-like structures in a unified phase-field setting.
\end{itemize}

Overall, the present construction should be read as a classical,
Newtonian toy model rather than as a proposal for a new theory of black
holes. We stay entirely within standard gravity, work with static phase
profiles in the weak-field limit, and use them to generate effective
mass shells whose exterior potential mimics that of a point mass.
Horizons, Hawking radiation and genuinely quantum aspects of black-hole
physics lie outside the scope of this note. Any future claims about new
observational signatures or fully relativistic Phase-Fibre black holes
would require a separate, explicitly relativistic and quantum layer,
built on top of the conservative framework developed here.


%======================================================================
\section*{Acknowledgements}
%======================================================================

This work is part of the ongoing PhaseGeometry programme exploring
phase variables and higher-dimensional fibres as carriers of
gravitational phenomena.

%======================================================================
\begin{thebibliography}{99}
%======================================================================

\bibitem{PhaseFibreSeries}
A.~Turchanov,
\newblock \emph{PhaseGeometry Phase-Fibre series: overview and core framework},
\newblock Zenodo series record (2025),
DOI:\,\texttt{10.5281/zenodo.17736918}.

\bibitem{SchunckMielke}
F.~E.~Schunck and E.~W.~Mielke,
\newblock ``General relativistic boson stars'',
\newblock Class.\ Quant.\ Grav.\ \textbf{20}, R301 (2003).

\bibitem{CardosoPani}
V.~Cardoso and P.~Pani,
\newblock ``Testing the nature of dark compact objects: a status report'',
\newblock Living Rev.\ Relativ.\ \textbf{22}, 4 (2019).

\bibitem{NFW}
J.~F.~Navarro, C.~S.~Frenk and S.~D.~M.~White,
\newblock ``The Structure of Cold Dark Matter Halos'',
\newblock Astrophys.\ J.\ \textbf{462}, 563 (1996).

\bibitem{Burkert}
A.~Burkert,
\newblock ``The Structure of Dark Matter Halos in Dwarf Galaxies'',
\newblock Astrophys.\ J.\ Lett.\ \textbf{447}, L25 (1995).

\end{thebibliography}

\end{document}

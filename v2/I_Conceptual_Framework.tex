\documentclass[11pt,a4paper]{article}

\usepackage[utf8]{inputenc}
\usepackage[T1]{fontenc}
\usepackage[english]{babel}

\usepackage{amsmath,amssymb}
\usepackage{geometry}
\usepackage{hyperref}
\geometry{margin=1in}

\numberwithin{equation}{section}

\title{PhaseGeometry Phase-Fibre Foundations I:\\[4pt]
Phase as a genuine fifth coordinate:\\[2pt]
A 5D phase-fibre description of gravity, electromagnetism and phase clocks\\[4pt]
Conceptual Framework}

\author{Aleksey Turchanov}
\date{November 2025}

\begin{document}

\maketitle

\begin{center}
\small
Licensed under Creative Commons Attribution 4.0 International (CC BY 4.0).\\[2pt]
Part of the PhaseGeometry Phase-Fibre programme; archived in the Core Package\\
DOI: \texttt{10.5281/zenodo.17783322}.
\end{center}

\vspace{1em}

\begin{abstract}
We introduce a deliberately conservative five-dimensional framework in which
phase is promoted to a \emph{genuine fifth coordinate} on the same footing as
space and time. Ordinary four-dimensional spacetime of general relativity is
extended by a compact internal U(1) phase coordinate $\phi$, forming a circle
fibre over each spacetime point. In this $(x^\mu,\phi)$ description the world
is naturally viewed as $3D + t + \phi$: three spatial coordinates answer
``where'', time answers ``when'', and the phase coordinate $\phi$ answers
``in which stage of the process'' a system currently is. Throughout, phase is
treated as a fully introduced fifth dimension with its own 5D metric structure,
rather than as an auxiliary scalar field living on $M_4$.

On this five-dimensional bundle we adopt a standard Kaluza--Klein ansatz: the
electromagnetic potential $A_\mu$ arises as the connection along the phase
fibre, while the metric component $g_{00}$ controls the rate of macroscopic
clocks, including superconducting/Josephson phase clocks. In the Newtonian
limit the gravitational potential $\Phi(\mathbf{x})$ is fully encoded in
$g_{00}$ and can be sourced by a simple static phase configuration
$\varphi(\mathbf{x})$. The corresponding effective mass density is
proportional to $|\nabla\varphi|^2$, in close analogy with London and
Ginzburg--Landau descriptions of superconductors. This yields a transparent
geometric interpretation of weak gravity and gravitational redshift in terms
of ``phase texture'' in the additional dimension.

Dynamically, the framework remains strictly within standard classical physics:
weak-field general relativity, Maxwell electrodynamics and effective
superconductivity/Josephson theory. The five-dimensional phase-fibre geometry
does not introduce new fields or interactions and does not modify Einstein's
equations or the Standard Model; it provides an equivalent but more unified
geometric language in which gravity, electromagnetism and phase-based clocks
are described by the same $(x^\mu,\phi)$ structure. The central conceptual
move is therefore not to change the laws, but to make the ubiquitous phase
variable explicit as a fifth coordinate and to use this 5D picture as the
organising backbone for the PhaseGeometry Phase-Fibre series.
\end{abstract}

\newpage
\tableofcontents

\vspace{1em}
%======================================================================
\section{Introduction and scope}
%======================================================================

Weak gravity, electromagnetism and precision clocks are three pillars of
modern physics that coexist in many practical experiments but are usually
discussed in quite different languages. In weak gravitational fields the
metric is close to Minkowski, and to leading order the only component
that ordinary clocks ``feel'' is
\begin{equation}
  g_{00}(\mathbf{x}) \simeq 1 + \frac{2\Phi(\mathbf{x})}{c^2},
\end{equation}
where $\Phi(\mathbf{x})$ is the Newtonian gravitational potential. This
component controls gravitational time dilation and redshift.
Electromagnetism lives its own life, governed by the gauge potential
$A_\mu$ and the field tensor $F_{\mu\nu}$, while the most accurate
clocks available today --- atomic fountains, optical lattice clocks,
Josephson oscillators --- are in essence phase devices: we measure the
rate of growth of some macroscopic phase $\theta(t)$ rather than the
ticking of a mechanical pointer.

This is particularly evident in superconductivity. A Josephson junction
realises the ac Josephson relation
\begin{equation}
  f_J = \frac{2e}{h}\,V,
\end{equation}
with relative accuracy at the level of $10^{-19}$--$10^{-20}$, and
responds simultaneously to voltage, electromagnetic fields and
gravitational potential. One and the same device can function as a
high-precision clock, a detector of weak gravity (through gravitational
redshift) and a macroscopic superconducting phase element.

The central proposal of this note is that the phase variable underlying
such devices should be treated not merely as an auxiliary label, but as a
\emph{genuine fifth coordinate} of the physical description. Concretely,
we attach to each spacetime point $x^\mu$ a compact phase circle
parameterised by $\phi \sim \phi + 2\pi$, forming a five-dimensional
bundle with topology $\mathcal{M}_4 \times S^1$. In this $3D + t + \phi$
view the usual spacetime coordinates say where and when an event occurs,
while the phase coordinate $\phi$ encodes where the system sits along its
internal cycle. We call the resulting geometric picture
\emph{phase-fibre geometry}. In this picture phase is a fully introduced
fifth dimension with its own 5D metric, on the same mathematical footing
as the usual spacetime coordinates.

On this bundle we introduce a Kaluza--Klein metric in which the
electromagnetic potential $A_\mu$ enters as the connection on the phase
fibre, while the base metric $g_{\mu\nu}$ carries the usual gravitational
information. Phase clocks are then described as objects that move
uniformly along the fibre with respect to their proper time $\tau$, and
gravitational redshift appears as a simple consequence of the relation
between $\tau$ and coordinate time $t$. In the Newtonian sector the same
phase framework reproduces a phase-field description where static phase
textures source weak gravitational potentials.

This note is intentionally conservative. We do not introduce new fields
or interactions; we do not modify Einstein's equations; we do not claim
to derive superconductivity from geometry. Instead, we take standard
ingredients from general relativity, Maxwell theory and effective
superconductivity, and reorganise them in a way that makes the interplay
between $g_{00}$, $A_\mu$ and phase-based clocks particularly transparent.
The five-dimensional geometry is used as a \emph{repackaging} of known
physics in which the role of phase is made explicit as a fifth coordinate,
rather than as a hidden variable inside complex amplitudes and boundary
conditions.

Within the Phase-Fibre line of the PhaseGeometry programme this paper
occupies a conceptual ``middle layer''. The accompanying
\emph{Foundations 0: Technical Passport} fixes the core ontology and
notation of the five-dimensional phase-fibre framework in a compact,
equation-driven form, while \emph{Foundations II: Field Equations and
Phase-Clock Configurations} develops the explicit action, field
equations and concrete examples. The present note uses the same
conventions as Foundations~0, but keeps the emphasis on physical
interpretation, intuitive cartoons and links to superconducting /
Josephson systems rather than on formal derivations.

This note is part of the broader \emph{PhaseGeometry} series. The
Newtonian phase-field sector, where static phase configurations generate
weak gravitational potentials and clock redshifts, is developed in a
separate companion work (PhaseGeometry Part~III, Ref.~\cite{TurchanovPhaseFieldPartIII}).
The present paper is deliberately conceptual and phenomenological: a more
technical companion, ``Phase-Fibre Geometry~II'', derives the 5D/4D
field equations from an explicit action and works out concrete
phase-clock configurations (two-height redshift, rotating loops, and
simple Josephson/SQUID-based scenarios).

In what follows we adopt the following notation consistently:
$\theta$ denotes the condensate phase of superconducting matter,
$\phi$ is the compact fibre coordinate on the U(1) phase circle in the
5D construction, and $\varphi(\mathbf{x})$ is the real Newtonian phase
field used in the static, weak-field analysis of Part~III.

The structure of the paper is as follows. In
Sec.~\ref{sec:standard_ingredients} we collect the standard weak-field
GR, Maxwell and phase-clock ingredients. Section~\ref{sec:5d_ansatz}
introduces the five-dimensional phase-fibre ansatz and recalls the
reduction to the Einstein--Maxwell action. In
Sec.~\ref{sec:phase_clocks_pf} we show how coherent matter and
phase-based clocks fit into this language and how gravitational
redshift appears. Sec.~\ref{sec:phase_field_newtonian} summarises the
Newtonian phase-field sector and its embedding into the phase-fibre
picture. Sec.~\ref{sec:scenarios} discusses simple illustrative
scenarios, and Sec.~\ref{sec:discussion} closes with a brief discussion
and outlook.

%======================================================================
\section{Standard ingredients: weak-field GR, Maxwell and phase clocks}
\label{sec:standard_ingredients}
%======================================================================

In this section we briefly recall the standard building blocks that we
will reorganise in the phase-fibre picture: weak-field general
relativity, Maxwell electrodynamics and phase-based clock dynamics in
superconductors. We do not attempt to be exhaustive; for detailed
background see e.g.\ Refs.~\cite{Carroll,Wald,Jackson,Tinkham}.

%----------------------------------------------------------------------
\subsection{Weak-field GR and $g_{00}$}
%----------------------------------------------------------------------

In the weak-field, slow-motion regime around a static mass distribution
the spacetime metric can be written as
\begin{equation}
  ds^2 = g_{\mu\nu} dx^\mu dx^\nu
  = g_{00}(\mathbf{x})\,c^2 dt^2 - g_{ij}(\mathbf{x})\,dx^i dx^j,
\end{equation}
with
\begin{equation}
  g_{00}(\mathbf{x}) \simeq 1 + \frac{2\Phi(\mathbf{x})}{c^2},
  \qquad
  g_{ij}(\mathbf{x}) \simeq \delta_{ij}
\end{equation}
and $\Phi(\mathbf{x})$ the Newtonian potential satisfying the Poisson
equation
\begin{equation}
  \nabla^2 \Phi(\mathbf{x}) = 4\pi G \rho(\mathbf{x}).
\end{equation}
For a worldline $x^\mu(\tau)$ of a slowly moving clock the proper time
increment is
\begin{equation}
  d\tau = \sqrt{g_{00}(\mathbf{x})}\,dt
  \simeq \left(1 + \frac{\Phi(\mathbf{x})}{c^2}\right) dt,
\end{equation}
so that the rate of the clock with respect to coordinate time $t$
depends on $g_{00}(\mathbf{x})$ and hence on $\Phi(\mathbf{x})$.
The gravitational redshift between two heights $z_1$ and $z_2$ in a
uniform field $g$ with $\Phi(z) \simeq gz$ is
\begin{equation}
  \frac{\Delta f}{f}
  \equiv \frac{f(z_2) - f(z_1)}{f(z_1)}
  \simeq \frac{\Phi(z_2) - \Phi(z_1)}{c^2}
  \simeq \frac{g(z_2 - z_1)}{c^2}.
\end{equation}

%----------------------------------------------------------------------
\subsection{Maxwell electrodynamics}
%----------------------------------------------------------------------

Classical electromagnetism in curved spacetime is described by the
field tensor $F_{\mu\nu} = \partial_\mu A_\nu - \partial_\nu A_\mu$
and the Maxwell action
\begin{equation}
  S_{\rm EM}
  = -\frac{1}{4\mu_0}
    \int d^4x\,\sqrt{-g}\,F_{\mu\nu} F^{\mu\nu}.
\end{equation}
The Einstein--Hilbert action with a cosmological constant neglected is
\begin{equation}
  S_{\rm EH}
  = \frac{1}{16\pi G}
    \int d^4x\,\sqrt{-g}\,R_4,
\end{equation}
and their sum gives the standard Einstein--Maxwell theory.

Gauge transformations act as
\begin{equation}
  A_\mu(x) \mapsto A_\mu(x) - \partial_\mu \chi(x),
\end{equation}
leaving $F_{\mu\nu}$ invariant. In the phase-fibre picture these
transformations will be reinterpreted as coordinate changes along the
fibre.

%----------------------------------------------------------------------
\subsection{Superconducting phase and Josephson clocks}
%----------------------------------------------------------------------

In effective superconductivity the condensate is described by a complex
field
\begin{equation}
  \Psi(\mathbf{x},t) = |\Psi(\mathbf{x},t)|\,e^{i\theta(\mathbf{x},t)},
\end{equation}
with the phase $\theta$ minimally coupled to the electromagnetic
potential. In the simplest Ginzburg--Landau/London-like description the
relevant combination is
\begin{equation}
  \partial_\mu\theta - q A_\mu,
\end{equation}
with $q=2e/\hbar$ for Cooper pairs. Supercurrents and Meissner screening
are encoded in this gauge-covariant derivative.

A Josephson junction between two superconductors with phases
$\theta_1$ and $\theta_2$ supports the relations
\begin{align}
  I_s &= I_c \sin\Delta\theta,
  \\
  \frac{d}{dt}\Delta\theta &= \frac{2e}{\hbar} V,
\end{align}
where $\Delta\theta = \theta_2 - \theta_1$ is the phase difference and
$V$ is the voltage across the junction. The ac Josephson relation for
the oscillation frequency is
\begin{equation}
  f_J = \frac{1}{2\pi}\frac{d}{dt}\Delta\theta
  = \frac{2e}{h}V.
\end{equation}
In this sense a Josephson junction is a phase clock: its observable
frequency directly measures the time derivative of a phase.

%======================================================================
\section{5D phase-fibre ansatz and reduction to Einstein--Maxwell}
\label{sec:5d_ansatz}
%======================================================================

We now extend the four-dimensional spacetime $(\mathcal{M}_4,g_{\mu\nu})$
by a compact U(1) phase circle $S^1$ with coordinate
$\phi \sim \phi + 2\pi$, forming a five-dimensional manifold
$\mathcal{M}_5 \simeq \mathcal{M}_4 \times S^1$. On this manifold we
introduce the Kaluza--Klein metric
\begin{equation}
  ds_5^2 = G_{AB}\,dX^A dX^B
  = g_{\mu\nu}(x)\,dx^\mu dx^\nu
    + R^2\bigl(d\phi + kA_\mu(x)\,dx^\mu\bigr)^2,
  \label{eq:5d_metric_pf}
\end{equation}
where $R$ is the radius of the fibre and $k$ is a constant that sets the
normalisation of the electromagnetic potential. The indices $A,B$ run
over the five coordinates $(x^\mu,\phi)$.

Under a shift of the fibre coordinate
\begin{equation}
  \phi \mapsto \phi' = \phi + \chi(x),
\end{equation}
the one-form $d\phi + kA_\mu dx^\mu$ transforms as
\begin{equation}
  d\phi' + k A'_\mu dx^\mu
  = d\phi + d\chi + kA'_\mu dx^\mu.
\end{equation}
Requiring that the metric $ds_5^2$ remains invariant implies the gauge
transformation
\begin{equation}
  A_\mu \mapsto A_\mu' = A_\mu - \frac{1}{k}\,\partial_\mu\chi(x).
\end{equation}
Thus the usual U(1) gauge transformations arise as coordinate changes
along the phase fibre.

\begin{figure}[t]
  \centering
  \includegraphics[width=0.8\textwidth]{5D_1.png}
  \caption{Phase-fibre picture: four-dimensional spacetime
  $(\mathcal{M}_4,g_{\mu\nu})$ as the base manifold, with a compact
  U(1) phase circle attached at each point. The electromagnetic
  potential $A_\mu(x)$ plays the role of a connection on this phase
  fibre, entering the five-dimensional metric as in
  Eq.~\eqref{eq:5d_metric_pf}.}
  \label{fig:phase_fibre_cartoon}
\end{figure}

If we start from the five-dimensional Einstein--Hilbert action
\begin{equation}
  S_5
  = \frac{1}{16\pi G_5}
    \int d^5X\,\sqrt{-G}\,R_5,
\end{equation}
insert the ansatz \eqref{eq:5d_metric_pf}, and assume that all fields
are independent of $\phi$, the five-dimensional Ricci scalar reduces to
\begin{equation}
  R_5 = R_4 - \frac{1}{4}(kR)^2 F_{\mu\nu}F^{\mu\nu},
\end{equation}
up to total derivatives. Integrating over the fibre coordinate $\phi$
then yields
\begin{equation}
  S_5 = \frac{2\pi R}{16\pi G_5}
    \int d^4x\,\sqrt{-g}\,
    \left(R_4 - \frac{1}{4}(kR)^2 F_{\mu\nu}F^{\mu\nu}\right).
\end{equation}
Identifying
\begin{equation}
  \frac{1}{16\pi G} = \frac{2\pi R}{16\pi G_5},
  \qquad
  \frac{1}{4\mu_0} = \frac{2\pi R}{16\pi G_5}\,\frac{1}{4}(kR)^2,
\end{equation}
we recover the standard four-dimensional Einstein--Maxwell action.
Thus, at the level of the action, the five-dimensional phase-fibre
ansatz reproduces Einstein--Maxwell with no new dynamical content. Its
value lies in the geometric interpretation of $A_\mu$ as a connection
on the phase fibre and in the natural placement of phase clocks in this
geometry, with phase promoted to an explicit fifth coordinate.

%======================================================================
\section{Coherent matter and phase clocks in phase-fibre geometry}
\label{sec:phase_clocks_pf}
%======================================================================

We now show how coherent matter and phase-based clocks can be described
in the phase-fibre picture. The key idea is to identify the condensate
phase $\theta$ with the fibre coordinate $\phi$ in a unitary gauge and
to interpret the usual combination $\partial_\mu\theta - qA_\mu$ as
coming from parallel transport along the fibre.

%----------------------------------------------------------------------
\subsection{Phase covariant derivative from 5D connection}
%----------------------------------------------------------------------

Consider a worldline $X^A(\lambda) = (x^\mu(\lambda),\phi(\lambda))$
in the five-dimensional spacetime with metric \eqref{eq:5d_metric_pf}.
The tangent vector is
\begin{equation}
  \dot{X}^A = (\dot{x}^\mu,\dot{\phi}),
\end{equation}
and the line element along the trajectory reads
\begin{equation}
  ds_5^2
  = g_{\mu\nu} \dot{x}^\mu \dot{x}^\nu\,d\lambda^2
    + R^2\bigl(\dot{\phi} + kA_\mu\dot{x}^\mu\bigr)^2 d\lambda^2.
\end{equation}
For a massive particle the action proportional to the proper length
is
\begin{equation}
  S_{\rm worldline}
  = -m \int d\lambda\,
    \sqrt{g_{\mu\nu}\dot{x}^\mu \dot{x}^\nu
          + R^2\bigl(\dot{\phi} + kA_\mu\dot{x}^\mu\bigr)^2}.
\end{equation}
In the non-relativistic limit and in a unitary gauge where we identify
$\theta \equiv \phi$, the combination
\begin{equation}
  D_\mu\phi \equiv \partial_\mu\phi + kA_\mu
\end{equation}
is directly analogous to the Ginzburg--Landau/London combination
$\partial_\mu\theta - qA_\mu$, with $q = -k$ (in appropriate units).
The five-dimensional geometric structure therefore reproduces the
phase covariant derivative of superconductivity as the projection
of parallel transport along the fibre.

%----------------------------------------------------------------------
\subsection{Proper time, phase rate and gravitational redshift}
%----------------------------------------------------------------------

To describe phase clocks we adopt a simple phenomenological rule:
an ideal phase clock is an object that advances uniformly along the
fibre with respect to its proper time $\tau$,
\begin{equation}
  \frac{d\phi}{d\tau} = \omega_0 = \text{const}.
\end{equation}
In a static spacetime with metric
\begin{equation}
  ds^2 = g_{00}(\mathbf{x})\,c^2 dt^2 - g_{ij}\,dx^i dx^j,
\end{equation}
and for a clock at rest in the chosen coordinate system, we have
\begin{equation}
  d\tau = \sqrt{g_{00}(\mathbf{x})}\,dt.
\end{equation}
The observable frequency of the clock with respect to coordinate
time is
\begin{equation}
  \frac{d\phi}{dt}
  = \frac{d\phi}{d\tau}\,\frac{d\tau}{dt}
  = \omega_0 \sqrt{g_{00}(\mathbf{x})}
  \simeq \omega_0\left(1 + \frac{\Phi(\mathbf{x})}{c^2}\right).
\end{equation}
Thus the gravitational redshift is immediately
\begin{equation}
  \frac{\Delta f}{f}
  = \frac{\Delta\omega}{\omega_0}
  \simeq \frac{\Delta\Phi}{c^2},
\end{equation}
in agreement with the standard weak-field result. In this language
the internal angular rate $\omega_0$ is universal and independent of
gravity, while gravity merely changes the ``gear ratio'' between
proper and coordinate time.

%======================================================================
\section{Phase-field Newtonian sector and its embedding}
\label{sec:phase_field_newtonian}
%======================================================================

The phase-field Newtonian sector developed in Part~III of the
PhaseGeometry series provides a purely static, non-relativistic
playground in which gravity and phase clocks are controlled by a real
scalar phase field $\varphi(\mathbf{x})$ in three-dimensional space.
Here we briefly summarise its structure and explain how it fits into
the phase-fibre picture.

%----------------------------------------------------------------------
\subsection{Static phase field and effective mass density}
%----------------------------------------------------------------------

In the Newtonian phase-field model we consider a real scalar field
$\varphi(\mathbf{x})$ with an energy density dominated by its spatial
gradients,
\begin{equation}
  \rho_{\rm eff}(\mathbf{x})
  = \frac{\kappa}{c^2}\,|\nabla\varphi(\mathbf{x})|^2,
\end{equation}
where $\kappa$ is a phenomenological parameter with dimensions chosen
so that $\rho_{\rm eff}$ has the units of mass density. The effective
mass contained in a localised phase texture is
\begin{equation}
  M_{\rm eff}
  = \frac{\kappa}{c^2}
    \int d^3x\,|\nabla\varphi(\mathbf{x})|^2.
\end{equation}
The Newtonian potential then satisfies the Poisson equation
\begin{equation}
  \nabla^2\Phi(\mathbf{x}) = 4\pi G \rho_{\rm eff}(\mathbf{x}),
\end{equation}
so that at large distances from a localised defect one recovers
\begin{equation}
  \Phi(r) \simeq - \frac{GM_{\rm eff}}{r}.
\end{equation}
This construction gives a simple way to relate phase textures to weak
gravitational wells.

%----------------------------------------------------------------------
\subsection{Phase clocks in the Newtonian limit}
%----------------------------------------------------------------------

Once the potential $\Phi(\mathbf{x})$ has been determined from
$\varphi(\mathbf{x})$ via the Poisson equation, the weak-field metric
component $g_{00}(\mathbf{x}) \simeq 1 + 2\Phi(\mathbf{x})/c^2$ follows.
Substituting this into the phase-clock relation
\begin{equation}
  \frac{d\phi}{dt}
  \simeq \omega_0\left(1 + \frac{\Phi(\mathbf{x})}{c^2}\right),
\end{equation}
we see that the same phase texture $\varphi(\mathbf{x})$ that creates
a gravitational potential well also slows down the rate of all
phase-based clocks in that region. In this sense the Newtonian
phase-field sector provides a useful ``ground floor'' for the
phase-fibre picture: slow particles, weak fields and static phase
textures can all be described consistently in terms of a single
scalar field $\varphi(\mathbf{x})$.

%======================================================================
\section{Illustrative scenarios for phase clocks}
\label{sec:scenarios}
%======================================================================

We now briefly discuss two simple scenarios where the phase-fibre
language makes standard results particularly transparent.

%----------------------------------------------------------------------
\subsection{Gravitational redshift of phase clocks at two heights}
%----------------------------------------------------------------------

Consider two identical phase clocks (for example, Josephson junctions)
located at heights $z_1$ and $z_2$ in a weak, static gravitational
field with potential $\Phi(z) \simeq gz$. Each clock advances along the
phase fibre at a universal rate $d\phi/d\tau = \omega_0$ with respect
to its proper time. The observable frequencies with respect to
coordinate time are
\begin{equation}
  f_1 = \frac{1}{2\pi}\frac{d\phi}{dt}\bigg|_{z_1},
  \qquad
  f_2 = \frac{1}{2\pi}\frac{d\phi}{dt}\bigg|_{z_2}.
\end{equation}
Using $d\tau = \sqrt{g_{00}(z)}\,dt$ with
$g_{00}(z) \simeq 1 + 2\Phi(z)/c^2$ we obtain
\begin{equation}
  \frac{\Delta f}{f}
  \equiv \frac{f_2 - f_1}{f_1}
  \simeq \frac{\Phi(z_2) - \Phi(z_1)}{c^2}
  = \frac{\Delta\Phi}{c^2},
  \label{eq:grav_redshift_clock}
\end{equation}
the standard expression for gravitational redshift.

\begin{figure}[t]
  \centering
  \includegraphics[width=0.7\textwidth]{5D_2.png}
  \caption{Phase clocks at two different gravitational potentials.
  Two identical clocks, located at heights $z_1$ and $z_2$ with
  $\Phi_1 = \Phi(z_1)$ and $\Phi_2 = \Phi(z_2)$, run at slightly
  different coordinate-time frequencies $f_1$ and $f_2$.
  In the phase-fibre language, imposing a fixed proper-time phase rate
  along the fibre leads directly to the standard gravitational redshift
  $\Delta f/f \simeq \Delta\Phi/c^2$, Eq.~\eqref{eq:grav_redshift_clock}.}
  \label{fig:phase_clocks_two_heights}
\end{figure}

The phase-fibre interpretation here is simple: the internal phase
angle $\phi$ advances uniformly per unit proper time, while gravity
changes the mapping between proper time and coordinate time at each
height. No new physics is introduced, but the geometric picture
helps to keep track of how $g_{00}$ enters the observable frequency.

%----------------------------------------------------------------------
\subsection{Rotating loops and Sagnac-like phases}
%----------------------------------------------------------------------

A second class of scenarios involves rotating loops and Sagnac-like
phase shifts. Consider a closed superconducting loop or SQUID ring
rotating with angular velocity $\boldsymbol{\Omega}$. The phase
accumulated along a closed path in the presence of rotation and
electromagnetic potentials can be written in terms of line integrals
of the phase covariant derivative. In the phase-fibre picture this
phase is simply the holonomy of the fibre connection around the loop,
with contributions from both $A_\mu$ and the effective gravitomagnetic
terms in the metric (in a fully relativistic treatment). In the
non-relativistic limit this reproduces the familiar Sagnac-like
expression for phase shifts proportional to $\boldsymbol{\Omega}\cdot\mathbf{A}$,
where $\mathbf{A}$ is the area vector enclosed by the loop. We do not
develop this in detail here; it is included only to illustrate that the
phase-fibre framework naturally accommodates both gravitational and
Aharonov--Bohm-type phases in a unified geometric language.

%======================================================================
\section{Discussion and outlook}
\label{sec:discussion}
%======================================================================

In this note we have presented a conservative five-dimensional
phase-fibre framework in which weak gravity, classical electromagnetism
and phase-based clocks are described in a single geometric picture with
phase as an explicit fifth coordinate. Starting from a standard
Kaluza--Klein ansatz with a compact U(1) phase fibre over four-dimensional
spacetime, we recalled how the Einstein--Maxwell action arises from the
five-dimensional Einstein--Hilbert action under dimensional reduction.
We then introduced a simple phenomenological rule for phase clocks ---
uniform advance along the fibre with respect to proper time --- and
showed how gravitational redshift follows immediately once the relation
between proper and coordinate time is taken into account.

We also summarised a Newtonian phase-field sector, developed in more
detail in Part~III of the PhaseGeometry series, in which a static real
phase field $\varphi(\mathbf{x})$ generates an effective mass density
proportional to $|\nabla\varphi|^2$ and thus a weak gravitational
potential via the Poisson equation. The same potential controls the
rate of phase clocks through $g_{00}$, so a given phase texture
$\varphi(\mathbf{x})$ simultaneously generates a gravitational well and
slows down all phase clocks placed within it. This provides a simple
intuition for how phase textures and weak gravity can be related within
the common language of the fifth coordinate.

To summarise why the phase-fibre construction is useful, even though it
does not modify any fundamental equations of motion:
(i) it promotes phase to a genuine fifth coordinate and thus provides a
unified geometric language for weak gravity, classical electromagnetism
and phase clocks, treating $g_{\mu\nu}$, $A_\mu$ and the internal phase
on the same footing;
(ii) it offers a natural bridge between weak-field general relativity
and analogue-gravity scenarios in superconductors and other coherent
media, including the Newtonian phase-field sector of Part~III; and
(iii) it gives a clean conceptual framework for analysing superconducting
and Josephson devices operating in weak gravitational or rotational
fields. In the present note we have focused on this intuitive 5D
picture; a separate technical companion, ``Phase-Fibre Geometry~II'',
develops the underlying action and field equations and works out explicit
phase-clock configurations in more detail.

\section*{Acknowledgements}

The author thanks colleagues and collaborators for discussions and
feedback on the PhaseGeometry notes. Any remaining errors or omissions
are the author's responsibility.

%======================================================================
\begin{thebibliography}{99}

\bibitem{Carroll}
S.~M.~Carroll,
\emph{Spacetime and Geometry: An Introduction to General Relativity},
Addison--Wesley (2004).

\bibitem{Wald}
R.~M.~Wald,
\emph{General Relativity},
University of Chicago Press (1984).

\bibitem{Kaluza}
T.~Kaluza,
``On the unification problem in physics,''
Sitzungsber.\ Preuss.\ Akad.\ Wiss.\ Berlin (Math.\ Phys.) (1921) 966.

\bibitem{Klein}
O.~Klein,
``Quantum theory and five-dimensional theory of relativity,''
Z.\ Phys.\ \textbf{37}, 895 (1926).

\bibitem{Jackson}
J.~D.~Jackson,
\emph{Classical Electrodynamics}, 3rd ed.,
Wiley (1998).

\bibitem{Tinkham}
M.~Tinkham,
\emph{Introduction to Superconductivity}, 2nd ed.,
McGraw--Hill (1996).

\bibitem{JosephsonMetrology}
C.~A.~Hamilton,
``Josephson voltage standards,''
Rev.\ Sci.\ Instrum.\ \textbf{71}, 3611--3623 (2000).

\bibitem{TurchanovPhaseFieldPartIII}
A.~Turchanov,
\emph{PhaseGeometry Phase-Fibre Foundations III: Newtonian phase-field sector},
Zenodo preprint (2025).

\bibitem{PhaseFibreCore}
A.~Turchanov,
\newblock \emph{PhaseGeometry Phase-Fibre Core Package v1.0},
\newblock Zenodo (2025),
DOI:\,\texttt{10.5281/zenodo.17783322}.

\end{thebibliography}

\end{document}

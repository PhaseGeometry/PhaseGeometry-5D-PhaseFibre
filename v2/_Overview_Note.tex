\documentclass[11pt,a4paper]{article}

\usepackage[utf8]{inputenc}
\usepackage[T1]{fontenc}
\usepackage[english]{babel}

\usepackage{amsmath,amssymb}
\usepackage{geometry}
\geometry{margin=1in}

\title{PhaseGeometry Phase-Fibre Programme:\\[4pt]
Registry and Overview Note}

\author{Aleksey Turchanov}
\date{December 2025}

\begin{document}

\maketitle

\begin{center}
\small
Licensed under Creative Commons Attribution 4.0 International (CC BY 4.0).\\[2pt]
Part of the PhaseGeometry Phase-Fibre programme; archived in the Core Package\\
DOI: \texttt{10.5281/zenodo.17783322}.
\end{center}

\vspace{1em}

\begin{abstract}
The PhaseGeometry Phase-Fibre programme develops a conservative five-dimensional
framework in which weak gravity, classical electromagnetism and phase-based clocks
are described in a single, geometrically unified language built on standard
Einstein--Maxwell theory. The key move is to treat phase $\varphi$ as a genuine
coordinate of state, so that the effective description of physical systems lives in
$4D + \varphi$ rather than in bare four-dimensional spacetime.

This short registry note plays three roles at once. First, it provides a human-level
map of the Phase-Fibre line: what the ``phase as fifth dimension'' move means, and
how it remains conservative at the level of classical GR+Maxwell theory. Second,
it fixes the naming and versioning conventions for the main Foundations notes and
related documents, as collected in the Phase-Fibre Core Package v1.0. Third, it
collects the core Zenodo information in a single citable place, so that readers and
referees can navigate the Phase-Fibre material as a layered programme rather than
as isolated preprints.
\end{abstract}

%======================================================================
\section{Phase as a fifth coordinate: organising idea}
%======================================================================

At the everyday level we already work with an implicit ``fifth coordinate''
all the time: the phase of a process. A wave on the shore, a heartbeat,
a circadian rhythm, a project timeline, a sleep stage --- in all these
cases we care not only about \emph{where} and \emph{when} something happens,
but also about which part of its cycle or trajectory the system is in.

In technical physics this appears in many guises:
\begin{itemize}
  \item the phase of a harmonic oscillator or wave,
        $h(x,t) = A\cos(kx - \omega t + \varphi_0)$;
  \item phase variables of superconducting condensates and Josephson junctions;
  \item phase-space coordinates $(x,p)$, configuration spaces $q_i$,
        order parameters and field configurations $\phi(x)$.
\end{itemize}

Conceptually, all of these answer a common question:
\begin{quote}
In which state or stage of its evolution / cycle is the system?
\end{quote}

The PhaseGeometry programme takes this seriously and treats phase
$\varphi$ as a bona fide coordinate of state, on the same footing as the
four coordinates of spacetime. At the most basic level the effective
description thus lives in $3D + t + \varphi$.

In the Phase-Fibre line this becomes explicitly geometrical: above each
spacetime point $x^\mu$ there is a compact $U(1)$ phase fibre with
coordinate $\varphi \in [0,2\pi)$, and the combined $4D + \varphi$
structure is encoded in a five-dimensional Kaluza--Klein-type metric.
The move is deliberately conservative:
\begin{itemize}
  \item the underlying gravitational and electromagnetic dynamics are
        those of standard Einstein--Maxwell theory;
  \item the ``extra dimension'' is compact and non-propagating: there
        are no new bulk modes beyond the usual 4D fields, but the phase
        fibre organises how gravity, electromagnetism and phase clocks
        appear together.
\end{itemize}

What changes is the language and bookkeeping, not the fundamental field
content of classical GR and Maxwell theory. At this stage the Phase-Fibre
line should be read as a geometric repackaging of known physics plus
simple Newtonian toy models, rather than as a source of new predictions.
More ambitious claims (fully relativistic solutions, quantum layers,
novel observables) are intentionally left for separate work.

%======================================================================
\section{Layered structure of the Phase-Fibre line}
%======================================================================

The Phase-Fibre part of the PhaseGeometry programme is organised into a
set of ``Foundations'' notes plus auxiliary conceptual and registry
documents. The aim is to keep the technical core compact while providing
entry points at different levels of detail.

The present Core Package v1.0 collects the main Foundations notes
(0--IV) and a small set of companion texts under a single Zenodo record
(\texttt{10.5281/zenodo.17783322}). Below we list the main layers as of
this version.

%----------------------------------------------------------------------
\subsection{Foundations layer (technical backbone)}
%----------------------------------------------------------------------

\begin{itemize}
  \item \textbf{Foundations 0: Technical Passport.}\\
        Compact technical summary of the Phase-Fibre framework. Fixes the
        ontology (spacetime base plus compact phase fibre), the basic
        Kaluza--Klein metric with a $U(1)$ fibre, and the minimal
        assumptions that keep the construction conservative. Provides a
        quick reference for notation and for the status of the framework
        relative to standard GR+EM.

  \item \textbf{Foundations I: Conceptual Framework.}\\
        Human-friendly conceptual picture of phase as a fifth coordinate:
        phase fibre over spacetime, weak gravity, phase-based clocks,
        qualitative links to superconductivity and Josephson effects.
        The emphasis is on intuitive cartoons and physical interpretation
        rather than on detailed derivations.

  \item \textbf{Foundations II: Field Equations and Phase-Clock Configurations.}\\
        Technical 5D/4D backbone. Starting from a five-dimensional metric
        with a compact $U(1)$ fibre, this note derives the 4D
        Einstein--Maxwell action by dimensional reduction. On top of
        this it introduces a minimal worldline action for a phase clock
        --- a timelike trajectory with an internal phase variable on the
        fibre --- and works out simple configurations (gravitational
        redshift between clocks at different heights, Sagnac-type phase
        shifts in rotating frames). All of this remains within classical
        GR and Maxwell theory.

  \item \textbf{Foundations III: Newtonian Halo Analogues from Phase Fields.}\\
        Newtonian phase-field sector. A single static scalar phase field
        $\phi(\mathbf{x})$ with effective energy density
        $\rho_{\rm eff} \propto |\nabla\phi|^2$ is used to generate
        halo-like gravitational potentials. Extended phase textures with
        an intermediate region where $d\phi/dr \propto 1/r$ lead to
        $\rho_{\rm eff}(r) \propto 1/r^2$, $M(r) \propto r$ and
        approximately flat circular-velocity curves $v_c(r) \simeq
        \mathrm{const}$. These are explicitly presented as classical,
        static toy models within Newtonian gravity, not as a complete
        theory of dark matter.

  \item \textbf{Foundations IV: Newtonian Black-Hole Analogues from Phase-Fibre Defects.}\\
        Phase-fibre bubble and black-hole analogues. A localised defect
        of the phase fibre --- a ``phase bubble'' --- is constructed so
        that its gradient energy generates a thin shell of effective
        mass. In the Newtonian limit the resulting potential reproduces
        a standard $1/r$ profile at large distances and approaches a
        black-hole-like regime when the bubble radius becomes comparable
        to its Schwarzschild radius. The exterior field is that of a
        point mass, while the interior is regular, so the construction
        serves as a transparent example of an exotic compact object
        built from phase structure in a fully classical setting.
\end{itemize}

%----------------------------------------------------------------------
\subsection{Phenomenology and applications}
%----------------------------------------------------------------------

Beyond the foundational layer, several notes develop phenomenology and
device-level applications, reusing the same $4D + \varphi$ bookkeeping
with emphasis on observable quantities (potentials, rotation curves,
clock rates) rather than on new fundamental fields. Examples include:
\begin{itemize}
  \item Newtonian phase-field phenomenology for cosmological backgrounds
        and the age of the Universe, built as an analogue layer to the
        PhaseGeometry Z$_2$ unified dark sector;
  \item Josephson and superconducting devices viewed as macroscopic
        phase clocks, including proposals for thick SNS weak links with
        local magnetic phase control and SQUID-type configurations,
        where the phase-fibre language serves as a compact bookkeeping
        tool for weak gravity and rotation effects.
\end{itemize}

These phenomenological texts are organised so that they can be read
either as stand-alone applications or as extensions of Foundations~II--IV.

%----------------------------------------------------------------------
\subsection{Conceptual and human-level notes}
%----------------------------------------------------------------------

In parallel, a set of conceptual notes and essays provide a bridge
between the technical Phase-Fibre material and broader physical or
philosophical questions:

\begin{itemize}
  \item \textbf{Phase as the Fifth Dimension: from 4D to $4D + \varphi$ without mysticism.}\\
        Popular-level essay explaining the idea of phase as a coordinate
        of state, using everyday examples (sea waves, day--night cycles,
        heartbeats, project phases). The message is that we effectively
        live in $3D + t + \varphi$ already; the Phase-Fibre framework
        simply makes this explicit in a geometric way. A condensed,
        updated version is archived in the same Core Package.

  \item \textbf{Foundations IV-B: Conceptual roadmap and phenomenological remarks.}\\
        Companion note to the technical Foundations~IV, summarising the
        black-hole analogue construction in non-technical language,
        clarifying what is and is not claimed, and positioning the work
        relative to other exotic compact object and analogue-gravity
        models.
\end{itemize}

These documents do not introduce new equations; they clarify language,
motivation and scope.

%======================================================================
\section{Role of this registry note}
%======================================================================

This short note is not meant to introduce new physics. Its rôle is
organisational and linguistic. Concretely, it serves three practical
purposes:

\begin{enumerate}
  \item \textbf{Human-level map.} To provide a concise, non-technical
        overview of the Phase-Fibre line: what the ``phase as fifth
        dimension'' move means, how it stays conservative at the level
        of Einstein--Maxwell theory, and how the different Foundations
        notes fit together conceptually.

  \item \textbf{Naming and versioning.} To fix stable names for the
        Foundations notes and related documents (technical vs.\ conceptual,
        Newtonian vs.\ 5D action), so that citations and references
        remain consistent across future work. The present text belongs
        to the Phase-Fibre Core Package v1.0, archived under a common
        Zenodo record.

  \item \textbf{Zenodo registry.} To record the Core Package DOI and
        the internal structure of the series in a citable way, allowing
        readers and referees to navigate the Phase-Fibre material as a
        unified but layered programme rather than as isolated preprints.
\end{enumerate}

In that sense this document is a ``registry and overview'' layer sitting
above the technical work: it explains the logic of the $4D + \varphi$
language and how the different pieces of the programme are meant to be
read and used.

%======================================================================
\section{How to cite the programme}
%======================================================================

For most purposes there are two levels of citation:
\begin{itemize}
  \item the Core Package record on Zenodo, which collects the Phase-Fibre
        Foundations notes and related material under a single umbrella;
  \item individual Foundations notes and companion texts, when a
        specific construction is used.
\end{itemize}

A generic reference to the programme can use the Core Package record:
\begin{quote}
A.~Turchanov, \emph{PhaseGeometry Phase-Fibre Core Package v1.0:\\
Foundations 0--IV and companion notes}, Zenodo (2025),\\
DOI: \texttt{10.5281/zenodo.17783322}.
\end{quote}

When a specific construction is used (e.g.\ Newtonian halo analogues or
black-hole analogues), the corresponding Foundations note should be
cited explicitly with its own title and version tag, as given in the
Core Package description. This registry note will be updated when new
core items are added to the series or when the package structure
changes.

%======================================================================
\section{Outlook}
%======================================================================

The Phase-Fibre line is deliberately modest in its assumptions: no new
fundamental bulk fields beyond Einstein--Maxwell, no changes to the
classical equations of motion, and --- at this stage --- no claims of
new observational signatures beyond standard weak-gravity and Newtonian
phenomenology. The novelty lies in treating phase as a genuine coordinate
of state and in using a compact fibre to organise how gravity,
electromagnetism and phase clocks interact in weak-field regimes.

From here, natural directions include:
\begin{itemize}
  \item refining the Newtonian phase-field sector and its cosmological
        and galactic phenomenology, with careful comparison to existing
        scalar-halo and exotic-compact-object literature;
  \item exploring precision phase-clock experiments (Josephson, optical,
        atomic) in weak gravitational and rotating backgrounds within
        the Phase-Fibre language;
  \item developing further black-hole and horizon analogues tied to
        phase structures in coherent media, and connecting them to the
        broader PhaseGeometry Z$_2$ programme where decoherence and
        branching are treated explicitly;
  \item formulating fully relativistic Phase-Fibre solutions and, in
        separate work, adding genuinely quantum layers on top of the
        classical geometry.
\end{itemize}

The registry note you are reading is intended as a stable entry point:
a compact ``map and index'' for the Phase-Fibre programme as it grows,
and a conservative baseline against which more ambitious extensions can
be judged.

\end{document}

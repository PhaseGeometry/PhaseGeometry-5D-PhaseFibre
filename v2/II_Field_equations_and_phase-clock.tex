\documentclass[11pt,a4paper]{article}

\usepackage[utf8]{inputenc}
\usepackage[T1]{fontenc}
\usepackage[english]{babel}

\usepackage{amsmath,amssymb}
\usepackage{geometry}
\usepackage{hyperref}
\geometry{margin=1in}

\numberwithin{equation}{section}

\title{PhaseGeometry Phase-Fibre Foundations II:\\[4pt]
Action, field equations and phase-clock configurations}

\author{Aleksey Turchanov}
\date{November 2025}

\begin{document}

\maketitle

\begin{center}
\small
Licensed under Creative Commons Attribution 4.0 International (CC BY 4.0).\\[2pt]
Part of the PhaseGeometry Phase-Fibre programme; archived in the Core Package\\
DOI: \texttt{10.5281/zenodo.17783322}.
\end{center}

\vspace{1em}



\begin{abstract}
We formulate a minimal phase-fibre framework in which classical gravity, electromagnetism and phase-based clocks are described within a single geometric setting built on standard general relativity and Maxwell theory. The construction starts from a five-dimensional Kaluza--Klein metric with a compact U(1) fibre coordinate $\phi$, so that phase is promoted to a \emph{genuine fifth coordinate} on the same mathematical footing as space and time. The five-dimensional Einstein--Hilbert action reduces to the usual four-dimensional Einstein--Maxwell action when the fields are independent of the fibre coordinate, and no new propagating degrees of freedom are introduced.

On top of this background we introduce a simple worldline action for a phase clock: a timelike test trajectory $x^\mu(\tau)$ with an internal phase variable $\phi(\tau)$ living on the fibre. The covariant phase rate $D\phi/D\tau = d\phi/d\tau + kA_\mu u^\mu$ is constrained to a constant reference frequency $\omega_0$ in proper time. This provides a compact description of phase clocks such as Josephson oscillators in a given background metric $g_{\mu\nu}$ and gauge field $A_\mu$.

We work out explicitly two benchmark configurations. In a static weak gravitational field with two identical clocks at different heights, we recover the standard gravitational redshift $\Delta f/f \simeq \Delta\Phi/c^2$ directly from the phase-clock condition and the relation between proper and coordinate time. In a simple rotating-ring configuration, we show how a Sagnac-type phase shift $\Delta\phi \sim (\omega_0/c^2)\,\Omega A$ arises for a phase signal propagating around the loop, where $\Omega$ is the angular velocity and $A$ the area of the ring. These examples illustrate that the phase-fibre picture is mathematically well-posed and reproduces standard weak-gravity clock physics in a compact geometric form while remaining entirely within classical four-dimensional GR and Maxwell theory.
\end{abstract}

\newpage
\tableofcontents

\vspace{1em}
%======================================================================
\section{Introduction and scope}
%======================================================================

This paper is the second part of the PhaseGeometry Phase-Fibre programme.
It is intended as a \emph{technical companion} to the conceptual note
Ref.~\cite{PhaseFibreI}, where the main ideas of the phase-fibre picture
were presented in a deliberately lightweight and intuitive form: a compact
phase fibre attached to each point of four-dimensional spacetime, a
Kaluza--Klein metric that encodes both gravity and electromagnetism, and
an interpretation of macroscopic phase-based clocks as worldlines that
advance uniformly along the fibre.

Within the Phase-Fibre line of the PhaseGeometry programme this paper
sits just above the core ``passport'' level. The accompanying
\emph{Foundations 0: Technical Passport} fixes the ontology and notation
of the five-dimensional phase-fibre framework in a compact,
equation-driven form, while \emph{Foundations I: Conceptual Framework}
emphasises the physical picture, intuitive cartoons and links to
superconducting / Josephson systems. The present Part~II uses the same
conventions as Foundations~0 and~I, but focuses on the explicit action,
field equations and worked examples for phase clocks in given
backgrounds.

In the present work we keep strictly to classical field theory and adopt
a conservative technical goal. We \emph{do not} modify general relativity
or Maxwell theory, nor do we introduce additional dynamical fields
beyond those already present in the four-dimensional Einstein--Maxwell
system. Instead, we take the phase-fibre ansatz seriously at the level
of the action and field equations, and we show in detail that it
reproduces standard weak-gravity clock physics in a compact geometric
language.

The key ingredients can be summarised as follows. Four-dimensional
spacetime $(\mathcal{M}_4,g_{\mu\nu})$ is taken as in standard general
relativity. Above each spacetime point hangs a compact U(1) fibre with
coordinate $\phi\in[0,2\pi)$, representing an internal phase. On the
resulting five-dimensional bundle $\mathcal{M}_5$ one introduces a
Kaluza--Klein metric
\begin{equation}
  ds_5^2
  = g_{\mu\nu}(x)\,dx^\mu dx^\nu
    + R^2\bigl(d\phi + kA_\mu(x)\,dx^\mu\bigr)^2,
\end{equation}
so that the electromagnetic potential $A_\mu$ appears as the connection
on the phase fibre, and the metric component $g_{00}$ controls the
relative rate between proper and coordinate time. Phase-based clocks
(superconducting condensates, Josephson oscillators, and other coherent
systems) are then described as objects whose internal phase winds
uniformly along the fibre with respect to proper time.

In the Newtonian limit, a separate phase-field note
(PhaseGeometry Series Part~III, Ref.~\cite{PhaseFieldIII})
showed that a static scalar field $\varphi(\mathbf{x})$ with gradient
energy density $\rho_{\rm eff}\propto|\nabla\varphi|^2$ can source a
Newtonian potential $\Phi(\mathbf{x})$ via the usual Poisson equation,
and that this potential immediately feeds into the gravitational
redshift of phase clocks through $g_{00}\simeq 1+2\Phi/c^2$.
In that context the emphasis was on a phenomenological low-energy
``phase floor'' for weak gravity.

The emphasis in the present paper is different. Here we aim to:
\begin{itemize}
  \item formulate the phase-fibre ansatz at the level of an explicit
        five-dimensional action and derive the corresponding
        four-dimensional Einstein--Maxwell theory by dimensional
        reduction;
  \item introduce a minimal worldline action for a phase clock --- a
        timelike trajectory with an internal phase degree of freedom
        on the fibre --- and obtain its phase evolution in a given
        background $(g_{\mu\nu},A_\mu)$;
  \item work out two simple clock configurations where the standard
        weak-gravity results (gravitational redshift and Sagnac-type
        phase shifts) appear naturally in the phase-fibre description.
\end{itemize}

Throughout we impose three conservative constraints:
\begin{enumerate}
  \item We work entirely within standard four-dimensional general
        relativity and classical Maxwell theory. The five-dimensional
        phase-fibre metric is used as a geometric parametrisation that
        reproduces the Einstein--Maxwell action; no extra five-dimensional
        modes or non-standard couplings are introduced.
  \item The phase clocks are treated as test systems on a given
        background. Their contribution to the stress--energy tensor
        and current is assumed negligible, so that the background
        geometry and electromagnetic field obey the usual
        Einstein--Maxwell equations with other matter sources.
  \item All results are derived in regimes where standard approximations
        are valid: weak gravitational fields for the redshift example,
        and small angular velocities for the rotating-ring configuration.
\end{enumerate}

The structure of the paper is as follows.
In Sec.~\ref{sec:phase_fibre_action} we introduce the five-dimensional
phase-fibre metric and derive the four-dimensional Einstein--Maxwell
action and field equations by dimensional reduction. In
Sec.~\ref{sec:phase_clocks_worldlines} we define a simple worldline
action for a phase clock and derive the corresponding phase evolution
equation in a given background. Section~\ref{sec:configA} applies this
framework to a static weak-field configuration with two identical clocks
at different heights and reproduces the standard gravitational redshift
formula. In Sec.~\ref{sec:configB} we consider a simple rotating-ring
setup and show how a Sagnac-type phase shift arises for a phase signal
propagating around the loop. We conclude in Sec.~\ref{sec:discussion}
with a brief outlook, including the advantages and potential
applications of the phase-fibre language and its connection to more
detailed device-level configurations discussed elsewhere.

%======================================================================
\section{Phase-fibre framework and effective action}
\label{sec:phase_fibre_action}
%======================================================================

In this section we formulate the phase-fibre ansatz at the level of the
action and derive the corresponding four-dimensional Einstein--Maxwell
theory by dimensional reduction. We work purely in classical general
relativity and classical electromagnetism.

%----------------------------------------------------------------------
\subsection{Five-dimensional phase-fibre metric}
%----------------------------------------------------------------------

We consider a five-dimensional manifold $\mathcal{M}_5$ which is a U(1)
fibre bundle over four-dimensional spacetime $\mathcal{M}_4$. Local
coordinates on $\mathcal{M}_5$ are written as
\begin{equation}
  X^A = (x^\mu,\phi), \qquad \mu = 0,1,2,3,
\end{equation}
where $\phi$ is a periodic coordinate on the internal circle $S^1$.

The phase-fibre metric is taken in the Kaluza--Klein form
\begin{equation}
  ds_5^2
  = G_{AB} dX^A dX^B
  = g_{\mu\nu}(x)\,dx^\mu dx^\nu
    + R^2\bigl(d\phi + kA_\mu(x)\,dx^\mu\bigr)^2.
  \label{eq:5d_metric_PF}
\end{equation}
Here $g_{\mu\nu}(x)$ is the four-dimensional spacetime metric and
$A_\mu(x)$ is a four-dimensional U(1) gauge field. The constants $R$
and $k$ set the scale of the fibre radius and the coupling of the fibre
coordinate to $A_\mu$; their combination will be fixed below by matching
to the standard electromagnetic action.

The one-form
\begin{equation}
  \mathcal{A} \equiv d\phi + k A_\mu(x)\,dx^\mu
\end{equation}
plays the role of a U(1) connection on the fibre. Under a local shift
of the fibre coordinate
\begin{equation}
  \phi(x) \to \phi(x) + \chi(x)
\end{equation}
the metric~\eqref{eq:5d_metric_PF} remains invariant provided
\begin{equation}
  A_\mu(x) \to A_\mu(x) - \frac{1}{k}\,\partial_\mu\chi(x).
\end{equation}
Thus the usual U(1) gauge symmetry is implemented geometrically as a
coordinate transformation along the phase fibre.

We impose the standard Kaluza--Klein assumption that $g_{\mu\nu}$ and
$A_\mu$ are independent of the fibre coordinate $\phi$, so that all
fields depend only on $x^\mu$.

%----------------------------------------------------------------------
\subsection{Five-dimensional Einstein--Hilbert action}
%----------------------------------------------------------------------

The purely gravitational sector in five dimensions is described by the
Einstein--Hilbert action
\begin{equation}
  S_5
  = \frac{1}{16\pi G_5}
    \int_{\mathcal{M}_5} d^5X\,\sqrt{-G}\,R_5,
  \label{eq:5D_EH_action}
\end{equation}
where $G \equiv \det G_{AB}$ and $R_5$ is the Ricci scalar constructed
from the metric~\eqref{eq:5d_metric_PF}.

For the metric ansatz~\eqref{eq:5d_metric_PF} one finds the standard
Kaluza--Klein decomposition
\begin{equation}
  R_5 = R_4 - \frac{1}{4}(kR)^2 F_{\mu\nu}F^{\mu\nu},
  \qquad
  \sqrt{-G} = R\sqrt{-g},
  \label{eq:R5_decomposition}
\end{equation}
where $R_4$ is the Ricci scalar of the four-dimensional metric
$g_{\mu\nu}$, $g \equiv \det g_{\mu\nu}$, and
\begin{equation}
  F_{\mu\nu} = \partial_\mu A_\nu - \partial_\nu A_\mu
\end{equation}
is the Maxwell field strength.

Substituting~\eqref{eq:R5_decomposition} into~\eqref{eq:5D_EH_action}
and integrating over the circle of circumference $2\pi$ yields
\begin{equation}
  S_5
  = \frac{2\pi R}{16\pi G_5}
    \int d^4x\,\sqrt{-g}\,
    \left[
      R_4 - \frac{1}{4}(kR)^2 F_{\mu\nu}F^{\mu\nu}
    \right].
  \label{eq:5D_action_reduced}
\end{equation}

%----------------------------------------------------------------------
\subsection{Dimensional reduction to Einstein--Maxwell}
%----------------------------------------------------------------------

We define the four-dimensional Newton constant by
\begin{equation}
  \frac{1}{16\pi G_4}
  \equiv
  \frac{2\pi R}{16\pi G_5},
  \label{eq:G4_definition}
\end{equation}
so that the Einstein--Hilbert term becomes
\begin{equation}
  S_{\text{EH,4D}}
  = \frac{1}{16\pi G_4}
    \int d^4x\,\sqrt{-g}\,R_4.
\end{equation}
To bring the Maxwell term into the standard form
\begin{equation}
  S_{\text{EM}}
  = -\frac{1}{4\mu_0}\int d^4x\,\sqrt{-g}\,F_{\mu\nu}F^{\mu\nu},
\end{equation}
we require
\begin{equation}
  \frac{2\pi R}{16\pi G_5}\,\frac{1}{4}(kR)^2
  = \frac{1}{4\mu_0}.
\end{equation}
Using~\eqref{eq:G4_definition} this can be written as
\begin{equation}
  \frac{1}{16\pi G_4}\,(kR)^2
  = \frac{1}{\mu_0},
\end{equation}
or equivalently
\begin{equation}
  (kR)^2 = \frac{16\pi G_4}{\mu_0}.
  \label{eq:kR_matching}
\end{equation}
In practice, $G_4$ and $\mu_0$ are fixed by experiment, and
Eq.~\eqref{eq:kR_matching} constrains only the combination $kR$. We do
not attempt to determine $k$ and $R$ separately.

With these identifications the effective four-dimensional gravitational
and electromagnetic action becomes
\begin{equation}
  S_{\text{grav+EM}}
  = \frac{1}{16\pi G_4}\int d^4x\,\sqrt{-g}\,R_4
    - \frac{1}{4\mu_0}\int d^4x\,\sqrt{-g}\,F_{\mu\nu}F^{\mu\nu}.
  \label{eq:Einstein_Maxwell_action}
\end{equation}
Thus the phase-fibre metric~\eqref{eq:5d_metric_PF} reproduces the
standard Einstein--Maxwell theory in four dimensions. The fibre
construction is an equivalent geometric formulation, without changing
the classical field equations for $g_{\mu\nu}$ and $A_\mu$.

%----------------------------------------------------------------------
\subsection{Field equations and sources}
%----------------------------------------------------------------------

Varying~\eqref{eq:Einstein_Maxwell_action} with respect to the metric
$g_{\mu\nu}$ gives
\begin{equation}
  \delta S_{\text{grav+EM}}
  = \frac{1}{16\pi G_4}
    \int d^4x\,\sqrt{-g}\,
      (G_{\mu\nu} - 8\pi G_4 T^{\text{EM}}_{\mu\nu})
      \,\delta g^{\mu\nu},
\end{equation}
where
\begin{equation}
  G_{\mu\nu} = R_{\mu\nu} - \frac{1}{2}g_{\mu\nu}R_4
\end{equation}
is the Einstein tensor and $T^{\text{EM}}_{\mu\nu}$ is the electromagnetic
stress--energy tensor
\begin{equation}
  T^{\text{EM}}_{\mu\nu}
  = \frac{1}{\mu_0}
    \left(
      F_{\mu\alpha}F_\nu{}^\alpha
      - \frac{1}{4}g_{\mu\nu}F_{\alpha\beta}F^{\alpha\beta}
    \right).
\end{equation}
In the presence of additional matter fields with stress--energy tensor
$T^{\text{mat}}_{\mu\nu}$ the Einstein equations are
\begin{equation}
  G_{\mu\nu}
  = 8\pi G_4
    \left(
      T^{\text{EM}}_{\mu\nu} + T^{\text{mat}}_{\mu\nu}
    \right).
  \label{eq:Einstein_eqs}
\end{equation}

Variation with respect to $A_\mu$ yields the inhomogeneous Maxwell
equations. Writing the matter contribution to the total action as
$S_{\text{mat}}[g_{\mu\nu},A_\mu,\dots]$, we define the electric
four-current by
\begin{equation}
  J^\mu(x)
  \equiv \frac{1}{\sqrt{-g}}
  \frac{\delta S_{\text{mat}}}{\delta A_\mu(x)}.
  \label{eq:J_def}
\end{equation}
Then variation of~\eqref{eq:Einstein_Maxwell_action} plus $S_{\text{mat}}$
gives
\begin{equation}
  \nabla_\mu F^{\mu\nu} = \mu_0 J^\nu,
  \label{eq:Maxwell_eqs}
\end{equation}
with $\nabla_\mu$ the covariant derivative compatible with $g_{\mu\nu}$.

Equations~\eqref{eq:Einstein_eqs} and~\eqref{eq:Maxwell_eqs} are the
standard Einstein--Maxwell field equations. In what follows we treat
phase clocks as test systems: their contribution to $T^{\text{mat}}_{\mu\nu}$
and $J^\mu$ is assumed small enough that the background metric and
electromagnetic field can be taken as given solutions of
Eqs.~\eqref{eq:Einstein_eqs}--\eqref{eq:Maxwell_eqs}.

%======================================================================
\section{Phase clocks as test worldlines}
\label{sec:phase_clocks_worldlines}
%======================================================================

We now introduce a minimal model for a phase-based clock as a test
worldline with an internal phase coordinate on the fibre. The goal is
to obtain a clean evolution equation for the phase in a given background
$(g_{\mu\nu},A_\mu)$ and to connect it to the phase-fibre picture.

%----------------------------------------------------------------------
\subsection{Worldline and proper time}
%----------------------------------------------------------------------

A point-like clock follows a timelike worldline $x^\mu(\lambda)$ in
spacetime. For a given metric $g_{\mu\nu}$ the line element is
\begin{equation}
  ds^2 = g_{\mu\nu}\,dx^\mu dx^\nu.
\end{equation}
We restrict attention to timelike worldlines with $ds^2 > 0$ and define
the proper time $\tau$ along the worldline by
\begin{equation}
  d\tau = \frac{1}{c}\sqrt{ds^2}
        = \frac{1}{c}\sqrt{g_{\mu\nu}u^\mu u^\nu}\,d\lambda,
\end{equation}
where
\begin{equation}
  u^\mu \equiv \frac{dx^\mu}{d\lambda}
\end{equation}
is the tangent vector. We fix the parametrisation by choosing
$\lambda = \tau$, so that
\begin{equation}
  u^\mu \equiv \frac{dx^\mu}{d\tau},
  \qquad
  g_{\mu\nu}u^\mu u^\nu = c^2.
  \label{eq:normalisation_u}
\end{equation}

%----------------------------------------------------------------------
\subsection{Phase variable and clock action}
%----------------------------------------------------------------------

We assign to the clock an internal phase variable $\phi(\tau)$ which
lives on the same U(1) fibre as the geometric coordinate in
Eq.~\eqref{eq:5d_metric_PF}. Along the worldline the natural
gauge-covariant phase rate is
\begin{equation}
  \frac{D\phi}{D\tau}
  \equiv \frac{d\phi}{d\tau} + k A_\mu u^\mu,
  \label{eq:covariant_phase_rate}
\end{equation}
which is the pullback of the fibre covariant derivative
$D_\mu\phi = \partial_\mu\phi + kA_\mu$.

We model the clock by the following worldline action:
\begin{equation}
  S_{\text{clock}}
  = -m c^2 \int d\tau
    -\frac{I}{2}\int d\tau\,
      \biggl(
        \frac{D\phi}{D\tau} - \omega_0
      \biggr)^2.
  \label{eq:S_clock_def}
\end{equation}
Here:
\begin{itemize}
  \item $m$ is the rest mass (assumed small enough that the clock can
        be treated as a test body);
  \item $I$ is an effective ``moment of inertia'' in phase space that
        controls how strongly the clock is constrained to run at the
        reference angular frequency $\omega_0$;
  \item $\omega_0$ is a constant frequency parameter, representing the
        intrinsic phase rate of the clock in proper time.
\end{itemize}

The first term in~\eqref{eq:S_clock_def} is the usual proper-time
action of a free point particle. The second term penalises deviations
of the covariant phase rate $D\phi/D\tau$ from $\omega_0$. In the
formal limit $I\to\infty$ the phase constraint is enforced exactly,
\begin{equation}
  \frac{D\phi}{D\tau} = \omega_0,
  \label{eq:clock_constraint}
\end{equation}
and the clock runs at a fixed phase rate along its worldline.

Since we work in the test-clock limit, we neglect the contribution of
$S_{\text{clock}}$ to the background field equations and treat
$g_{\mu\nu}$ and $A_\mu$ as prescribed.

%----------------------------------------------------------------------
\subsection{Variation with respect to the phase}
%----------------------------------------------------------------------

Varying the action~\eqref{eq:S_clock_def} with respect to $\phi(\tau)$,
keeping $x^\mu(\tau)$ fixed, gives
\begin{equation}
  \delta S_{\text{clock}}
  = -I\int d\tau\,
    \biggl(
      \frac{D\phi}{D\tau} - \omega_0
    \biggr)
    \delta\!\left(
      \frac{D\phi}{D\tau}
    \right).
\end{equation}
Since
\begin{equation}
  \delta\!\left(\frac{D\phi}{D\tau}\right)
  = \frac{d}{d\tau}\delta\phi,
\end{equation}
integrating by parts (and assuming fixed endpoints $\delta\phi=0$) we
obtain
\begin{equation}
  \delta S_{\text{clock}}
  = I\int d\tau\,
    \frac{d}{d\tau}
    \biggl(
      \frac{D\phi}{D\tau} - \omega_0
    \biggr)\delta\phi.
\end{equation}
Requiring $\delta S_{\text{clock}}=0$ for arbitrary $\delta\phi$ yields
the phase evolution equation
\begin{equation}
  \frac{d}{d\tau}
  \biggl(
    \frac{D\phi}{D\tau} - \omega_0
  \biggr) = 0.
\end{equation}
The general solution is
\begin{equation}
  \frac{D\phi}{D\tau}
  = \omega_0 + C,
\end{equation}
where $C$ is a constant along the worldline. Choosing initial
conditions such that $D\phi/D\tau = \omega_0$ at some reference
proper time fixes $C=0$, so that
\begin{equation}
  \frac{D\phi}{D\tau}
  = \omega_0,
  \label{eq:clock_condition_final}
\end{equation}
which is the phase-clock condition used in the concept note.

Equivalently, in terms of $d\phi/d\tau$,
Eq.~\eqref{eq:covariant_phase_rate} and
\eqref{eq:clock_condition_final} imply
\begin{equation}
  \frac{d\phi}{d\tau}
  = \omega_0 - kA_\mu u^\mu.
  \label{eq:dphi_dtau_explicit}
\end{equation}
In the absence of electromagnetic fields (or in a gauge where
$A_\mu u^\mu=0$), the phase increases at a constant rate
$d\phi/d\tau=\omega_0$ along the worldline.

%----------------------------------------------------------------------
\subsection{Worldline trajectory and test limit}
%----------------------------------------------------------------------

Variation of~\eqref{eq:S_clock_def} with respect to $x^\mu(\tau)$
generally yields a modified equation of motion containing the usual
geodesic term, a Lorentz-force term proportional to $F_{\mu\nu}$, and
additional contributions from the phase sector proportional to $I$.
In the strict test-clock limit we assume that:
\begin{itemize}
  \item the clock mass $m$ is sufficiently small not to perturb the
        background metric;
  \item the phase inertia $I$ is small enough that the phase constraint
        does not significantly affect the trajectory $x^\mu(\tau)$.
\end{itemize}
Under these assumptions the clock follows a prescribed timelike
trajectory in the given background $(g_{\mu\nu},A_\mu)$, and we use
Eq.~\eqref{eq:dphi_dtau_explicit} to compute its phase evolution.

In the applications below we specialise further to static metrics and
worldlines of clocks at rest in the chosen coordinates. In that case
$u^i=0$ and $u^0 = dt/d\tau$ is determined by $g_{00}$, so
Eq.~\eqref{eq:dphi_dtau_explicit} reduces to the simple phase-clock
relation
\begin{equation}
  \frac{d\phi}{d\tau} = \omega_0,
\end{equation}
and the gravitational redshift enters only through the relation
between $\tau$ and the coordinate time $t$.

%======================================================================
\section{Configuration A: static weak-field phase clock at two heights}
\label{sec:configA}
%======================================================================

We now apply the phase-clock framework to a simple configuration:
two identical clocks located at different heights in a static weak
gravitational field. The goal is to derive the standard gravitational
redshift
\begin{equation}
  \frac{\Delta f}{f}
  \simeq \frac{\Delta\Phi}{c^2},
  \label{eq:grav_redshift_standard}
\end{equation}
with $\Phi$ the Newtonian potential, directly from the phase-clock
condition~\eqref{eq:clock_condition_final} and the relation between
proper and coordinate time in the given metric.

%----------------------------------------------------------------------
\subsection{Weak-field metric and proper time}
%----------------------------------------------------------------------

In the Newtonian weak-field regime the metric can be written as
\begin{equation}
  ds^2
  = g_{00}(\mathbf{x})\,c^2 dt^2
    - \delta_{ij}\,dx^i dx^j,
  \qquad
  g_{00}(\mathbf{x}) \simeq 1 + \frac{2\Phi(\mathbf{x})}{c^2},
  \label{eq:weak_metric_again}
\end{equation}
where $\Phi(\mathbf{x})$ is the Newtonian gravitational potential and
$|\Phi|/c^2 \ll 1$.

Consider a clock at rest at a fixed spatial position $\mathbf{x}$ in
these coordinates. Along its worldline $dx^i=0$ and the line element
reduces to
\begin{equation}
  ds^2 = g_{00}(\mathbf{x})\,c^2 dt^2.
\end{equation}
The proper time increment is therefore
\begin{equation}
  d\tau
  = \frac{1}{c}\sqrt{ds^2}
  = \sqrt{g_{00}(\mathbf{x})}\,dt
  \simeq \left(1 + \frac{\Phi(\mathbf{x})}{c^2}\right) dt,
  \label{eq:tau_vs_t_weak}
\end{equation}
where we expanded $\sqrt{1+\epsilon} \simeq 1 + \epsilon/2$ with
$\epsilon = 2\Phi/c^2$.

%----------------------------------------------------------------------
\subsection{Phase evolution and observed frequency}
%----------------------------------------------------------------------

For a phase clock the covariant phase rate along the worldline obeys
Eq.~\eqref{eq:clock_condition_final},
\begin{equation}
  \frac{D\phi}{D\tau} = \omega_0.
\end{equation}
For a static configuration with $u^i=0$ and $A_\mu u^\mu = 0$ (or in a
gauge where this holds) the covariant derivative reduces to an ordinary
proper-time derivative,
\begin{equation}
  \frac{d\phi}{d\tau} = \omega_0.
\end{equation}
Using~\eqref{eq:tau_vs_t_weak}, the phase rate with respect to
coordinate time $t$ is
\begin{equation}
  \frac{d\phi}{dt}
  = \frac{d\phi}{d\tau}\,\frac{d\tau}{dt}
  = \omega_0 \sqrt{g_{00}(\mathbf{x})}
  \simeq \omega_0\left(1 + \frac{\Phi(\mathbf{x})}{c^2}\right).
  \label{eq:dphi_dt_coord}
\end{equation}
We define the coordinate-time frequency of the clock as
\begin{equation}
  f(\mathbf{x})
  \equiv \frac{1}{2\pi}\frac{d\phi}{dt}
  = \frac{\omega_0}{2\pi}\sqrt{g_{00}(\mathbf{x})}.
  \label{eq:f_coord_def_configA}
\end{equation}

%----------------------------------------------------------------------
\subsection{Frequency ratio between two heights}
%----------------------------------------------------------------------

Now consider two identical clocks at positions $\mathbf{x}_1$ and
$\mathbf{x}_2$, with corresponding potentials
\(\Phi_1 = \Phi(\mathbf{x}_1)\) and \(\Phi_2 = \Phi(\mathbf{x}_2)\).
Their frequencies are
\begin{equation}
  f_1
  = \frac{\omega_0}{2\pi}\sqrt{g_{00}(\mathbf{x}_1)},
  \qquad
  f_2
  = \frac{\omega_0}{2\pi}\sqrt{g_{00}(\mathbf{x}_2)}.
\end{equation}
The ratio is
\begin{equation}
  \frac{f_2}{f_1}
  = \sqrt{\frac{g_{00}(\mathbf{x}_2)}{g_{00}(\mathbf{x}_1)}}.
\end{equation}
Using~\eqref{eq:weak_metric_again} and expanding to first order,
\begin{equation}
  g_{00}(\mathbf{x}_a)
  \simeq 1 + \frac{2\Phi_a}{c^2},
  \qquad a=1,2,
\end{equation}
so that
\begin{equation}
  \sqrt{g_{00}(\mathbf{x}_a)}
  \simeq 1 + \frac{\Phi_a}{c^2}.
\end{equation}
Hence
\begin{equation}
  \frac{f_2}{f_1}
  \simeq
  \frac{1 + \Phi_2/c^2}{1 + \Phi_1/c^2}
  \simeq 1 + \frac{\Phi_2 - \Phi_1}{c^2},
\end{equation}
and the relative frequency shift is
\begin{equation}
  \frac{\Delta f}{f}
  \equiv \frac{f_2 - f_1}{f_1}
  \simeq \frac{\Phi_2 - \Phi_1}{c^2}
  = \frac{\Delta\Phi}{c^2},
\end{equation}
which is Eq.~\eqref{eq:grav_redshift_standard}. For an approximately
uniform field $\Phi(z)\simeq gz$ this reduces to the familiar form
\begin{equation}
  \frac{\Delta f}{f}
  \simeq \frac{g(z_2 - z_1)}{c^2}.
\end{equation}

The key point is that the internal clock parameter $\omega_0$ cancels
in the ratio: the gravitational field affects only the relation between
proper and coordinate time via $g_{00}$, while the intrinsic phase rate
in proper time remains fixed.

%----------------------------------------------------------------------
\subsection{Relation to Josephson phase clocks}
%----------------------------------------------------------------------

In superconducting systems the macroscopic phase of the condensate plays
the role of $\phi$, and the combination
\begin{equation}
  \partial_\mu\theta - qA_\mu
\end{equation}
appearing in London and Ginzburg--Landau theory can be identified with
the fibre covariant derivative $D_\mu\phi$ for suitable units. In a
Josephson junction the ac effect realises a phase clock with
\begin{equation}
  \frac{d\phi}{dt_{\text{proper}}}
  = \frac{2e}{\hbar}V,
\end{equation}
so that in flat spacetime the clock frequency is $f_J = (2e/h)V$.
Within the phase-fibre picture, such a device is simply a particular
realisation of the abstract phase clock considered above.

As long as the junction can be treated as a coherent phase oscillator
and the weak-field approximation holds, the gravitational redshift
derived in this section applies directly: for two identical Josephson
clocks operating at the same local bias but located at different
potentials $\Phi_1$ and $\Phi_2$,
\begin{equation}
  \frac{\Delta f_J}{f_J}
  \simeq \frac{\Delta\Phi}{c^2}.
\end{equation}
The phase-fibre formalism does not change this prediction; it provides
a compact geometric way to see how gravity (via $g_{00}$) and
electromagnetism (via $A_\mu$) enter jointly into the phase evolution
of such clocks.

%======================================================================
\section{Configuration B: rotating phase-clock loop (Sagnac-type)}
\label{sec:configB}
%======================================================================

As a second benchmark we consider a simple rotating-ring configuration
and show how a Sagnac-type phase shift arises in the phase-clock
language. For definiteness we work with a circular loop of radius $R$
in the $z=0$ plane, rotating slowly with angular velocity $\Omega$
about the $z$-axis.

%----------------------------------------------------------------------
\subsection{Metric in a rotating frame}
%----------------------------------------------------------------------

We start from flat spacetime in cylindrical coordinates
$(t_{\rm lab}, r, \varphi_{\rm lab}, z)$:
\begin{equation}
  ds^2 = c^2 dt_{\rm lab}^2 - dr^2 - r^2 d\varphi_{\rm lab}^2 - dz^2.
\end{equation}
We now move to a frame rotating with angular velocity $\Omega$ about
the $z$-axis. Let $(t,\varphi)$ denote the rotating coordinates and
$(t_{\rm lab},\varphi_{\rm lab})$ the inertial lab coordinates. We define
\begin{equation}
  t_{\rm lab} = t, 
  \qquad
  \varphi_{\rm lab} = \varphi + \Omega t.
\end{equation}
Differentiating, we obtain
\begin{equation}
  d\varphi_{\rm lab} = d\varphi + \Omega\,dt.
\end{equation}
Substituting into the line element and expressing everything in terms
of $(t,r,\varphi,z)$, we find
\begin{equation}
  ds^2
  = (c^2 - \Omega^2 r^2)\,dt^2
    + 2\Omega r^2\,d\varphi\,dt
    - dr^2 - r^2 d\varphi^2 - dz^2.
  \label{eq:rotating_metric}
\end{equation}

%----------------------------------------------------------------------
\subsection{Null phase signal around the loop}
%----------------------------------------------------------------------

To extract a Sagnac-type phase shift we consider a phase signal
propagating around the ring at $r=R$, $z=0$ in both directions. For
simplicity we model the signal as moving at the speed of light along the
ring, so that its worldline segments satisfy $ds^2 = 0$ with $dr=dz=0$.
From~\eqref{eq:rotating_metric} this gives
\begin{equation}
  (c^2 - \Omega^2 R^2) dt^2
  + 2\Omega R^2 d\varphi\,dt
  - R^2 d\varphi^2 = 0.
\end{equation}
Writing $d\varphi/dt = \omega$, we obtain a quadratic equation
\begin{equation}
  (c^2 - \Omega^2 R^2)
  + 2\Omega R^2 \omega
  - R^2 \omega^2 = 0.
\end{equation}
Multiplying by $-1$,
\begin{equation}
  R^2 \omega^2
  - 2\Omega R^2 \omega
  - (c^2 - \Omega^2 R^2) = 0,
\end{equation}
whose solutions are
\begin{equation}
  \omega_\pm
  = \Omega \pm \frac{c}{R}.
\end{equation}
These correspond to co-rotating ($+$) and counter-rotating ($-$) null
signals around the ring in the rotating coordinates.

The coordinate times required to complete one full revolution
$\Delta\varphi = 2\pi$ are
\begin{equation}
  t_\pm
  = \int_0^{2\pi} \frac{d\varphi}{\omega_\pm}
  = \frac{2\pi}{\Omega \pm c/R}.
\end{equation}
For small angular velocities $\Omega R \ll c$ we can expand to first
order in $\epsilon \equiv \Omega R / c \ll 1$:
\begin{equation}
  t_+ \simeq \frac{2\pi R}{c}\,(1 - \epsilon),
  \qquad
  t_- \simeq \frac{2\pi R}{c}\,(1 + \epsilon).
\end{equation}
The difference between the co- and counter-rotating travel times is
then
\begin{equation}
  \Delta t
  \equiv t_- - t_+
  \simeq \frac{4\pi R}{c}\,\epsilon
  = \frac{4\pi R}{c}\,\frac{\Omega R}{c}
  = \frac{4\Omega A}{c^2},
\end{equation}
where $A=\pi R^2$ is the area enclosed by the ring. This is the familiar
Sagnac time delay for a circular loop.

In the weak-rotation regime the proper time difference measured by a
clock comoving with the ring is equal to $\Delta t$ up to corrections
of higher order in $(\Omega R/c)^2$, so we identify $\Delta\tau\simeq
\Delta t$ for our purposes.

%----------------------------------------------------------------------
\subsection{Phase-clock interpretation}
%----------------------------------------------------------------------

Within the phase-clock framework, we can interpret the ring as carrying
a phase degree of freedom that propagates along the loop with a fixed
frequency $\omega_0$ in proper time. For example, one may imagine a
phase signal launched from a reference point on the ring and allowed
to propagate once around the loop in both directions, after which
it is compared to a local reference clock.

If the intrinsic phase frequency in proper time is $\omega_0$, then the
phase accumulated along each path is approximately
\begin{equation}
  \phi_\pm
  \simeq \omega_0 \tau_\pm,
\end{equation}
where $\tau_\pm$ are the proper times along the co- and
counter-rotating paths. In the weak-rotation limit we may set
$\tau_\pm \simeq t_\pm$, so that the phase difference between the two
signals is
\begin{equation}
  \Delta\phi
  \equiv \phi_- - \phi_+
  \simeq \omega_0 (t_- - t_+)
  \simeq \omega_0 \frac{4\Omega A}{c^2}.
\end{equation}
This is a Sagnac-type phase shift proportional to the angular velocity
$\Omega$, the area $A$ enclosed by the ring, and the proper-time phase
frequency $\omega_0$.

The result is independent of the detailed microscopic realisation of
the phase signal. It applies equally to electromagnetic waves, matter
waves or superconducting phase modes, provided that a well-defined
phase with frequency $\omega_0$ in proper time can be associated with
the propagation around the ring. The phase-fibre language encapsulates
the effect as a consequence of the non-trivial $g_{0\varphi}$ component
in the rotating metric~\eqref{eq:rotating_metric}, i.e.\ as a holonomy
associated with the off-diagonal structure of the spacetime metric.

%======================================================================
\section{Discussion and outlook}
\label{sec:discussion}
%======================================================================

In this note we have provided a technical backend for the phase-fibre
picture developed in the companion concept paper. Starting from a
five-dimensional Kaluza--Klein metric with a compact phase fibre we
derived the standard four-dimensional Einstein--Maxwell action by
dimensional reduction. We then introduced a minimal worldline action
for a phase clock, in which the covariant phase rate along the fibre is
constrained to a fixed reference frequency in proper time. This yields a
clean evolution equation for the phase in a given background
$(g_{\mu\nu},A_\mu)$.

Two simple configurations were worked out explicitly. In a static
weak-field metric with two identical clocks at different heights we
recovered the standard gravitational redshift formula
$\Delta f/f \simeq \Delta\Phi/c^2$ directly from the phase-clock
condition and the relation between proper and coordinate time. In a
rotating-ring setup we derived a Sagnac-type phase shift
$\Delta\phi \simeq (\omega_0/c^2)\,4\Omega A$ for a phase signal
propagating once around the loop in both directions. Both results are
well-known in their respective contexts; the role of the phase-fibre
picture here is not to change the physics but to present it in a unified
geometric language in which gravity, electromagnetism and phase clocks
are treated on the same footing.

%----------------------------------------------------------------------
\subsection{Advantages and potential applications}
%----------------------------------------------------------------------

Although the present construction does not attempt to provide a
unified fundamental field theory, it does offer a non-trivial step
towards a unified \emph{description} of several phenomena that are
usually treated separately. From a practical point of view, the
phase-fibre language has several advantages:

\begin{itemize}
  \item \textbf{Bookkeeping for phase-based clocks in gravity and rotation.}
        For theorists working with Josephson junctions, SQUIDs or other
        phase-coherent devices in weak gravitational or rotating
        environments, the framework provides a compact bookkeeping tool:
        it makes explicit where $g_{00}$ enters (proper vs.\ coordinate
        time), where $A_\mu$ enters (electromagnetic coupling along
        the fibre), and how an effective phase inertia $I$ appears in
        the worldline description of the clock.
  \item \textbf{Natural interface to analogue gravity.}
        The construction fits naturally with analogue-gravity scenarios
        in superconductors and other coherent media. Phase textures,
        effective potentials and clock redshifts can all be described
        in the same geometric language, making it easier to compare
        different analogue models of gravitational effects.
  \item \textbf{Unified view of gravitational and Aharonov--Bohm-type phases.}
        Gravitational redshift, electromagnetic Aharonov--Bohm phases
        and rotation-induced (Sagnac) phase shifts all appear as
        manifestations of parallel transport along the phase fibre in
        a given background metric and gauge field. This provides a
        unified way of organising phase effects in superconducting
        loops and interferometric setups.
  \item \textbf{Potential reference for future precision experiments.}
        If future experiments attempt to measure gravitational redshifts
        or rotation-induced shifts directly at Josephson frequencies
        or in other macroscopic phase clocks, a technically explicit
        phase-fibre formulation can serve as a convenient reference:
        the action, field equations and clock configurations are all
        written in a single, conservative framework based on standard
        GR and Maxwell theory.
\end{itemize}

In this sense the phase-fibre framework is more than a change of
notation: it provides a coherent geometric language that connects
weak-field gravity, classical electromagnetism and macroscopic phase
dynamics in systems that are often studied by different communities.
This does not replace the underlying microscopic theories (BCS,
Ginzburg--Landau, RCSJ, etc.), but it helps to align their effective
descriptions when gravitational and rotational effects are relevant.

%----------------------------------------------------------------------
\subsection{Further directions}
%----------------------------------------------------------------------

Several natural extensions suggest themselves:
\begin{itemize}
  \item A more systematic treatment of the Newtonian phase-field sector
        described in the PhaseGeometry Part~III note, embedding the
        static scalar field $\varphi(\mathbf{x})$ and its effective
        energy density into the present framework as a low-energy
        limit for weak gravitational fields and slow clocks.
  \item Device-level configurations, such as thick SNS weak links with
        local magnetic phase control and SQUID-type rings, where the
        phase-fibre picture can be used as a bookkeeping tool for
        gravitational and rotational contributions to phase evolution
        on top of the standard RCSJ and Ginzburg--Landau analysis.
  \item Possible generalisations beyond the test-clock limit, where
        ensembles of phase clocks contribute non-trivially to the
        stress--energy tensor and current, and back-react on the
        geometry and electromagnetic field.
\end{itemize}

All of these directions remain well within classical four-dimensional
GR and Maxwell theory. The phase-fibre construction is best viewed as
a compact geometric language for organising weak-gravity, electromagnetic
and phase-clock effects, rather than as a proposal for modifying
general relativity or the Standard Model. In this sense the present
note closes the technical side of the phase-fibre construction at the
level of actions, field equations and simple clock configurations, and
provides a base for future work on concrete superconducting and
Josephson devices in weak gravitational and rotational fields.

\\ \\Finally, it is worth stating explicitly the conservative scope of the
present work. Throughout this note we remain strictly within standard
classical general relativity and Maxwell theory, in weak-field and
test-clock regimes. The phase-fibre construction does not introduce new
dynamical laws; it repackages known gravitational, electromagnetic and
phase-clock effects in a unified geometric language. No genuinely new
predictions are claimed at this stage. Possible extensions towards a
quantum layer for phase degrees of freedom, and towards regimes where
phase clocks back-react on the geometry, are deliberately left for
future work and for separate, more microscopic notes in the broader
PhaseGeometry programme.


%======================================================================
\section*{Acknowledgements}
%======================================================================

The author thanks the PhaseGeometry project for providing a consistent
context across the different notes, and acknowledges helpful discussions
with colleagues on the interpretation of phase-based clocks in weak
gravitational fields.

%======================================================================
\begin{thebibliography}{99}
%======================================================================

\bibitem{PhaseFibreSeries}
A.~Turchanov,
\newblock \emph{PhaseGeometry Phase-Fibre series: overview and core framework},
\newblock Zenodo series record (2025),
DOI:\,\texttt{10.5281/zenodo.17736918}.

\bibitem{PhaseFibreI}
A.~Turchanov,
\newblock \emph{PhaseGeometry Phase-Fibre Foundations I: Conceptual Framework},
\newblock Zenodo preprint (2025).

\bibitem{PhaseFieldIII}
A.~Turchanov,
\newblock \emph{PhaseGeometry Phase-Field Newtonian sector: phase textures and weak gravity},
\newblock Zenodo preprint (2025).

\bibitem{Carroll}
S.~M.~Carroll,
\newblock \emph{Spacetime and Geometry: An Introduction to General Relativity},
\newblock Addison--Wesley (2004).

\bibitem{Jackson}
J.~D.~Jackson,
\newblock \emph{Classical Electrodynamics}, 3rd ed.,
\newblock Wiley (1998).

\bibitem{Tinkham}
M.~Tinkham,
\newblock \emph{Introduction to Superconductivity}, 2nd ed.,
\newblock McGraw--Hill (1996).

\end{thebibliography}

\end{document}

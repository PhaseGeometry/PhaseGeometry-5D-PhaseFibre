\documentclass[11pt,a4paper]{article}

\usepackage[utf8]{inputenc}
\usepackage[T1]{fontenc}
\usepackage[english]{babel}

\usepackage{amsmath,amssymb}
\usepackage{geometry}
\usepackage{hyperref}
\usepackage{graphicx}
\geometry{margin=1in}

\numberwithin{equation}{section}

\title{PhaseGeometry Phase-Fibre Foundations III:\\[4pt]
Newtonian halo analogues from phase fields}

\author{Aleksey Turchanov}

\date{November 2025}

\begin{document}

\maketitle

\begin{center}
\small
Licensed under Creative Commons Attribution 4.0 International (CC BY 4.0).\\[2pt]
Part of the PhaseGeometry Phase-Fibre programme; archived in the Core Package\\
DOI: \texttt{10.5281/zenodo.17783322}.
\end{center}

\vspace{1em}

%======================================================================
\begin{abstract}
%======================================================================

Within the PhaseGeometry Phase-Fibre programme we develop a minimal
Newtonian phase-field sector in which gradients of a scalar phase field
act as an effective source for weak gravity. The aim is to show
explicitly how static phase textures can generate dark-matter-like halo
phenomenology while fitting naturally into the five-dimensional
phase-fibre picture established in Foundations~I and~II.

We introduce an abstract scalar phase field $\varphi(x)$ whose only role
is to define an effective mass density via its gradients,
$\rho_{\rm eff}\propto|\nabla\varphi|^2$, and we let this density source
the Newtonian potential in the standard Poisson equation. For static,
spherically symmetric configurations --- \emph{phase defects} --- we
consider simple radial profiles with a regular core, an intermediate
halo-like region, and a rapidly decaying outer tail. In a toy profile
with $d\varphi/dr\propto 1/r$ over an intermediate range the effective
density behaves as $\rho_{\rm eff}(r)\propto 1/r^2$, producing
$M(r)\propto r$ and an almost flat circular velocity
$v_c(r)\simeq\mathrm{const}$ in that region, as in dark-matter-dominated
haloes.

We formulate the conditions under which such halo-like behaviour
appears and discuss its interpretation in the broader PhaseGeometry
programme. In particular, the same phase configuration simultaneously
defines a gravitational potential $\Phi(r)$ and controls the local rate
of phase-based clocks through $g_{00}(r)\simeq 1+2\Phi(r)/c^2$. We then
outline how these Newtonian configurations embed into the five-dimensional
phase-fibre geometry of Foundations~I and~II, and sketch a roadmap
towards deeper phase defects and Newtonian black-hole analogues,
developed further in Foundations~IV. Throughout we remain within
standard Newtonian gravity and classical field theory; the phase-fibre
construction provides a compact geometric language rather than a
modification of general relativity.

\end{abstract}

\tableofcontents

%======================================================================
\section{Introduction and scope}
\label{sec:intro}
%======================================================================

The PhaseGeometry Phase-Fibre programme aims to repackage weak gravity,
classical electromagnetism and phase-based clocks into a compact
geometric language. In the conceptual note (Foundations~I) and its
technical companion (Foundations~II), four-dimensional spacetime
$(\mathcal{M}_4,g_{\mu\nu})$ is extended by a compact U(1) phase fibre
with coordinate $\varphi$, and a Kaluza--Klein metric is introduced in
which the electromagnetic potential $A_\mu$ and the metric component
$g_{00}$ enter on the same footing. Phase clocks are described as
worldlines that advance uniformly along the phase fibre with respect to
proper time, and weak-field gravitational redshift appears as a simple
geometric effect.

In the present note we focus on a complementary sector of the programme:
a Newtonian phase field that acts as a source for gravity and
simultaneously controls the rate of phase-based clocks. The idea is
deliberately minimal. We introduce an abstract scalar phase field
$\varphi(x)$ and associate to it an effective mass density
$\rho_{\rm eff}(x)$ proportional to the square of its gradients. This
density enters the Poisson equation for the Newtonian potential
$\Phi(x)$, which in turn determines the weak-field metric component
$g_{00}(x)\simeq 1+2\Phi(x)/c^2$ and hence the redshift of clocks,
including those whose internal dynamics are themselves phase-based.

Our goal is \emph{not} to construct a complete microscopic theory of
dark matter, but to show that within a minimal PhaseGeometry setting,
static phase textures can play a rôle analogous to dark-matter haloes at
the level of Newtonian phenomenology. In particular, we show that a
single static, spherically symmetric phase defect can generate:
\begin{itemize}
  \item an effective density profile with a roughly constant-density
        core, an intermediate region with $\rho_{\rm eff}(r)\propto 1/r^2$,
        and a rapidly decaying tail;
  \item an enclosed mass $M(r)$ that grows approximately linearly with
        radius in the intermediate region, $M(r)\propto r$;
  \item a nearly flat circular-velocity curve $v_c(r)\simeq\mathrm{const}$
        over a finite radial range, as in dark-matter-dominated haloes.
\end{itemize}

A key conceptual point is that the same phase texture $\varphi(r)$ is
responsible both for the gravitational well and for the local clock rate
through $g_{00}(r)$. In the broader Phase-Fibre picture this is exactly
the kind of configuration that can be lifted into a five-dimensional
geometry, where the Newtonian phase field, the weak-field metric and the
internal fibre coordinate are viewed as different projections of a
single phase-based structure.

Technically, this note sits just above the Phase-Fibre ``Technical
Passport'' (Foundations~0), which fixes the basic ontology and notation
of the five-dimensional phase-fibre framework. It shares the geometric
conventions of Foundations~I and~II, where the Kaluza--Klein metric and
phase-clock worldline actions are spelled out. Here we remain in the
Newtonian regime throughout, but we organise the phase-field sector in a
way that is ready to be embedded into the full phase-fibre construction.

The structure of the note is as follows. In
Sec.~\ref{sec:phase_field_sector} we define the Newtonian phase-field
setup, introducing the effective density $\rho_{\rm eff}$, the Poisson
equation with phase-field source, and the relevant scales. In
Sec.~\ref{sec:defects} we specialise to static, spherically symmetric
phase defects and analyse their qualitative properties.
Section~\ref{sec:rotation_curves} develops a simple toy profile and
shows under which conditions it produces halo-like rotation curves. In
Sec.~\ref{sec:embedding} we embed these Newtonian configurations into
the phase-fibre language of Foundations~I and~II.
Section~\ref{sec:BH_analogues} sketches how deeper defects may approach
a Newtonian black-hole-like regime and connects to analogue-horizon
intuition, providing a bridge to the explicit black-hole analogues
constructed in Foundations~IV. Finally, Sec.~\ref{sec:discussion}
summarises the main lessons, clarifies what we do and do not claim,
situates the work relative to other approaches, and outlines future
directions.

Throughout we assume standard Newtonian gravity and classical field
theory. The phase-fibre construction is used purely as a unifying
language for weak gravity, electromagnetism and phase clocks, not as a
proposal for modifying general relativity.

%======================================================================
\section{Newtonian phase-field sector: effective source for gravity}
\label{sec:phase_field_sector}
%======================================================================

In this section we introduce the Newtonian phase-field setup: an
abstract scalar phase field whose gradients generate an effective mass
density, which in turn sources the Newtonian potential in the usual
Poisson equation.

%----------------------------------------------------------------------
\subsection{Phase field and gradient energy}
%----------------------------------------------------------------------

We work in Euclidean three-space with Cartesian coordinates
$\mathbf{x}=(x^1,x^2,x^3)$ and consider a real scalar phase field
$\varphi(\mathbf{x})$. We deliberately keep $\varphi$ at an abstract
level: it may be thought of as a coarse-grained phase associated with an
underlying coherent medium, but for the purposes of this note its only
role is to define a structured effective density via its gradients.

We introduce the gradient-squared
\begin{equation}
  |\nabla\varphi(\mathbf{x})|^2
  \equiv \delta^{ij}\,\partial_i\varphi(\mathbf{x})\,
                      \partial_j\varphi(\mathbf{x}),
  \label{eq:grad_sq}
\end{equation}
and posit that the corresponding effective energy density is
\begin{equation}
  \rho_{\rm eff}(\mathbf{x})
  = \frac{\kappa}{2c^2}\,|\nabla\varphi(\mathbf{x})|^2,
  \label{eq:rho_eff_def}
\end{equation}
where $\kappa>0$ is a phenomenological constant and $c$ is the speed of
light. The factor of $c^2$ is included so that $\rho_{\rm eff}$ has the
dimensions of mass density.

In a more microscopic theory, $\kappa$ would be derived from the
underlying medium (e.g.\ a condensate stiffness or effective
Cooper-pair density). Here we simply treat $\kappa$ as a parameter
controlling the strength of the phase-field contribution to gravity.

%----------------------------------------------------------------------
\subsection{Poisson equation with phase-field source}
%----------------------------------------------------------------------

Given $\rho_{\rm eff}(\mathbf{x})$ we let it act as a source for the
Newtonian potential $\Phi(\mathbf{x})$ in the Poisson equation. In the
simplest case, with no additional matter components, we write
\begin{equation}
  \nabla^2\Phi(\mathbf{x})
  = 4\pi G\,\rho_{\rm eff}(\mathbf{x})
  = 2\pi G\,\frac{\kappa}{c^2}\,
    |\nabla\varphi(\mathbf{x})|^2,
  \label{eq:Poisson_pure}
\end{equation}
where $G$ is the Newton constant.

More generally, if ordinary matter with density $\rho_{\rm mat}$ is
present, the total source is
\begin{equation}
  \nabla^2\Phi(\mathbf{x})
  = 4\pi G\bigl[\rho_{\rm mat}(\mathbf{x})
                + \rho_{\rm eff}(\mathbf{x})\bigr].
  \label{eq:Poisson_total}
\end{equation}
In this note we mostly focus on configurations where $\rho_{\rm eff}$
dominates the gravitational potential in some radial range of interest,
so that ordinary matter can be neglected as a first approximation.

In the weak-field limit, the Newtonian potential determines the
time-time component of the metric via
\begin{equation}
  g_{00}(\mathbf{x})
  \simeq 1 + \frac{2\Phi(\mathbf{x})}{c^2},
\end{equation}
which in turn controls the rate of phase-based clocks located at
$\mathbf{x}$. Thus the same phase field $\varphi$ generates both a
gravitational well and a pattern of clock redshifts.

%----------------------------------------------------------------------
\subsection{Parameters and scales}
%----------------------------------------------------------------------

The single parameter $\kappa$ controls the overall strength of the
phase-field contribution to the gravitational potential. Two additional
scales enter the problem:
\begin{itemize}
  \item a core scale $r_{\rm core}$, below which the phase field is
        approximately constant and the effective density is roughly
        homogeneous;
  \item a decay scale $r_{\rm out}$, beyond which the gradients of
        $\varphi$ become negligible and the effective density rapidly
        falls off.
\end{itemize}
Between these scales there may exist an extended intermediate range
$r_{\rm core}\ll r\ll r_{\rm out}$ in which the phase gradients follow a
simple power-law form and generate halo-like density profiles.

In the next section we specialise to static, spherically symmetric
configurations --- phase defects --- and analyse how such scales appear
naturally in the radial structure of $\varphi(r)$, $\rho_{\rm eff}(r)$,
$M(r)$ and the circular velocity $v_c(r)$.

%======================================================================
\section{Static spherically symmetric phase defects}
\label{sec:defects}
%======================================================================

We now specialise to static, spherically symmetric configurations in
which the phase field depends only on the radial coordinate:
\begin{equation}
  \varphi(\mathbf{x}) = \varphi(r), \qquad r = |\mathbf{x}|.
  \label{eq:phi_of_r}
\end{equation}
Such configurations will be referred to as \emph{phase defects}.

%----------------------------------------------------------------------
\subsection{Gradient energy and effective density}
%----------------------------------------------------------------------

For a purely radial profile the gradient squared is
\begin{equation}
  |\nabla\varphi(\mathbf{x})|^2
  = \left(\frac{d\varphi}{dr}\right)^2,
  \label{eq:grad_radial}
\end{equation}
and the effective density \eqref{eq:rho_eff_def} becomes
\begin{equation}
  \rho_{\rm eff}(r)
  = \frac{\kappa}{2c^2}
    \left(\frac{d\varphi}{dr}\right)^2.
  \label{eq:rho_eff_radial}
\end{equation}

The effective mass enclosed within radius $r$ is
\begin{equation}
  M(r)
  = 4\pi\int_0^r dr'\,r'^2\,\rho_{\rm eff}(r')
  = \frac{2\pi\kappa}{c^2}
    \int_0^r dr'\,r'^2
    \left(\frac{d\varphi}{dr'}\right)^2.
  \label{eq:M_r_general}
\end{equation}
The corresponding Newtonian potential satisfies
\begin{equation}
  \frac{d\Phi}{dr}
  = \frac{GM(r)}{r^2},
  \label{eq:dPhi_dr_general}
\end{equation}
and at large $r$ one recovers the usual $1/r$ tail if $M(r)$ tends to a
finite limit.

In what follows we do not derive $\varphi(r)$ from a specific microscopic
field equation. Instead, we adopt a phenomenological approach: we
consider simple radial profiles with physically reasonable behaviour
(regular core, decaying gradients at large $r$) and analyse the
resulting $\rho_{\rm eff}(r)$, $M(r)$ and circular velocity $v_c(r)$.

%----------------------------------------------------------------------
\subsection{Qualitative structure of a phase defect}
%----------------------------------------------------------------------

We assume that a physically acceptable phase defect obeys the following
qualitative conditions:
\begin{enumerate}
  \item \textbf{Regular core:} near $r=0$ the phase is approximately
        constant, $\varphi(r)\simeq\varphi_0$, so that $d\varphi/dr\simeq 0$
        and $\rho_{\rm eff}(r)$ remains finite.
  \item \textbf{Intermediate halo-like region:} for
        $r_{\rm core}\ll r\ll r_{\rm out}$, the radial derivative
        $d\varphi/dr$ follows a simple power law, leading to a density
        profile with $\rho_{\rm eff}(r)\propto 1/r^2$ and hence
        $M(r)\propto r$.
  \item \textbf{Outer tail:} for $r\gg r_{\rm out}$ the phase approaches
        a constant, gradients decay rapidly, and $\rho_{\rm eff}(r)$
        falls off faster than $1/r^3$, so that $M(r)$ tends to a finite
        limit.
\end{enumerate}

Under these conditions the phase defect generates a weak gravitational
well with a roughly constant-density core, an extended halo-like region
with $\rho_{\rm eff}(r)\sim 1/r^2$, and an outer Keplerian regime.

%----------------------------------------------------------------------
\subsection{Toy radial profile and three regimes}
%----------------------------------------------------------------------

As a concrete illustration we consider a toy profile in which the radial
derivative of the phase takes the form
\begin{equation}
  \frac{d\varphi}{dr}
  \simeq
  \begin{cases}
    0,
      & 0 \le r \lesssim r_{\rm core},\\[4pt]
    \displaystyle \frac{\alpha}{r},
      & r_{\rm core}\ll r\ll r_{\rm out},\\[8pt]
    \displaystyle \frac{\alpha}{r}\,f(r),
      & r\gtrsim r_{\rm out},
  \end{cases}
  \label{eq:dphi_dr_toy}
\end{equation}
where $\alpha$ is a dimensionless phase amplitude and $f(r)$ is a
decaying function that ensures rapid fall-off at large radii (e.g.\ an
exponential or a higher power of $r$ in the denominator).

In the intermediate region this gives
\begin{equation}
  \rho_{\rm eff}(r)
  \simeq
  \frac{\kappa}{2c^2}\,\frac{\alpha^2}{r^2},
  \qquad
  r_{\rm core}\ll r\ll r_{\rm out},
  \label{eq:rho_eff_intermediate}
\end{equation}
so that
\begin{equation}
  M(r)
  \simeq 4\pi\int_0^r dr'\,r'^2\,\rho_{\rm eff}(r')
  \simeq 4\pi\int_{r_{\rm core}}^r
         dr'\,r'^2\,\frac{\kappa}{2c^2}\,\frac{\alpha^2}{r'^2}
  = \frac{2\pi\kappa\alpha^2}{c^2}\,(r - r_{\rm core}).
  \label{eq:M_r_intermediate_exact}
\end{equation}
For $r\gg r_{\rm core}$ this simplifies to
\begin{equation}
  M(r)
  \simeq \frac{2\pi\kappa\alpha^2}{c^2}\,r,
  \qquad
  r_{\rm core}\ll r\ll r_{\rm out}.
  \label{eq:M_r_intermediate_linear}
\end{equation}

The circular velocity of a test particle on a circular orbit of radius
$r$ is
\begin{equation}
  v_c^2(r)
  = \frac{GM(r)}{r}
  \simeq \frac{2\pi G\kappa\alpha^2}{c^2},
  \label{eq:vc2_const}
\end{equation}
i.e.\ approximately constant:
\begin{equation}
  v_c(r)
  \simeq v_0 \equiv
  \left(
    \frac{2\pi G\kappa\alpha^2}{c^2}
  \right)^{1/2},
  \qquad
  r_{\rm core}\ll r\ll r_{\rm out}.
  \label{eq:vc_const}
\end{equation}

This is the characteristic halo-like behaviour that we will develop in
more detail in the next section. In the core and outer regimes the
density and mass profiles behave differently (approximately constant
density in the core and rapidly decaying density in the tail), leading
to rising $v_c(r)$ at small $r$ and Keplerian fall-off at large $r$.

%======================================================================
\section{Rotation curves and halo phenomenology}
\label{sec:rotation_curves}
%======================================================================

We now compute the circular-velocity profiles for the phase-field
defects introduced above and discuss the conditions under which they
exhibit halo-like behaviour.

%----------------------------------------------------------------------
\subsection{Circular velocity and enclosed mass}
%----------------------------------------------------------------------

For a spherically symmetric potential $\Phi(r)$ the circular velocity of
a test particle on a circular orbit of radius $r$ is given by
\begin{equation}
  v_c^2(r)
  = r\,\frac{d\Phi}{dr}
  = \frac{GM(r)}{r},
  \label{eq:vc_general}
\end{equation}
where $M(r)$ is the enclosed mass defined in
Eq.~\eqref{eq:M_r_general}.

Using the toy profile \eqref{eq:dphi_dr_toy} one obtains three regimes:
\begin{itemize}
  \item \textbf{Core ($r\lesssim r_{\rm core}$):}
        $\rho_{\rm eff}(r)\simeq\rho_0\approx\mathrm{const}$ and
        $M(r)\propto r^3$, so $v_c(r)\propto r$.
  \item \textbf{Intermediate (halo-like) region
        ($r_{\rm core}\ll r\ll r_{\rm out}$):}
        $M(r)\propto r$ as in \eqref{eq:M_r_intermediate_linear}, so
        $v_c(r)\approx\mathrm{const}$.
  \item \textbf{Outer tail ($r\gg r_{\rm out}$):}
        $M(r)$ tends to a constant and $v_c(r)\propto r^{-1/2}$.
\end{itemize}

%----------------------------------------------------------------------
\subsection{Three-regime circular-velocity profile}
%----------------------------------------------------------------------

Equations~\eqref{eq:vc_general} and the results above show that a single
static phase defect generates three distinct regimes in the circular
velocity: a solid-body core with $v_c \propto r$, an approximately flat
``halo'' region with $v_c \approx \mathrm{const}$, and an outer
Kepler-like decline $v_c \propto r^{-1/2}$. A schematic example is
shown in Fig.~\ref{fig:vc_schematic}.

\begin{figure}[t]
  \centering
  \includegraphics[width=0.75\textwidth]{5D_III_1.png}
  \caption{Schematic circular-velocity profile $v_c(r)$ generated by
  a single static phase defect in the Newtonian phase-field model.
  Three characteristic regimes are indicated: a solid-body core
  ($v_c \propto r$ for $r \lesssim r_{\rm core}$), an approximately
  flat ``halo'' region ($v_c \approx \mathrm{const}$ for
  $r_{\rm core} \lesssim r \lesssim r_{\rm out}$), and an outer
  Kepler-like decline ($v_c \propto r^{-1/2}$ for
  $r \gtrsim r_{\rm out}$). Radii $r_{\rm core}$ and $r_{\rm out}$ are
  shown only as indicative scales; both axes are in arbitrary units,
  and the curve is not a fit to any particular galaxy but a qualitative
  illustration of the analytic results in this section.}
  \label{fig:vc_schematic}
\end{figure}

%----------------------------------------------------------------------
\subsection{Conditions for flat or gently rising curves}
%----------------------------------------------------------------------

The flatness of the circular velocity in the intermediate regime depends
on how accurately the density follows $\rho_{\rm eff}(r)\propto 1/r^2$
and on the width of the interval $[r_{\rm core},r_{\rm out}]$. In a more
realistic treatment, deviations from the idealised
$d\varphi/dr\propto 1/r$ profile and the presence of ordinary matter
will produce gently rising or falling rotation curves.

We can summarise the requirements for an approximately flat region as:
\begin{enumerate}
  \item The phase gradient should behave as
        $d\varphi/dr \simeq \alpha/r$ over at least one decade in
        radius.
  \item The effective density should be dominated by the phase field
        over that interval, i.e.\ $\rho_{\rm eff}\gg \rho_{\rm mat}$.
  \item The core and outer scales should be well separated,
        $r_{\rm out}/r_{\rm core}\gg 1$, to provide an extended region
        where $M(r)\propto r$ holds to good approximation.
\end{enumerate}

When these conditions are satisfied, the phase-field defect provides a
halo-like gravitational well with a flat or gently rising circular
velocity profile over a finite but potentially broad radial range.

%----------------------------------------------------------------------
\subsection{Interpretation of halo-like phase defects}
%----------------------------------------------------------------------

From the Newtonian perspective, the key point is that the same phase
texture $\varphi(r)$:
\begin{itemize}
  \item defines a structured effective mass density via its gradients,
        $\rho_{\rm eff}(r)\propto \bigl(d\varphi/dr\bigr)^2$;
  \item generates a gravitational potential $\Phi(r)$ via the Poisson
        equation with $\rho_{\rm eff}$ as the source;
  \item controls the rate of phase-based clocks through $g_{00}(r)$ and
        the weak-field relation $g_{00}(r)\simeq 1+2\Phi(r)/c^2$.
\end{itemize}

In this sense a static phase defect provides a single geometric object
that simultaneously shapes the gravitational well and the local time
rate of clocks (including those whose internal dynamics are themselves
phase-based).

In the broader PhaseGeometry programme this is precisely the type of
structure that can be lifted into the phase-fibre language, where the
Newtonian phase field $\varphi(r)$, the potential $\Phi(r)$ and the
internal fibre coordinate $\varphi$ can be viewed as different
projections of one and the same phase-based geometry.

%======================================================================
\section{Embedding into phase-fibre geometry}
\label{sec:embedding}
%======================================================================

We now explain how the Newtonian phase-field configurations described
above embed into the phase-fibre framework developed in
Foundations~I and~II.

%----------------------------------------------------------------------
\subsection{From phase field to weak-field metric}
%----------------------------------------------------------------------

Given a static phase configuration $\varphi(\mathbf{x})$ and the
resulting effective density $\rho_{\rm eff}(\mathbf{x})$, the Newtonian
potential $\Phi(\mathbf{x})$ is obtained by solving the Poisson equation
\eqref{eq:Poisson_pure} or \eqref{eq:Poisson_total}. In the weak-field
regime the corresponding metric is
\begin{equation}
  ds^2
  = g_{00}(\mathbf{x})\,c^2 dt^2
    - \delta_{ij}\,dx^i dx^j,
  \qquad
  g_{00}(\mathbf{x}) \simeq 1 + \frac{2\Phi(\mathbf{x})}{c^2},
  \label{eq:weak_field_metric}
\end{equation}
which is the same weak-field background used in the phase-clock
examples of Foundations~II.

In particular, for the spherically symmetric defects considered above
we obtain $g_{00}(r)$ directly from $\Phi(r)$ via
Eq.~\eqref{eq:dPhi_dr_general}, and the halo-like region
$M(r)\propto r$ translates into a weak-field metric with a
logarithmically growing potential and approximately constant circular
velocity.

%----------------------------------------------------------------------
\subsection{Phase clocks in halo backgrounds}
%----------------------------------------------------------------------

Phase-based clocks (superconducting condensates, Josephson oscillators,
atomic clocks viewed as phase devices, etc.) are described in the
phase-fibre framework as timelike worldlines $x^\mu(\tau)$ with an
internal fibre coordinate $\varphi(\tau)$ whose covariant phase rate is
constrained to a reference frequency in proper time. In the notation of
Foundations~II, the covariant phase rate along a worldline is
\begin{equation}
  \frac{D\varphi}{D\tau}
  = \frac{d\varphi}{d\tau} + k A_\mu u^\mu,
\end{equation}
and the clock condition enforces $D\varphi/D\tau = \omega_0$.

In a weak-field halo background with metric
\eqref{eq:weak_field_metric}, clocks located at different radii $r$
experience different proper-time rates relative to coordinate time:
\begin{equation}
  d\tau
  = \sqrt{g_{00}(r)}\,dt
  \simeq \left(1 + \frac{\Phi(r)}{c^2}\right) dt.
\end{equation}
As in Foundations~II, the local clock frequency with respect to
coordinate time is then
\begin{equation}
  f(r)
  = \frac{\omega_0}{2\pi}\sqrt{g_{00}(r)}
  \simeq \frac{\omega_0}{2\pi}
         \left(1 + \frac{\Phi(r)}{c^2}\right),
\end{equation}
and the relative frequency shift between two radii is
\begin{equation}
  \frac{\Delta f}{f}
  \simeq \frac{\Delta\Phi}{c^2}.
\end{equation}

Thus, once the phase defect $\varphi(r)$ and the effective density
$\rho_{\rm eff}(r)$ are specified, the same defect structure controls
both the approximate flatness of $v_c(r)$ and the pattern of clock
redshifts across the halo.

%----------------------------------------------------------------------
\subsection{Relation to phase-fibre cartoons}
%----------------------------------------------------------------------

In the five-dimensional phase-fibre picture, spacetime
$(\mathcal{M}_4,g_{\mu\nu})$ is extended by a compact U(1) fibre with
coordinate $\varphi\in[0,2\pi)$, and the Kaluza--Klein metric
\begin{equation}
  ds_5^2
  = g_{\mu\nu}(x)\,dx^\mu dx^\nu
    + R^2\bigl(d\varphi + kA_\mu(x)\,dx^\mu\bigr)^2
\end{equation}
encodes both gravity and electromagnetism. In this language,
phase-based clocks are worldlines that advance uniformly along the
fibre, while electromagnetic effects enter through the connection
$A_\mu$.

The Newtonian phase field $\varphi(\mathbf{x})$ considered in this note
can be thought of as a coarse-grained imprint of how the local phase
environment modulates the effective mass density and thus the weak-field
metric. In qualitative cartoons, one may imagine that regions with large
$|\nabla\varphi|$ correspond to ``thicker'' or more strongly twisted
segments of the phase fibre in the underlying five-dimensional geometry,
whose gradient energy appears in the four-dimensional description as
$\rho_{\rm eff}(x)$.

While we do not attempt here to derive such an embedding from a full
five-dimensional action, the conceptual alignment is straightforward:
phase defects in the Newtonian sector are exactly the kind of
inhomogeneous phase structures that the phase-fibre framework is
designed to describe.

%======================================================================
\section{Towards Newtonian black-hole analogues}
\label{sec:BH_analogues}
%======================================================================

Finally, we comment briefly on the regime of deeper defects and the
limitations of the Newtonian approximation, and sketch how one might
extend the present picture towards black-hole-like configurations.
These ideas are developed more concretely for compact phase-fibre
defects in Foundations~IV.

%----------------------------------------------------------------------
\subsection{Deep defects and breakdown of the Newtonian approximation}
\label{sec:deep_defects}
%----------------------------------------------------------------------

The analysis so far has assumed weak fields, $|\Phi(r)|/c^2\ll 1$, and
phase gradients that are moderate enough for the Newtonian treatment to
remain valid. In principle, however, one can consider phase defects
where the gradient energy density becomes large enough that the
resulting potential $\Phi(r)$ approaches $-c^2/2$ in some region,
corresponding formally to $g_{00}(r)\to 0$ in the weak-field metric
\eqref{eq:weak_field_metric}.

In such \emph{deep defect} configurations the Newtonian approximation
breaks down. A fully relativistic treatment would be required, with the
phase field (or its relativistic counterpart) entering the Einstein
equations and back-reacting on the metric. From the viewpoint of
coherent media and analogue gravity this situation is reminiscent of
\emph{analogue horizons}: in acoustic or optical analogue models,
regions where the flow speed exceeds the relevant signal speed (sound
speed, group velocity) define effective horizons for perturbations.
Here one may view deep defects as creating regions where the effective
escape conditions for phase excitations mimic a horizon-like behaviour,
even though the underlying description is still classical and
non-relativistic.

We do not attempt to model such horizons quantitatively in this note.
The key point is qualitative: as $|\Phi(r)|/c^2$ approaches order unity,
the same phase texture that previously acted as a mild halo source now
creates a strong gravitational well in which notions of escape, trapping
and redshift for phase-based signals begin to resemble those in
black-hole and analogue-horizon settings.

%----------------------------------------------------------------------
\subsection{Roadmap for relativistic extensions}
%----------------------------------------------------------------------

A natural next step is to embed the phase-field sector into a
relativistic PhaseGeometry action, where the phase field (or fields)
contribute directly to the stress--energy tensor and the full Einstein
equations are solved. In such a framework one could:
\begin{itemize}
  \item study relativistic phase defects whose weak-field limit reduces
        to the Newtonian haloes described here;
  \item analyse whether certain defect configurations admit genuine
        horizons or only remain in the quasi-Newtonian regime;
  \item explore the interplay between phase-based clocks, redshift and
        strong gravity in the phase-fibre geometry.
\end{itemize}

The Newtonian analysis of this note can then be viewed as a controlled
low-energy limit of a more complete relativistic phase-fibre model,
providing intuition and phenomenological guidance for the fully
relativistic constructions.

%======================================================================
\section{Discussion and outlook}
\label{sec:discussion}
%======================================================================

We close by summarising the main lessons, stating explicitly what we do
\emph{not} claim, relating the present work to other approaches, and
listing a few directions for future exploration.

%----------------------------------------------------------------------
\subsection{Take-home message}
%----------------------------------------------------------------------

The core message of this note can be stated in a few lines:

\begin{itemize}
  \item A single scalar phase field $\phi(\mathbf{x})$, treated as an
        effective Newtonian degree of freedom with
        $\rho_{\rm eff}\propto|\nabla\phi|^2$, can generate halo-like
        gravitational potentials when its static configurations include
        defects with an intermediate region where $d\phi/dr\propto 1/r$.
  \item In that region the effective density scales as
        $\rho_{\rm eff}(r)\propto 1/r^2$, the enclosed mass grows as
        $M(r)\propto r$, and the circular-velocity curve is
        approximately flat, $v_c(r)\simeq\mathrm{const}$.
  \item The same phase texture that shapes the halo also controls the
        local clock rate via $g_{00}(r)\simeq 1+2\Phi(r)/c^2$, so that
        phase-based clocks (including Josephson and other coherent
        systems) run faster or slower depending on their position in the
        defect-induced potential.
\end{itemize}

In the broader PhaseGeometry programme, this provides a concrete
example of how phase structures can simultaneously encode gravitational
wells and clock redshifts, fitting naturally into the phase-fibre
language of Foundations~I and~II and linking forward to the black-hole
analogues of Foundations~IV.

%----------------------------------------------------------------------
\subsection{What we do not claim}
%----------------------------------------------------------------------

It is equally important to state explicitly what we \emph{do not} claim:

\begin{itemize}
  \item We do not claim to have derived the observed dark-matter
        phenomenology from a fundamental microscopic theory. The
        phase-field model here is phenomenological and minimal.
  \item We do not introduce new particles, fields or non-standard
        gravitational couplings beyond the effective phase field and the
        Newtonian potential. The underlying gravity is standard
        Newtonian (and, in prospective extensions, standard GR).
  \item We do not address nonlinear structure formation, cosmological
        constraints, or detailed fits to galactic rotation curves. The
        focus is on static toy configurations that exhibit the basic
        halo-like features.
  \item We do not claim that every dark-matter-like halo must arise from
        phase defects of this kind. The construction is intended as a
        proof-of-principle within the PhaseGeometry paradigm, not as a
        unique explanation.
\end{itemize}

Within these boundaries, the Newtonian phase-field sector provides a
clean and controllable playground for phase-based halo analogues.

%----------------------------------------------------------------------
\subsection{Relation to other approaches}
%----------------------------------------------------------------------

The idea that extended scalar-field configurations can mimic dark-matter
haloes is not new: scalar-field dark matter, fuzzy dark matter and
various condensate models all explore related themes. The present
PhaseGeometry construction differs in three respects:

\begin{itemize}
  \item The field is treated explicitly as a \emph{phase} degree of
        freedom, with an emphasis on its gradient energy and its role in
        controlling phase clocks, rather than as a generic scalar
        matter component.
  \item The halo-like behaviour is obtained in a minimal, static,
        Newtonian setting, without invoking quantum pressure or wave
        interference; the key ingredient is the $1/r$ behaviour of the
        phase gradient in an intermediate region.
  \item The entire construction is designed from the outset to fit into
        the phase-fibre geometry of Foundations~0/I/II, where phase
        clocks, electromagnetic potentials and weak gravity are described
        in a unified higher-dimensional language.
\end{itemize}

In this sense, the Newtonian phase-field haloes of the present note are
best viewed as a PhaseGeometry-specific realisation of a broader class
of scalar-halo ideas, tailored to interface cleanly with the
phase-fibre framework and to connect smoothly to the black-hole
analogues in Foundations~IV.

%----------------------------------------------------------------------
\subsection{Future directions}
%----------------------------------------------------------------------

Several natural extensions suggest themselves:

\begin{itemize}
  \item \textbf{Relativistic embedding:} formulate a relativistic
        phase-fibre action in which phase fields contribute to the
        stress--energy tensor, and study static spherically symmetric
        solutions whose weak-field limit reproduces the halo analogues
        described here.
  \item \textbf{More realistic profiles:} replace the toy
        $d\phi/dr\propto 1/r$ ansatz by solutions of a concrete field
        equation, possibly with self-interactions, and compare the
        resulting $v_c(r)$ to observed galactic rotation curves.
  \item \textbf{Coupling to ordinary matter:} include baryonic matter
        and explore how the phase-field halo coexists with disks and
        bulges, and how this affects the flatness and extent of
        rotation curves.
  \item \textbf{Analogue gravity and horizons:} develop the connection
        sketched in Sec.~\ref{sec:deep_defects} between deep phase
        defects, strong potentials and analogue-horizon intuition in
        coherent media, linking more explicitly to the phase-fibre
        black-hole analogues of Foundations~IV.
\end{itemize}

All these directions remain within standard classical gravity (Newtonian
or GR). The role of the phase-fibre framework is to provide a coherent
geometric language in which weak gravity, electromagnetism and
phase-based clocks can be discussed together, and in which phase-field
haloes appear as natural structural elements rather than ad hoc
additions.

To summarise the scope, the Newtonian halo configurations constructed
here are deliberately conservative: they are classical, static toy
models within standard Newtonian gravity, sourced by an effective
phase-field density $\rho_{\rm eff}\propto|\nabla\phi|^2$. The
phase-field is treated as a macroscopic, coarse-grained degree of
freedom rather than as a fundamental quantum field, and no specific new
observational predictions are claimed beyond the standard halo
phenomenology encoded in $v_c(r)$. A genuine quantum layer for phase
degrees of freedom, and fully relativistic Phase-Fibre halo and
black-hole solutions, are reserved for separate work.



%======================================================================
\section*{Acknowledgements}
%======================================================================

The author thanks the PhaseGeometry project for providing a consistent
context across the different notes, and acknowledges helpful discussions
with colleagues on the interpretation of phase-based clocks and halo
phenomenology in weak gravitational fields.

%======================================================================
\begin{thebibliography}{99}
%======================================================================

\bibitem{PhaseFibreSeries}
A.~Turchanov,
\newblock \emph{PhaseGeometry Phase-Fibre series: overview and core framework},
\newblock Zenodo series record (2025),
DOI:\,\texttt{10.5281/zenodo.17783322}.

\bibitem{Carroll}
S.~M.~Carroll,
\newblock \emph{Spacetime and Geometry: An Introduction to General Relativity},
\newblock Addison--Wesley (2004).

\bibitem{Jackson}
J.~D.~Jackson,
\newblock \emph{Classical Electrodynamics}, 3rd ed.,
\newblock Wiley (1998).

\end{thebibliography}

\end{document}

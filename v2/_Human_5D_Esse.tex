\documentclass[11pt,a4paper]{article}

\usepackage[utf8]{inputenc}
\usepackage[T1]{fontenc}
\usepackage[english]{babel}

\usepackage{amsmath,amssymb}
\usepackage{geometry}
\geometry{margin=1in}

\title{Phase as the Fifth Dimension:\\[4pt]
from 4D to 5D without mysticism}

\author{Aleksey Turchanov}
\date{November 2025}

\begin{document}

\maketitle

\begin{center}
\small
This essay is part of the \emph{PhaseGeometry Phase-Fibre programme}.\\[2pt]

\scriptsize
Licensed under Creative Commons Attribution 4.0 International (CC BY 4.0).\\[1pt]
PhaseGeometry Phase-Fibre Core Package v1.0: DOI\,\texttt{10.5281/zenodo.17783322}.
\end{center}

\vspace{1em}

\begin{abstract}
Since Newton we have been used to saying that we live in a four-dimensional
world ``space $3D$ + time $t$'' and to describing phenomena in these
coordinates. In practice, however, we constantly use one more coordinate:
the phase $\phi$, which tells us ``where in the cycle'' a process is.
Ocean waves, tides, planetary orbits, changing seasons, heartbeats,
pendulum motion, day--night cycles, ``project stages'' or ``sleep phases''
are all most naturally described in terms of their position along a cycle.

In effect we have long been living in a five-dimensional world
$3D + t + \phi$, where phase is a coordinate of state: it answers the
question ``which point of a repeating process the system currently
occupies''. This note suggests treating phase as a genuine ``fifth
dimension'' and looking at physics from that angle --- without mysticism,
but with a new language that later connects naturally to the
phase-fibre PhaseGeometry framework.
\end{abstract}

%======================================================================
\section{Phase as a forgotten coordinate}
%======================================================================

The classical ``shop-window'' picture of physics is simple: there are
three spatial coordinates $(x,y,z)$ and there is time $t$. Together
this gives $3D + t$ --- a four-dimensional description everyone is
used to.

But this picture answers only two questions:
\emph{where} and \emph{when}. Very often we also care about a third:
\begin{quote}
\textbf{In which stage of its process are we now?}
\end{quote}

This third ingredient --- \emph{position within the course of a
process} --- has traditionally been pushed ``outside the brackets'' and
hidden inside heavy formulas, parametric descriptions and special
notations. Meanwhile, it can be organised as one more coordinate, and a
fairly simple one: the \emph{phase} $\phi$.

%======================================================================
\section{What is phase?}
%======================================================================

From a physics point of view:
\begin{quote}
Phase $\phi$ is a quantity that tells us at which point of its repeating
or developing process a system is, relative to some reference state.
\end{quote}

In everyday terms:
\begin{quote}
Phase is the ``stage'' of a process expressed as a number: an angle on
a circle, a fraction between 0 and 1, or a value between 0 and $2\pi$.
\end{quote}

We can distinguish two main regimes.

%----------------------------------------------------------------------
\subsection{Linear process (with beginning and end)}
%----------------------------------------------------------------------

For processes with a clear start and finish it is convenient to think:
\begin{itemize}
  \item $\phi = 0$ --- start;
  \item $\phi = \pi$ --- roughly halfway;
  \item $\phi = 2\pi$ --- completion of one pass through the process.
\end{itemize}

One can define phase via the completion fraction:
\begin{equation}
  \phi = 2\pi \times \text{completion fraction between 0 and 1}.
\end{equation}
For example, a project is 50\% done $\Rightarrow \phi = \pi$.

%----------------------------------------------------------------------
\subsection{Cyclic process (repeats again and again)}
%----------------------------------------------------------------------

For repeating processes (waves, daily rhythm, rotations):
\begin{itemize}
  \item $\phi = 0$ and $\phi = 2\pi$ describe the same state; only one
        more turn has been completed;
  \item here phase is not ``start--finish'' but \emph{position on the
        circle}.
\end{itemize}

So:
\begin{itemize}
  \item for project-like, life-like, single-pass processes it is natural
        to think: $0$ --- beginning, $2\pi$ --- end of one run;
  \item for oscillators and waves $2\pi$ means a return to the initial
        state on a new cycle.
\end{itemize}

%======================================================================
\section{We already speak the language of phase, we just do not call it so}
%======================================================================

In everyday speech, phase language has long been present:
\begin{itemize}
  \item ``I'm in a different phase of life now'';
  \item ``The project is at an early stage'';
  \item ``Sleep was shallow, I woke up in the wrong phase'';
  \item ``The economy has entered a recession phase''.
\end{itemize}

In all these cases we are in fact describing a \emph{coordinate of
state}, but not via $(x,y,z,t)$; instead we talk about the stage of
a process.

Examples:
\begin{itemize}
  \item \textbf{Day}: morning, afternoon, evening, night --- four rough
        phases of the daily cycle;
  \item \textbf{Breathing}: inhale $\rightarrow$ pause $\rightarrow$
        exhale $\rightarrow$ pause;
  \item \textbf{Heart}: systole / diastole;
  \item \textbf{Life of a project}: idea $\rightarrow$ prototype
        $\rightarrow$ pilot $\rightarrow$ deployment
        $\rightarrow$ maintenance.
\end{itemize}

Thus, on top of $3D + t$ we constantly add one more coordinate $\phi$:
\begin{quote}
\emph{position within the process}.
\end{quote}
We simply do not make this coordinate explicit in the usual picture of
the world.

%======================================================================
\section{How physics has historically treated phase}
%======================================================================

It is important to understand: physics did not ``sleep through'' the
idea of phase. On the contrary, within physics, phase has been a
working tool for many generations.

%----------------------------------------------------------------------
\subsection{Waves and oscillations}
%----------------------------------------------------------------------

For a harmonic wave one writes:
\begin{equation}
  h(x,t) = A \cos(kx - \omega t + \phi_0),
\end{equation}
where $\phi_0$ is the initial phase. At time $t$ the state of the
system (wave height, voltage, current) is determined by the argument
\[
  \phi(x,t) = kx - \omega t + \phi_0,
\]
i.e.\ by a point on the circle $S^1$.

Before the compact phase language became standard, this was expressed
in much more cumbersome terms:
\begin{itemize}
  \item people spoke of ``delay'' of one wave relative to another
        as a fraction of a period;
  \item in optics one operated with ``path difference'' of rays instead
        of simply talking about a phase shift.
\end{itemize}
Phase compresses all of that into a single parameter $\phi$.

%----------------------------------------------------------------------
\subsection{State coordinates inside physics}
%----------------------------------------------------------------------

In different areas of physics, coordinates of state have existed for a
long time, even if they are rarely presented to the outside world as
one unified image:
\begin{itemize}
  \item \emph{phase space} $(x,p)$ in classical mechanics;
  \item \emph{configuration space} $q_i$ for complex systems;
  \item in quantum mechanics the state is a vector in Hilbert space;
  \item in thermodynamics a state is given by $(T,P,V,S,\dots)$;
  \item in field theory a state is a field configuration $\varphi(x)$
        in all points.
\end{itemize}

So \emph{inside} physics, state coordinates do exist, and many of them.
But they are:
\begin{itemize}
  \item fragmented between subfields;
  \item not presented as a simple, unified ``fifth layer'' of world
        description, understandable both to physicists and
        non-specialists.
\end{itemize}

The shop-window picture remains $3D + t$, while everything that
concerns state is hidden in special notations and equations. From the
point of view of the cultural image of the world, this part is
effectively ``pulled outside the brackets''.

%======================================================================
\section{An ocean wave as a simple 5D example}
%======================================================================

Consider a concrete point on the shore. At a moment, say
\texttt{15:30:12.05}:
\begin{itemize}
  \item the spatial coordinates $(x,y,z)$ are fixed;
  \item the time $t$ is known (clock, date, experimental timestamp);
  \item but the water height depends on \emph{which phase of the wave}
        we are in: rising towards the crest, at the crest, going down,
        in the trough.
\end{itemize}

We can say:
\begin{itemize}
  \item $t = \texttt{15:30:12.05}$;
  \item $\phi \approx 0.8\pi$ --- the wave is almost at its maximum.
\end{itemize}

This is already an honest $5D$ description:
\begin{itemize}
  \item $(x,y,z)$ --- where on the shore;
  \item $t$ --- when;
  \item $\phi$ --- where within the current wave cycle.
\end{itemize}

For a physicist this is just the language of $\cos(\omega t + \phi_0)$.
For a non-specialist it is the intuitive sentence:
\begin{quote}
``Right now the wave is almost at the peak of its run.''
\end{quote}
The only difference is whether we formalise this as a separate
coordinate $\phi$ or leave it in words.

%======================================================================
\section{Project, life, economy as phase processes}
%======================================================================

For non-cyclic or slowly cyclic processes (large projects, a life,
economic cycles) linear time by itself does not say much. The sentence
``I am 35'' carries less information than ``I am at such-and-such
stage of life''.

One can introduce phase for a long process:
\begin{itemize}
  \item $\phi = 0$ --- project start;
  \item $\phi = \pi$ --- roughly mid-way;
  \item $\phi = 2\pi$ --- completion of one full cycle.
\end{itemize}

The same can be done for:
\begin{itemize}
  \item life path (childhood, youth, maturity, old age);
  \item a product or company (emergence, growth, saturation, decline);
  \item an economic cycle (expansion, overheating, crisis, recovery).
\end{itemize}

From the point of view of subjective experience we already live in the
coordinate ``phase'':
\begin{quote}
wakefulness/sleep, work/rest, energy rise/fatigue, ebb/flow of strength.
\end{quote}
In this sense the language $3D + t + \phi$ is closer to inner
experience than pure $3D + t$.

%======================================================================
\section{Why an explicit 5D language is useful for society}
%======================================================================

\subsection*{(i) An honest description of reality}

We already use a coordinate of state $\phi$ across all scales, from
neural rhythms and heart beats to economic and cultural oscillations.
Making it explicit simply means acknowledging what already exists in
our experience.

\subsection*{(ii) A common language for different fields}

\begin{itemize}
  \item a physicist speaks of phase of a wave or oscillator;
  \item a doctor --- of sleep phase or arrhythmia;
  \item an engineer --- of the phase of a device cycle;
  \item a psychologist --- of the phase of a crisis or remission;
  \item an economist --- of the phase of a market cycle.
\end{itemize}

In a $5D$ language these are all special cases of the same coordinate:
\begin{quote}
the state of a process along its course, $\phi$.
\end{quote}

\subsection*{(iii) A soft bridge between science and everyday life}

If we say: ``we live in 5D, where the fifth is not a magical dimension
but the phase of a process'', this does not sound like esoterica; it is
a careful clarification:
\begin{quote}
besides space and time we almost always need one more coordinate to
describe what is \emph{really} happening.
\end{quote}

\subsection*{(iv) A tool for thinking and practice}

Phase language allows more precise formulations:
\begin{itemize}
  \item instead of ``pure chaos'' --- ``subsystems are in different
        phases of their cycles'';
  \item instead of ``failure'' --- ``we entered the decline phase too
        early / too late relative to other elements of the system''.
\end{itemize}

This does not replace equations and models, but it makes it easier to
talk about complex processes in understandable terms.

%======================================================================
\section{Phase as a bridge to unified frameworks}
\label{sec:bridge}
%======================================================================

In a more technical context, introducing an explicit phase coordinate
$\phi$ opens the door to neat unified descriptions.

In particular, one can build a framework in which:
\begin{itemize}
  \item weak-field gravity (through the $g_{00}$ component);
  \item the electromagnetic field (through the potential $A_\mu$);
  \item phase clocks (Josephson oscillators, coherent condensates and
        other systems where we measure phase growth rate)
\end{itemize}
are described in a single geometric language. Technically this may look
like a five-dimensional construction with a compact phase dimension
$S^1$, where the coordinate $\phi$ plays the role of a \emph{phase
fibre} over ordinary spacetime.

One specific realisation of this approach is developed by the author in
the \emph{PhaseGeometry Phase-Fibre Foundations} series (in English):

\begin{itemize}
  \item \textbf{Foundations 0: Technical Passport} --- a compact
        technical summary of the framework, ontology and notation;
  \item \textbf{Foundations I: Conceptual Framework} --- conceptual
        picture: phase fibre, weak gravity, phase clocks, links to
        superconductivity and the Josephson effect;
  \item \textbf{Foundations II: Field Equations and Phase-Clock Configurations} ---
        derivation of the action and equations of motion, simple
        configurations of phase clocks in gravitational and rotating
        backgrounds;
  \item \textbf{Foundations III: Newtonian Halo Analogues from Phase Fields} ---
        Newtonian phase sector, phase defects and halo-like rotation
        curves;
  \item \textbf{Foundations IV: Newtonian Black-Hole Analogues from Phase-Fibre Defects} ---
        phase-fibre ``bubbles'' and Newtonian black-hole analogues.
\end{itemize}

All these works, as well as other related PhaseGeometry notes, are
collected in the \emph{PhaseGeometry Phase-Fibre Core Package v1.0} on
Zenodo (DOI \texttt{10.5281/zenodo.17783322}).

These notes form the technical layer beneath the intuitive ``5D
language'' discussed in the present popular-physics essay.

Such an approach:
\begin{itemize}
  \item does not require mysticism and does not modify the basic
        equations of general relativity or Maxwell theory;
  \item changes the \emph{language of description} and the emphasis:
        gravity, electromagnetism and phase clocks are treated
        together as parts of a single stage design;
  \item is convenient both for analysing fundamental effects
        (gravitational redshift of phase clocks, Sagnac-like phase
        shifts) and for applied questions in superconducting devices.
\end{itemize}

%======================================================================
\section{Discussion: phase as an honest fifth dimension}
\label{sec:discussion}
%======================================================================

In this essay we introduce the phase $\phi$ as a full, honest fifth coordinate
of state on the same footing as $x, y, z$ and $t$. The idea is not to add yet
another “exotic dimension” on top of familiar $3D + t$, but to make explicit
something we already use all the time when we talk about waves, cycles, stages
and phases of life or a project.

In essence, the main claim can be stated as follows:
we have long been living and thinking in a five–coordinate world $3D + t + \phi$.
Three coordinates describe “where”, time describes “when”, and the phase $\phi$
answers the question “at which point in the process we are right now”.
The four–dimensional description $3D + t$ is convenient and familiar, but it arose
historically as a shop–window simplification, in which the state coordinate was
taken out of sight and hidden inside special symbols and equations.

The text highlights two key observations. First, the phase coordinate is already
present de facto on all scales: from harmonic waves and pendulums to biological
rhythms, sleep, heartbeat, project phases, economic cycles and the stages of a
human life. All of this is naturally described in the language of “where we are
on the circle”, that is, in terms of a quantity equivalent to the phase $\phi$
on $S^1$. Second, this language is equally accessible to a non–physicist and to
a professional physicist: for one it is an “early/late stage”, for the other it
is the argument of a cosine or a point on a phase trajectory.

It is important to stress that such a 5D description requires no mysticism and
does not contradict existing physical theories. At the level of strict
formalisation, the phase is treated as an additional coordinate in a
phase–fibre picture: above ordinary space–time there hangs a compact phase
dimension $S^1$, and the state of the system at any moment is specified not
only by $(x, y, z, t)$, but also by its position on this phase circle. In this
sense the proposed picture is not an alternative to classical physics, but a
refinement of its language: we add an explicit fifth layer of description that
has long been hidden inside the formulas.

From a cultural point of view, this shift of language has several effects.
It brings the physical picture of the world closer to subjective experience:
people already think in terms of phases and stages, they simply rarely see this
framed as a coordinate. It provides a soft common language for different
domains — from neuroscience and medicine to economics and psychology — where
one and the same idea of “position on a cycle” keeps reappearing. And finally,
it creates a neat bridge to more technical frameworks such as PhaseGeometry
Phase–Fibre Foundations, where the same coordinate $\phi$ is used in a
fully geometric way to describe gravity, electromagnetism and phase clocks.

Put in the simplest possible way, the conclusion of the essay can be reduced
to one phrase:
we have always lived in 5D — in space, in time and in the phases of our
processes. Perhaps it is time to start speaking about this openly ---
both in human and in physical language.

\bigskip

\end{document}

\documentclass[11pt]{article}

\usepackage[utf8]{inputenc}
\usepackage[english]{babel}
\usepackage{amsmath,amssymb}
\usepackage{geometry}
\geometry{a4paper, margin=2.5cm}

\title{%
  {\large PhaseGeometry Z$_2$ Foundations I -- Technical Passport\\[2pt]}
  {\Large\boldmath Minimal Binary Phase Model for Dark Sector and Gravity\\[4pt]}
}

\author{Aleksey Turchanov}

\date{November 2025}

\begin{document}

\maketitle

\begin{center}
\scriptsize
This note is part of the PhaseGeometry Z$_2$ Core Package v2.5 (strict).\\
Licensed under Creative Commons Attribution 4.0 International (CC BY 4.0).\\
Zenodo record: DOI\,10.5281/zenodo.17807433.
\end{center}

\vspace{0.5em}

\begin{abstract}
This note defines the strict PhaseGeometry Z$_2$ core as the simplest possible
binary--phase engine for the dark sector and gravity. A single real scalar field
with a $\mathbb{Z}_2$--symmetric quartic potential is coupled minimally to
general relativity and interpreted as an order parameter whose homogeneous
broken phase mimics dark energy, while inhomogeneities and defects provide
dark--matter--like phenomenology and a medium for quantum branching.

We deliberately keep the framework as minimal and rigid as possible in order to
make its capabilities and limitations transparent: one can see exactly what this
strict Z$_2$ engine can reproduce (background cosmology, halo--like profiles,
decoherence toy models) and where it becomes physically doubtful. In this sense,
the present note plays the role of a \emph{Technical Passport} for the strict
PhaseGeometry Z$_2$ core used throughout the PhaseGeometry Z$_2$ Core Package
v2.5 (strict).
\end{abstract}


\vspace{1em}
\flushbottom

%=========================================================
\section{Introduction and purpose of the strict core}
\label{sec:intro}
%=========================================================

The standard flat $\Lambda$CDM model provides an excellent phenomenological
description of late--time cosmology, but it treats the dark sector as a
collection of independent components: a bare cosmological constant and a
pressureless cold dark matter fluid, with no internal structure or common
origin beyond gravity. The PhaseGeometry programme takes a different route and
asks whether the dark sector and parts of quantum phenomenology can be encoded
in the phase structure of a single underlying field.

In this note we define the \emph{strict} PhaseGeometry Z$_2$ core: the simplest
possible binary--phase engine that is still rich enough to talk about dark
energy, dark--matter--like haloes and a phase medium for branching. The
ingredients are deliberately minimal:
\begin{itemize}
  \item one real scalar field $\phi$ with a $\mathbb{Z}_2$--symmetric quartic
        potential $V(\phi)$;
  \item minimal coupling to general relativity with a fixed Newton constant
        $G_0$;
  \item an interpretation of the homogeneous broken phase as an effective vacuum
        energy and of inhomogeneous configurations and defects as providers of
        dark--matter--like gravity and a medium for decoherence and branching.
\end{itemize}
No extra fields, no modified gravity sector and no additional symmetry layers
are introduced at this level. The strict core is meant to be read as a
controlled toy model with a very small number of moving parts.

This model represents the \emph{strict Z$_2$ core} of PhaseGeometry — a minimal implementation where a single binary field accounts for the dark sector and emergent gravity. Its purpose is to serve as a \emph{rigid baseline} for testing which phenomena can be explained within a simple binary-phase picture, and where extensions become necessary.

The goal of this technical passport is twofold:
\begin{enumerate}
  \item \textbf{To define the minimal Z$_2$ framework precisely.} We collect in
        one place the action, phase structure, effective cosmological constant,
        defect sector and the basic dictionary that underlies all subsequent
        Z$_2$ notes in the PhaseGeometry series. This provides a clean, reusable
        reference for later work on background cosmology, haloes and quantum
        phenomenology.
  \item \textbf{To make the strengths and limitations of the strict core
        explicit.} By keeping the construction rigid and low--dimensional, we
        can see clearly what this binary phase engine can already reproduce
        (e.g.\ $\Lambda_{\rm eff}$ from a broken phase, halo--like profiles,
        simple decoherence scenarios) and where it runs into structural
        problems.
\end{enumerate}

A key part of this second goal is to acknowledge the \emph{structural demons}
that are generated precisely by the simplicity of the strict Z$_2$ setup:
domain wall networks and their cosmological energy density, tensions in halo
scaling and ``coldness'', the possibility of scalar hair and wall--like shells
around black holes, and single--scale constraints tying together cosmology and
galactic structure. The present note does not attempt to solve these problems;
instead, it records the minimal assumptions under which they appear and
organises the strict core as a well--defined baseline against which future
extensions can be tested.

Within the PhaseGeometry Z$_2$ Core Package v2.5 (strict), the role of this
Foundations~I note is to serve as a hub for the other components:
\begin{itemize}
  \item \emph{Foundations II -- Lambda from Broken Phase} makes precise how the
        homogeneous broken phase generates an effective cosmological constant
        $\Lambda_{\rm eff}$.
  \item \emph{Foundations III -- DarkMatterFromDefects} develops the defect
        sector and its dark--matter--like interpretation.
  \item \emph{Foundations IV--V} use the same phase medium to discuss observers,
        branching, decoherence and Quantum Darwinism.
  \item \emph{Phenomenology I--III} apply the strict core to concrete
        configurations: FRW background cosmology, halo and rotation--curve
        modelling, and static black holes with binary phase hair.
\end{itemize}
All of these notes assume the minimal structure defined here; any future,
non--strict extensions (for example, additional phase coordinates or more
general phase manifolds) are intentionally kept outside the scope of this
document.

%=========================================================
\section*{0. Purpose of this document}
%=========================================================

This document specifies a minimal technical framework in which:
\begin{itemize}
  \item a single underlying ``pre--geometric'' field carries a binary symmetry
        $+/-$ around a special point $0$;
  \item an order parameter for this symmetry has distinct phases:
    \begin{itemize}
      \item a symmetric phase,
      \item a symmetry--broken homogeneous phase (interpreted as dark energy /
            $\Lambda$),
      \item and a symmetry--broken inhomogeneous phase with defects (interpreted
            as dark matter);
    \end{itemize}
  \item gravity appears as the effective dynamics of this order parameter and
        the induced geometry.
\end{itemize}

The goal is not to provide a complete theory, but a stable ``passport'' of
assumptions and equations that further work (derivations, phenomenology, papers)
can consistently refer to.
%=========================================================
\section{Ontology: underlying field and emergent description}
%=========================================================

\subsection{Pre--geometric field}

We postulate the existence of a pre--geometric field (or a set of fields)
\[
  \Phi(x),
\]
which is not itself the spacetime metric and not ordinary matter. It is an
abstract carrier of possible geometries and physical regimes.

Conventional entities
\begin{itemize}
  \item the metric $g_{\mu\nu}$,
  \item Standard Model fields,
  \item effective fluids and matter components,
\end{itemize}
are interpreted as phases, regimes, or collective excitations of the underlying
field $\Phi$.

\subsection{Structure rather than a particle zoo}

At the fundamental level, the ontology is not a list of independent ``building
blocks'' (electrons, quarks, photons, etc.), but a structural principle: a
binary symmetry and its breaking. Objects and fields arise as stable or
metastable configurations of this structure.

\subsection{Vacuum and binary phase as a field}

In standard classical and quantum field theory the electromagnetic field (and
other matter fields) is simply postulated to exist as a continuous field on
space--time: even when $E=B=0$ one still assumes that a well-defined field
configuration is present at every point. The vacuum is not an empty container;
it is a particular \emph{state of a field}.

In the strict $\mathbb{Z}_2$ PhaseGeometry programme we make an analogous, but
even more economical move. Instead of postulating a whole collection of
independent fields, we postulate a single real scalar \emph{order parameter
field} $\phi(x)$ with a double-well, $\mathbb{Z}_2$-symmetric potential. The
mere assumption that the Universe carries a global binary structure
(``$+$''/``$-$'' branch, or $\phi\to-\phi$ symmetry) already means that space
is filled by a binary medium. One can think of a microscopic $\mathbb{Z}_2$
label $s(x)=\pm1$, with $\phi(x)$ playing the rôle of a coarse-grained,
continuum description of this underlying binary field.

From this point of view the ``vacuum'' is literally a binary phase background:
large regions in which $\phi$ sits near $+v$ or $-v$ provide an effective
vacuum energy density $V(\pm v)$, while kinks, domain walls and halo-like
profiles are just inhomogeneous configurations of the \emph{same} field. The
existence of a global $\mathbb{Z}_2$ label is therefore not an extra decoration
on top of ordinary fields; in the strict PhaseGeometry Z$_2$ framework it
\emph{is} the content of a minimal order parameter field $\phi(x)$ that
underlies the dark sector and branching structure.

In this sense all macroscopic objects --- matter distributions, measuring
devices, records and even observers themselves --- can be treated, in
principle, as metastable, redundantly encoded \emph{patterns} in the binary
phase medium. What we usually call ``particles'', ``bodies'' or ``apparatus
states'' are different long-lived configurations of $\phi(x)$ (often together
with, at the effective level, the familiar Standard--Model fields), sitting on
top of the same Z$_2$ phase background.


%=========================================================
\section{Binary symmetry and order parameter}
%=========================================================

The basic structural input is a \emph{binary} (``yes/no'', ``plus/minus'')
internal symmetry around a distinguished reference state.

We introduce a real scalar order parameter field
\[
  \phi(x)
\]
transforming as
\begin{equation}
  \phi \;\to\; -\phi.
\end{equation}
This is a $\mathbb{Z}_2$ symmetry.

The special point
\[
  \phi = 0
\]
represents the symmetric state, in which neither branch (``$+$'' nor ``$-$'')
has been selected.

%=========================================================
\section{Effective potential and phase structure}
%=========================================================

The minimal effective potential for the order parameter is taken to be
\begin{equation}
  V(\phi) = \alpha \,\phi^2 + \frac{\beta}{2} \,\phi^4,
  \qquad \beta > 0.
  \label{eq:Vphi}
\end{equation}

\subsection{Symmetric phase}

For
\[
  \alpha > 0,
\]
the potential is minimized at
\[
  \phi = 0.
\]

In this phase, the $\mathbb{Z}_2$ symmetry is unbroken. The system resides in
the ``zero'' configuration between the two branches. This regime is interpreted
as a pre--geometric or high--energy phase in which no classical spacetime
geometry has yet formed (or in which the order parameter does not yet carry a
macroscopic role).

\subsection{Symmetry--broken phase}

For
\[
  \alpha < 0,
\]
the potential develops two degenerate minima
\begin{equation}
  \phi = +v, 
  \qquad
  \phi = -v,
  \qquad
  v = \sqrt{-\alpha / \beta}.
  \label{eq:v-minima}
\end{equation}

The $\mathbb{Z}_2$ symmetry is spontaneously broken: the system must choose one
of the two branches,
\[
  \phi \approx +v \quad \text{or} \quad \phi \approx -v,
\]
leading to distinct macroscopic realizations of the same underlying symmetry.

%=========================================================
\section{Phases of the Universe in this framework}
%=========================================================

We distinguish three regimes of the order parameter:

\subsection{Symmetric (pre--geometric) regime}

\begin{itemize}
  \item $\phi \approx 0$,
  \item $\mathbb{Z}_2$ symmetry unbroken,
  \item no stable classical geometry in the usual sense, or geometry dominated
        by fluctuations.
\end{itemize}

This can be interpreted as a ``pre--geometric'' or ultra--high--energy regime.

\subsection{Homogeneous broken phase (dark energy / $\Lambda$)}

Consider a state in which, on cosmological scales,
\[
  \phi(x) \approx +v \quad \text{or} \quad \phi(x) \approx -v,
\]
with small gradients,
\[
  \partial_\mu \phi \approx 0.
\]

Then the energy density contributed by $\phi$ is approximately
\begin{equation}
  \rho_{\text{vac}} \sim V(v).
\end{equation}

This behaves as an effective cosmological constant $\Lambda$ and is interpreted
as \emph{dark energy} (DE).

Interpretive note: in this picture, dark energy is the energetic cost of the
Universe having selected a branch of the binary symmetry, rather than remaining
at $\phi = 0$.

\subsection{Inhomogeneous broken phase (dark matter / defects)}

On smaller (galactic and cluster) scales, the order parameter can be
inhomogeneous:
\begin{itemize}
  \item $|\phi| \approx v$ typically, but the sign and spatial configuration may
        vary;
  \item there can be domains with $\phi \approx +v$ and others with
        $\phi \approx -v$;
  \item between domains, $\phi(x)$ passes through values near $0$, forming
        domain walls and related defects.
\end{itemize}

In such regions
\[
  \partial_\mu \phi \neq 0,
\]
and the gradient term contributes additional local energy density. This energy
\begin{itemize}
  \item does not couple directly to light (is effectively ``dark''),
  \item interacts gravitationally,
  \item can be distributed in halos and large--scale structures.
\end{itemize}

Interpretive note: this contribution is interpreted as \emph{dark matter}: not
as a new species of particle, but as the energy stored in boundaries, defects
and inhomogeneities of the symmetry--broken phase of the order parameter.

%=========================================================
\section{Gravity as dynamics of the broken phase}
%=========================================================

We now couple the order parameter to an emergent spacetime metric $g_{\mu\nu}$
via an effective action
\begin{equation}
  S[g,\phi] = \int d^4x \,\sqrt{-g}
  \left[
    \frac{1}{2} g^{\mu\nu}\partial_\mu \phi \partial_\nu \phi
    - V(\phi)
    - \frac{1}{16\pi G_0} R
  \right],
  \label{eq:action-real}
\end{equation}
where
\begin{itemize}
  \item $g_{\mu\nu}$ is the effective spacetime metric,
  \item $R$ is the Ricci scalar,
  \item $G_0$ is a bare gravitational coupling (which may differ from the
        observed $G$).
\end{itemize}

Variation with respect to $g_{\mu\nu}$ yields
\begin{equation}
  G_{\mu\nu} = 8\pi G_0 
  \left( 
    T_{\mu\nu}^{(\phi)} + T_{\mu\nu}^{\text{(ordinary)}}
  \right),
  \label{eq:Einstein}
\end{equation}
where $G_{\mu\nu}$ is the Einstein tensor and $T_{\mu\nu}^{(\phi)}$ is the
stress--energy tensor of the field $\phi$. It includes
\begin{itemize}
  \item the homogeneous contribution $V(v)$ (dark energy / $\Lambda$),
  \item local contributions from gradients and inhomogeneities
        (dark--matter--like mass).
\end{itemize}

Interpretive note: in this framework, gravity is not a separate ``force'' added
by hand, but the effective dynamics and elasticity of a symmetry--broken phase,
expressed geometrically by $g_{\mu\nu}$ and $R$.

%=========================================================
\section{Extension: complex order parameter and supercurrent regime}
%=========================================================

For phenomena involving currents, vortices and possible ``gravitational
superconductivity'', it is useful to consider a complex order parameter
\[
  \Psi(x) = \rho(x)\,e^{i\theta(x)}.
\]

The binary symmetry can then be realized as
\begin{equation}
  \theta \;\to\; \theta + \pi,
\end{equation}
corresponding to phase / anti--phase. In addition, a continuous $U(1)$ phase
symmetry can be allowed to support supercurrents.

A minimal Ginzburg--Landau--type effective action in curved spacetime is
\begin{equation}
  S_{\text{eff}}[\Psi,g] =
  \int d^4x\,\sqrt{-g}\,
  \left[
    \frac{1}{2m^\*}\, g^{\mu\nu} (\nabla_\mu \Psi)^\* (\nabla_\nu \Psi)
    - \alpha\,|\Psi|^2 - \frac{\beta}{2}\,|\Psi|^4
    - \frac{1}{16\pi G_0}\,R
  \right],
  \label{eq:action-GL}
\end{equation}
with effective parameters $m^\*$, $\alpha$, $\beta$.

In this description:
\begin{itemize}
  \item a homogeneous state with $|\Psi| = \text{const}$ contributes to the
        vacuum energy (dark energy);
  \item vortices and supercurrents in the phase $\theta(x)$ generate additional
        gravitational effects that can phenomenologically resemble galactic
        halos and flat rotation curves (dark--matter--like behaviour).
\end{itemize}

This complex order parameter can be understood as a coarse--grained description
of more microscopic structures of the underlying field $\Phi$ (for example,
configurations related by the binary symmetry), but the present document keeps
this level abstract on purpose. For the strict core, the real scalar version
\eqref{eq:action-real} is sufficient; the complex extension is recorded here as
a natural route for later, non--strict generalisations.

%=========================================================
\section{Role of the strict Z$_2$ core}
\label{sec:role}
%=========================================================

This minimal model serves three concrete purposes within the PhaseGeometry programme:

\begin{enumerate}
  \item \textbf{Control baseline:} It provides a well--defined, low--dimensional reference against which the explanatory power of a simple binary--phase engine can be measured. Phenomena that can be reproduced within this strict core (e.g.\ $\Lambda_{\rm eff}$, halo--like profiles, phase--medium decoherence) do not require more elaborate constructions.
  
  \item \textbf{Structural stress test:} The ``structural demons'' mentioned in Section~\ref{sec:intro} (domain--wall networks, halo--scaling tensions, black--hole hair constraints) arise directly from the model's minimality. They serve as clear signposts for where extensions (additional fields, symmetries, or energy scales) become necessary.
  
  \item \textbf{Technical passport:} It supplies a single, fixed set of equations and conventions that all subsequent notes in the PhaseGeometry Z$_2$ Core Package (Foundations II--V, Phenomenology I--III) can refer to, ensuring consistency across different applications.
\end{enumerate}

The strict Z$_2$ core is therefore not intended as a complete physical theory, but as a \emph{deliberately rigid scaffold} that makes both its successes and its limitations transparent. Future non--strict extensions will be judged by their ability to preserve the core's explanatory successes while resolving its structural problems.

%=========================================================
\section{Interpretive summary (for internal use)}
%=========================================================

For later reference, the conceptual core of this framework can be stated as
follows:

\begin{enumerate}
  \item At the foundation, there is not a zoo of independent entities, but a
        \emph{binary distinction} ``plus/minus'' around a special reference
        state ``zero''.
  \item The real order parameter $\phi$ (or complex $\Psi$) tracks how this
        binary symmetry is broken in spacetime.
  \item A homogeneous symmetry--broken state yields an effective cosmological
        constant (dark energy).
  \item Defects, domain walls, vortices and inhomogeneities in the broken phase
        contribute as dark--matter--like mass.
  \item Gravity is the effective geometric expression of the dynamics and
        elasticity of this broken phase, encoded in the metric $g_{\mu\nu}$ and
        curvature $R$.
  \item This technical passport, together with the other documents of the
        PhaseGeometry Z$_2$ Core Package, forms a self--consistent reference
        system for exploring the consequences of a minimal binary--phase
        description of the dark sector and gravity.
\end{enumerate}

This ``technical passport'' is intended as a stable reference point: future
calculations, phenomenological models and papers should explicitly identify
which assumptions here are being used or modified.

%=========================================================
\section*{Next steps and Core Package structure}
%=========================================================

This note is part of the PhaseGeometry Z$_2$ Core Package v2.5 (strict). Together
with the other notes listed below it defines a minimal binary-phase framework
for dark energy, dark-matter–like haloes, black holes and quantum branching in
a single Z$_2$ medium. The Core Package is archived as a bundled record under
Zenodo DOI\,10.5281/zenodo.17807433.

The current structure of the Core Package is:

\begin{itemize}
  \item \textbf{Foundations I -- Technical Passport: Minimal Binary Phase Model for Dark Sector and Gravity}\\
        Defines the strict Z$_2$ action $S[g,\phi]$, potential $V(\phi)$, phase
        structure and basic ontology of the binary phase medium.
  \item \textbf{Foundations II -- Lambda from Broken Phase}\\
        Derives how the homogeneous broken phase $\phi\simeq\pm v$ acts as an
        effective cosmological constant $\Lambda_{\rm eff}=8\pi G_0 V(v)$.
  \item \textbf{Foundations III -- Dark Matter from Defects}\\
        Develops the defect/inhomogeneous sector and defines the excess energy
        density $\rho_\phi(r)$ as a dark-matter–like component.
  \item \textbf{Foundations IV -- Phase medium, observer and branching histories}\\
        Interprets the Z$_2$ field as a phase medium hosting observers and
        classical branches in an Everett–Zurek picture.
  \item \textbf{Foundations V -- Decoherence and Quantum Darwinism in a binary phase medium}\\
        Implements decoherence and Quantum Darwinism explicitly in finite Z$_2$
        chains, with redundant records stored in environmental fragments.
  \item \textbf{Phenomenology I -- Background cosmology and the age of the Universe}\\
        Confronts the homogeneous broken phase with FRW background evolution,
        distance–redshift relations and cosmic age constraints.
  \item \textbf{Phenomenology II -- Dark-matter–like haloes from defects}\\
        Studies static halo profiles supported by Z$_2$ defects and compares
        them with rotation-curve phenomenology.
  \item \textbf{Phenomenology III -- Static black holes with binary phase hair}\\
        Embeds black holes into the same binary phase medium and explores weak
        scalar hair and defect-like shells in the strong-gravity regime.
\end{itemize}

Across these layers, the same strict Z$_2$ order parameter $\phi$ underlies:
(i) the effective cosmological constant from the homogeneous broken phase,
(ii) dark-matter–like haloes from defects and inhomogeneities,
(iii) phase-dressed black holes, and
(iv) quantum branching and classical records realised as patterns in the binary medium.

The present note should be read as one component of this unified picture. It is
designed to be technically self-contained, but its full meaning emerges when
combined with the other items in the PhaseGeometry Z$_2$ Core Package v2.5 (strict).



\end{document}
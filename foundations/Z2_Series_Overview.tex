\documentclass[12pt]{article}

\usepackage[utf8]{inputenc}
\usepackage[T1]{fontenc}
\usepackage[english]{babel}

\usepackage{amsmath,amssymb}
\usepackage{geometry}
\geometry{a4paper, margin=2.5cm}

\title{%
  {\large PhaseGeometry Z$_2$ Series Overview\\[2pt]}
  {\Large\boldmath Minimal Binary Phase Framework for Dark Sector and Gravity\\[4pt]}
}

\author{Aleksey Turchanov}

\date{December 2025}

\begin{document}

\maketitle

\begin{center}
\scriptsize
This note is part of the PhaseGeometry Z$_2$ Core Package v2.5 (strict).\\
Licensed under Creative Commons Attribution 4.0 International (CC BY 4.0).\\
Zenodo record: DOI\,10.5281/zenodo.17807433.
\end{center}

\vspace{0.8em}

\begin{abstract}
We outline a deliberately minimal theoretical framework in which a single real scalar field with
a binary $\mathbb{Z}_2$ symmetry and a quartic double-well potential accounts, in principle,
for dark energy, dark-matter--like haloes and gravitational dynamics. The aim of the
PhaseGeometry Z$_2$ series is to explore how far one can go with such a minimal binary phase
and to make explicit the boundaries of its applicability. Instead of introducing additional
fields, extra dimensions or modified gravity, we show that even this simple structure can
reproduce key observational features: late-time accelerated expansion, flat galactic rotation
curves and characteristic properties of black holes. The series is organised into foundational
and phenomenological notes that move from the strict core action and phase structure to
cosmology, haloes, black holes and quantum aspects of the binary medium.
\end{abstract}

%======================================================================
\section{Introduction}
%======================================================================

The standard $\Lambda$CDM cosmological model describes the late-time Universe in
terms of a cosmological constant $\Lambda$ and a cold dark matter (CDM) fluid.
Phenomenologically it works remarkably well, yet it leaves open both the nature
of these components and any structural link between them. Dark energy and dark
matter appear as two independent substances whose only guaranteed coupling is
gravitational.

In the PhaseGeometry Z$_2$ programme we pursue the opposite strategy. Rather
than adding new sectors, higher dimensions or nontrivial geometries, we ask how
far one can go with a single minimal ingredient: a binary phase that underlies
the entire dark sector and its coupling to gravity.

The core object is a real scalar order parameter $\phi(x)$ with quartic
potential
\begin{equation}
  V(\phi) = \alpha \phi^2 + \frac{\beta}{2}\phi^4,
  \qquad \beta>0,
\end{equation}
and a discrete symmetry
\begin{equation}
  \phi \to -\phi.
\end{equation}
For $\alpha<0$ the potential develops two degenerate minima
\begin{equation}
  \phi = \pm v, \qquad v = \sqrt{-\alpha/\beta},
\end{equation}
corresponding to a broken binary phase. We interpret:
\begin{itemize}
  \item the homogeneous broken phase $\phi \approx \pm v$ as the source of an
        effective cosmological constant
        $\Lambda_{\rm eff} = 8\pi G_0 V(v)$ (dark energy);
  \item inhomogeneities and defects in this phase (kinks, domain walls and
        related configurations) as sources of additional gravitating energy
        density that behaves like dark matter in galactic haloes.
\end{itemize}

In this view the dark sector is not a collection of unrelated ingredients but a
single binary medium with different regimes. The PhaseGeometry Z$_2$ series of
notes develops this idea step by step, from the minimal technical scaffold to
galactic and black-hole phenomenology.

%======================================================================
\section{Minimal binary phase model}
%======================================================================

At the level of the strict core, the PhaseGeometry Z$_2$ framework consists of:
\begin{itemize}
  \item a single real scalar field $\phi(x)$ with $\mathbb{Z}_2$ symmetry and
        quartic potential $V(\phi)$;
  \item a minimally coupled metric $g_{\mu\nu}$ with Einstein--Hilbert action
        and no bare cosmological constant term;
  \item an interpretation of the broken phase as a physical medium filling
        space, rather than as an abstract field living on empty spacetime.
\end{itemize}

The homogeneous broken phase $\phi \simeq \pm v$ acts as a vacuum with energy
density $\rho_{\rm vac} = V(v)$ and equation of state $w=-1$, leading to an
effective cosmological constant
\begin{equation}
  \Lambda_{\rm eff} = 8\pi G_0 \rho_{\rm vac} = 8\pi G_0 V(v).
\end{equation}
Inhomogeneous configurations of the same field carry an excess energy density
$\rho_\phi(r)$ on top of this vacuum background, and under suitable conditions
can reproduce dark-matter–like halo profiles with
$\rho_\phi(r)\propto r^{-2}$ and flat rotation curves.

A key feature of the framework is minimalism: we do not introduce new fields,
modified gravity or extra dimensions. All dark-sector phenomenology arises
from the structure and dynamics of a single binary phase.


%======================================================================
\section{Structure of the PhaseGeometry Z$_2$ series}
%======================================================================

The Z$_2$ series is organised into foundational notes (which define the strict
core and its interpretations) and phenomenological notes (which confront it
with observations and specific gravitational configurations).

\subsection*{Foundations 0: Ontological Minimalism and the Binary Phase Medium}

\emph{Ontological and philosophical prelude to the strict Z$_2$ core.}  
Explains the shift from the classical ``many fields on an empty stage''
paradigm to a picture in which a single binary phase medium underlies vacuum,
dark energy, dark-matter--like haloes and gravity. Introduces the language of
ontological minimalism and structural realism that guides Foundations~I--III
and the phenomenological notes.

\subsection*{Foundations I: Technical Passport}

\emph{Minimal Z$_2$ framework for dark sector and gravity.}  
Defines the strict Z$_2$ action $S[g,\phi]$, potential $V(\phi)$, phase
structure and interpretational scheme. This note sets the ``rules of the
game'' for all subsequent work.

\subsection*{Foundations II: Lambda from Broken Phase}

\emph{Effective cosmological constant from the homogeneous broken phase.}  
Shows explicitly how the configuration $\phi=\pm v$ leads to
$\Lambda_{\rm eff}$ without introducing a bare cosmological constant in the
gravitational sector.

\subsection*{Foundations III: Dark Matter from Defects}

\emph{Dark matter as energy inhomogeneities of the binary phase.}  
Derives defect-supported density profiles $\rho_\phi(r)\propto r^{-2}$ and
approximately flat rotation curves from static inhomogeneous solutions of the
same scalar field.

\subsection*{Phenomenology I: Background Cosmology and the Age of the Universe}

\emph{Comparison with the observed expansion history.}  
Demonstrates numerically that the homogeneous background evolution of the
PhaseGeometry scalar closely tracks that of flat $\Lambda$CDM, with a cosmic
age compatible with Planck data for suitable parameter choices.


\subsection*{Phenomenology II: Dark-Matter--Like Haloes from Defects}

\emph{Phenomenology of galactic haloes.}  
Constructs model rotation curves based on scalar profiles, explores the
parameter dependence of halo scales and compares the resulting behaviour with
observed galactic data (e.g.\ SPARC-type samples).

\subsection*{Phenomenology III: Static Black Holes with Binary Phase Hair}

\emph{Black holes in the binary medium.}  
Investigates static black-hole configurations in the presence of the Z$_2$
medium, including the possibility of phase hair and defect-like shells near
the horizon.

Foundations IV--V (phase medium, observer and branching histories; decoherence
and Quantum Darwinism in a binary medium) extend the same Z$_2$ structure to
quantum aspects of measurement and branching, but they lie beyond the present
overview and are described in their own dedicated notes.

%======================================================================
\section{Key ideas of the programme}
%======================================================================

The PhaseGeometry Z$_2$ series is built around a small set of recurring ideas:

\begin{itemize}
  \item \textbf{Unified origin of the dark sector.} Dark energy and
        dark-matter--like haloes are different regimes of a single order
        parameter $\phi(x)$, not independent substances.
  \item \textbf{Binary phase as a physical medium.} Space is filled with the
        field $\phi$ even in ``vacuum''; the vacuum corresponds to the broken
        phase $\phi \approx \pm v$.
  \item \textbf{Minimalism.} No extra fields, modified gravity or additional
        dimensions are introduced. The only new ingredient beyond standard GR
        with ordinary matter is the binary phase itself.
  \item \textbf{Defects as dark matter.} Kinks, domain walls and related
        inhomogeneities in $\phi$ generate an effective density
        $\rho_\phi(r)\propto r^{-2}$ and flat rotation curves over a suitable
        radial range, providing a scalar-field analogue of CDM haloes.
  \item \textbf{Consistency across scales.} The same parameters
        $(\alpha,\beta,G_0)$ control the homogeneous cosmology, galactic
        haloes and black-hole configurations, tying together phenomena usually
        treated as independent.
\end{itemize}

%======================================================================
\section{Goals and structural tests}
%======================================================================

Conceptually, the goals of the Z$_2$ series are:

\begin{itemize}
  \item to demonstrate the possibility of a unified description of the dark
        sector within a minimal Z$_2$ model;
  \item to confront this description quantitatively with observational data:
        age of the Universe, expansion history, rotation curves, lensing
        profiles and related probes;
  \item to identify the structural constraints ``demons'' of the model:
        cosmological issues with domain walls, halo scaling relations,
        stability conditions for static solutions and compatibility with CMB
        and large-scale structure;
  \item to provide a technically consistent basis for future extensions
        (complex fields, phase coherence, quantum aspects) that remain
        anchored in the same binary phase language;
  \item to isolate falsifiable predictions and potential no-go results that
        either support or rule out the minimal Z$_2$ picture in specific
        astrophysical and cosmological contexts.
\end{itemize}

In this sense, the PhaseGeometry Z$_2$ series functions simultaneously as a
technical passport, a phenomenological toolkit and a map of limitations for a
minimalist approach to the dark sector.

%======================================================================
\section{Summary and outlook}
%======================================================================

The PhaseGeometry Z$_2$ series presents a self-contained attempt to describe
dark energy, dark-matter--like haloes and aspects of gravity within a single
minimal field with binary symmetry. We start from a simple real scalar with a
double-well potential, show how its homogeneous phase reproduces an effective
$\Lambda_{\rm eff}$, and how its inhomogeneities mimic dark-matter haloes in
galaxies, and then extend the framework to black holes and quantum aspects of
the phase medium.

The main lesson is twofold. On the one hand, such a simple scaffold can cover
an unexpectedly wide range of phenomena. On the other hand, its structural
boundaries are sharply defined: domain-wall cosmology, halo scaling and
parameter constraints all act as stringent tests. These limits are not a
failure of the framework but a roadmap: they signal where the strict Z$_2$
core must evolve, where new concepts should be introduced, and how future
extensions may connect more tightly to conventional classical physics. Future
work will clarify whether the strict Z$_2$ core survives these tests
quantitatively, or whether it must be generalised while retaining the same
phase-based intuition. In its present form, the PhaseGeometry Z$_2$ Core
Package v2.5 (strict) is deliberately confined to classical dark-sector and
gravitational phenomenology, with the quantum and observer-related layers
outlined as an explicit roadmap for future work rather than completed parts
of the series.

The series is intended for researchers in theoretical cosmology, gravity and
dark-sector physics, as well as for readers interested in minimalist approaches
to the fundamental problems of contemporary astrophysics.

\end{document}

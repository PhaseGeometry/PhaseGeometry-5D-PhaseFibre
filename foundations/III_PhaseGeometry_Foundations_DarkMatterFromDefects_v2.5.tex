\documentclass[11pt]{article}

\usepackage[utf8]{inputenc}
\usepackage[english]{babel}
\usepackage{amsmath,amssymb}
\usepackage{geometry}
\usepackage{hyperref}
\geometry{a4paper, margin=2.5cm}

\title{%
  {\large PhaseGeometry Z$_2$ Foundations III -- Dark Matter from Defects\\[2pt]}
  {\Large\boldmath Dark-Matter-Like Halos from Inhomogeneous Binary Phase\\[4pt]}
}

\author{Aleksey Turchanov}

\date{November 2025}

\begin{document}

\maketitle

\begin{center}
\scriptsize
This note is part of the PhaseGeometry Z$_2$ Core Package v2.5 (strict).\\
Licensed under Creative Commons Attribution 4.0 International (CC BY 4.0).\\
Zenodo record: DOI\,10.5281/zenodo.17807433.
\end{center}

\vspace{0.5em}

\begin{abstract}
Within the minimal PhaseGeometry Z$_2$ framework defined in Foundations I (Technical Passport), we show how inhomogeneous configurations of the same order parameter that yields an effective cosmological constant $\Lambda_{\rm eff}$ (Foundations II) can also produce dark-matter-like phenomenology. Working with static, spherically symmetric defect profiles $\phi(r)$ that interpolate between the broken-phase minima, we derive the defect energy density $\rho_\phi(r) = \frac{1}{2}\phi'(r)^2 + V(\phi(r)) - V(v)$. Linearising around the minimum $v$ for $r \gg r_{\rm core}$ yields $\rho_\phi(r) \approx A^2 m_\phi^2/r^2$ in an intermediate radial range, where $m_\phi^2 = 2|\alpha|$. This $1/r^2$ scaling produces an enclosed mass $M_\phi(r) \propto r$ and consequently an approximately flat rotation curve $v_c(r) \approx {\rm const}$ over the same range. The mechanism demonstrates how a single binary-phase field can, in principle, account for both dark energy (via its homogeneous vacuum) and dark-matter-like halos (via its inhomogeneous defects), without introducing new fields beyond the strict Z$_2$ core.
\end{abstract}

%===========================================================
\section{Introduction: From $\Lambda_{\rm eff}$ to DM-like defects}
%===========================================================

This note is the third in the PhaseGeometry Z$_2$ Foundations series. It builds directly on:
\begin{itemize}
  \item \textbf{Foundations I (Technical Passport)}: Minimal Z$_2$ framework with one real scalar $\phi$, quartic potential $V(\phi)=\alpha\phi^2+\frac{\beta}{2}\phi^4$, and minimal coupling to gravity.
  \item \textbf{Foundations II (Lambda from Broken Phase)}: The homogeneous broken phase $\phi=\pm v$ yields an effective cosmological constant $\Lambda_{\rm eff}=8\pi G_0 V(v)$.
\end{itemize}

Here we ask: What happens when the broken phase is \emph{not} homogeneous? If $\phi(x)$ stays in the broken branch asymptotically ($\phi\to +v$ as $r\to\infty$) but develops spatial inhomogeneities---domain walls, kinks, or more general defects---these configurations carry additional energy density beyond the constant $V(v)$. We show that this excess energy behaves gravitationally like a dark-matter component, and under simple analytic approximations produces halo-like profiles with flat rotation curves.

The analysis stays strictly within the passport action $S[g,\phi]$; no new fields, parameters, or couplings are introduced. The goal is to see how much dark-matter-like phenomenology can be extracted from the minimal binary-phase engine before hitting its structural limitations.

%===========================================================
\section{Setup: Minimal framework and defect configurations}
%===========================================================

\subsection{Recap of the passport action and homogeneous phase}

The action is
\begin{equation}
  S[g,\phi] = \int d^4x\sqrt{-g}\Big[
    \frac{1}{2}g^{\mu\nu}\partial_\mu\phi\,\partial_\nu\phi
    - V(\phi)
    - \frac{1}{16\pi G_0}R
  \Big],
  \label{eq:action}
\end{equation}
with
\begin{equation}
  V(\phi) = \alpha\phi^2 + \frac{\beta}{2}\phi^4,\quad \beta>0.
  \label{eq:V}
\end{equation}
For $\alpha<0$ the potential has two degenerate minima
\begin{equation}
  v = \sqrt{-\alpha/\beta}, \qquad \phi = \pm v,
  \label{eq:minima}
\end{equation}
with
\begin{equation}
  V(v) = \alpha v^2 + \frac{\beta}{2}v^4 = -\frac{\alpha^2}{2\beta}.
  \label{eq:Vv}
\end{equation}

The stress-energy tensor of $\phi$ is
\begin{equation}
  T^{(\phi)}_{\mu\nu}
  = \partial_\mu\phi\,\partial_\nu\phi
    - g_{\mu\nu}\Big[
        \frac{1}{2}g^{\alpha\beta}\partial_\alpha\phi\,\partial_\beta\phi
        + V(\phi)
      \Big].
  \label{eq:Tphi}
\end{equation}

For the homogeneous vacuum $\phi_{\rm vac}=\pm v$, $\partial_\mu\phi_{\rm vac}=0$,
\begin{equation}
  T^{(\phi)}_{\mu\nu}[\phi_{\rm vac}] = -V(v)g_{\mu\nu},
\end{equation}
which acts as an effective cosmological constant
\begin{equation}
  \Lambda_{\rm eff} = 8\pi G_0 V(v).
  \label{eq:Lambda-eff}
\end{equation}

\subsection{Inhomogeneous broken phase and defect energy density}

In the strict Z$_2$ version of PhaseGeometry we treat such inhomogeneous
broken–phase configurations as structured regions of the same binary
order–parameter field that realises the homogeneous vacuum. Large regions with
$\phi \simeq \pm v$ form a binary background with vacuum energy density
$V(\pm v)$, while halo–like solutions correspond to static defects in which
$\phi(x)$ departs from $v$ in a localized region before returning to it at
large radii. In this note all dark–matter–like contributions will therefore be
attributed to the defect part of $\phi(x)$, rather than to an additional
dark–matter fluid.

Now consider a configuration where $\phi(x)$ is not constant but approaches the broken minimum at spatial infinity:
\begin{equation}
  \phi(x) \to +v \quad \text{as} \quad r\to\infty.
  \label{eq:bc-infinity}
\end{equation}
Such a configuration could describe, for example, a spherical domain wall or kink where $\phi$ interpolates from $-v$ at the origin to $+v$ at infinity.

It is convenient to split $T^{(\phi)}_{\mu\nu}$ into a vacuum part and a defect part:
\begin{equation}
  T^{(\phi)}_{\mu\nu}
  = T^{\rm(vac)}_{\mu\nu} + \tau^{(\phi)}_{\mu\nu},
  \qquad
  T^{\rm(vac)}_{\mu\nu} \equiv -V(v)g_{\mu\nu},
\end{equation}
so that
\begin{equation}
  \tau^{(\phi)}_{\mu\nu}
  = \partial_\mu\phi\,\partial_\nu\phi
    - g_{\mu\nu}\Big[
        \frac{1}{2}g^{\alpha\beta}\partial_\alpha\phi\,\partial_\beta\phi
        + V(\phi) - V(v)
      \Big].
  \label{eq:tau-def}
\end{equation}
The Einstein equations then become
\begin{equation}
  G_{\mu\nu} + \Lambda_{\rm eff}g_{\mu\nu}
  = 8\pi G_0\Big(
    T^{\rm(ordinary)}_{\mu\nu} + \tau^{(\phi)}_{\mu\nu}
  \Big),
  \label{eq:Einstein-split}
\end{equation}
where $\Lambda_{\rm eff}$ is given by \eqref{eq:Lambda-eff}. The tensor $\tau^{(\phi)}_{\mu\nu}$ encodes the additional energy-momentum due to inhomogeneities; we will interpret it as a dark-matter-like contribution.

For a static configuration in the weak-field limit, the energy density measured by a comoving observer is
\begin{equation}
  T^{(\phi)0}{}_{0} \simeq \frac{1}{2}(\nabla\phi)^2 + V(\phi).
\end{equation}
We define the \emph{defect energy density} as the excess over the homogeneous vacuum:
\begin{equation}
  \rho_\phi(r)
  \equiv T^{(\phi)0}{}_{0}[\phi(r)] - T^{(\phi)0}{}_{0}[\phi_{\rm vac}]
  \simeq \frac{1}{2}\big(\partial_r\phi\big)^2 + V\big(\phi(r)\big) - V(v).
  \label{eq:rho-phi-def}
\end{equation}
By construction, $\rho_\phi(r)\to0$ as $r\to\infty$ if $\phi(r)\to v$ and $\phi'(r)\to0$.

%===========================================================
\section{Static spherical defect: Equation of motion and linearisation}
%===========================================================

\subsection{Full non-linear equation}

For a static, spherically symmetric configuration $\phi(r)$ in a weakly curved background, the leading-order equation of motion is the flat-space Klein–Gordon equation:
\begin{equation}
  \phi''(r) + \frac{2}{r}\phi'(r) = \frac{dV}{d\phi}
  = 2\alpha\phi + 2\beta\phi^3.
  \label{eq:EOM-full}
\end{equation}
We impose the boundary conditions
\begin{equation}
  \phi(r) \to +v \quad (r\to\infty), \qquad
  \phi(0) = \phi_c \quad (\text{e.g. } \phi_c \approx -v \text{ for a kink}).
  \label{eq:BCs}
\end{equation}

Equation \eqref{eq:EOM-full} with these boundary conditions describes a non-linear defect (domain wall) with a characteristic core radius $r_{\rm core}$ where $\phi$ crosses through zero. An exact analytic solution is not available in closed form, but we can study the behaviour outside the core by linearising around the minimum.

\subsection{Linearisation around $\phi=v$}

For $r \gg r_{\rm core}$, assume $\phi(r)$ is close to the asymptotic value $v$:
\begin{equation}
  \phi(r) = v + \delta\phi(r), \qquad |\delta\phi(r)| \ll v.
  \label{eq:linear-ansatz}
\end{equation}

Expand the potential to quadratic order:
\begin{align}
  V(\phi) &= V(v) + V'(v)\delta\phi + \frac{1}{2}V''(v)\delta\phi^2 + \mathcal{O}(\delta\phi^3) \\
          &= V(v) + \frac{1}{2}m_\phi^2\,\delta\phi^2 + \mathcal{O}(\delta\phi^3),
\end{align}
since $V'(v)=0$ at the minimum, and
\begin{equation}
  m_\phi^2 \equiv V''(v) = 2\alpha + 6\beta v^2.
\end{equation}
Using $v^2 = -\alpha/\beta$ from \eqref{eq:minima},
\begin{equation}
  m_\phi^2 = 2\alpha + 6\beta\left(-\frac{\alpha}{\beta}\right)
           = 2\alpha - 6\alpha = -4\alpha.
\end{equation}
With $\alpha<0$, set $|\alpha| = -\alpha$, then
\begin{equation}
  m_\phi^2 = 4|\alpha| \equiv (2m)^2, \quad \text{where } m \equiv \sqrt{|\alpha|}.
  \label{eq:mphi}
\end{equation}
Thus $m_\phi$ is the mass of small fluctuations around the broken minimum.

The right-hand side of \eqref{eq:EOM-full} becomes
\begin{align}
  \frac{dV}{d\phi}
  &= 2\alpha\phi + 2\beta\phi^3 \\
  &= 2\alpha(v+\delta\phi) + 2\beta(v^3 + 3v^2\delta\phi + 3v\delta\phi^2 + \delta\phi^3) \\
  &= \underbrace{(2\alpha v + 2\beta v^3)}_{=0} 
     + \underbrace{(2\alpha + 6\beta v^2)}_{=m_\phi^2}\delta\phi
     + \mathcal{O}(\delta\phi^2) \\
  &= m_\phi^2\,\delta\phi + \mathcal{O}(\delta\phi^2).
\end{align}

Dropping terms of order $\delta\phi^2$ and higher, the linearised equation of motion is
\begin{equation}
  \delta\phi''(r) + \frac{2}{r}\delta\phi'(r) - m_\phi^2\,\delta\phi(r) = 0.
  \label{eq:EOM-linear}
\end{equation}

\subsection{Solution of the linear equation}

Equation \eqref{eq:EOM-linear} is the modified spherical Bessel equation of order zero. Its general solution is
\begin{equation}
  \delta\phi(r) = C_1\frac{e^{-m_\phi r}}{r} + C_2\frac{e^{+m_\phi r}}{r}.
\end{equation}
Regularity at infinity requires $C_2=0$. Hence
\begin{equation}
  \delta\phi(r) = -\,\frac{A}{r}\,e^{-m_\phi r},
  \label{eq:delta-phi-sol}
\end{equation}
where we have written the constant as $-A$ for later convenience. The amplitude $A$ is determined by matching to the non-linear core solution at $r\sim r_{\rm core}$.

From \eqref{eq:delta-phi-sol},
\begin{equation}
  \delta\phi'(r) = -A\frac{e^{-m_\phi r}}{r}\Big(m_\phi + \frac{1}{r}\Big).
  \label{eq:delta-phi-prime}
\end{equation}

%===========================================================
\section{Defect energy density in the linear regime}
%===========================================================

\subsection{Expression for $\rho_\phi(r)$}

Insert the linearised expressions into the definition \eqref{eq:rho-phi-def}. First compute the gradient term:
\begin{align}
  \frac{1}{2}\big[\phi'(r)\big]^2
  &= \frac{1}{2}\big[\delta\phi'(r)\big]^2 \\
  &= \frac{A^2}{2}\,\frac{e^{-2m_\phi r}}{r^2}
     \Big(m_\phi^2 + \frac{2m_\phi}{r} + \frac{1}{r^2}\Big).
\end{align}

Next, the potential difference:
\begin{align}
  V(\phi) - V(v)
  &\simeq \frac{1}{2}m_\phi^2[\delta\phi(r)]^2 \\
  &= \frac{1}{2}m_\phi^2\,\frac{A^2}{r^2}e^{-2m_\phi r}.
\end{align}

Adding the two contributions,
\begin{align}
  \rho_\phi(r)
  &\simeq \frac{A^2}{2}\,\frac{e^{-2m_\phi r}}{r^2}
          \Big(m_\phi^2 + \frac{2m_\phi}{r} + \frac{1}{r^2}\Big)
          + \frac{1}{2}m_\phi^2\frac{A^2}{r^2}e^{-2m_\phi r} \\
  &= \frac{A^2}{2}\,e^{-2m_\phi r}
     \Big[
       \frac{m_\phi^2}{r^2} + \frac{2m_\phi}{r^3} + \frac{1}{r^4}
       + \frac{m_\phi^2}{r^2}
     \Big] \\
  &= \frac{A^2}{2}\,e^{-2m_\phi r}
     \Big[
       \frac{2m_\phi^2}{r^2}
       + \frac{2m_\phi}{r^3}
       + \frac{1}{r^4}
     \Big].
  \label{eq:rho-phi-explicit}
\end{align}

\subsection{Three radial regimes}

Equation \eqref{eq:rho-phi-explicit} contains three power-law terms multiplied by an exponential cutoff. Their relative importance changes with radius.

\begin{enumerate}
  \item \textbf{Core region} ($r \lesssim r_{\rm core}$): \\
  The linearisation $|\delta\phi|\ll v$ breaks down. Here $\phi$ is far from $v$, often crossing through zero. The full non-linear equation \eqref{eq:EOM-full} must be solved. For qualitative estimates we may approximate the core as having roughly constant density $\rho_{\rm core}$.

  \item \textbf{Intermediate halo} ($r_{\rm core} \ll r \ll m_\phi^{-1}$): \\
  In this range $e^{-2m_\phi r}\approx 1$, and the terms in \eqref{eq:rho-phi-explicit} scale as $r^{-2}$, $r^{-3}$, $r^{-4}$. For $r \gg m_\phi^{-1}$ the exponential kills everything, but here the $r^{-2}$ term dominates:
  \begin{equation}
    \rho_\phi(r) \approx \frac{A^2 m_\phi^2}{r^2},
    \qquad r_{\rm core} \ll r \ll m_\phi^{-1}.
    \label{eq:rho-r2}
  \end{equation}

  \item \textbf{Outer region} ($r \gg m_\phi^{-1}$): \\
  The exponential suppression $e^{-2m_\phi r}$ becomes decisive, and $\rho_\phi(r)$ decays faster than any power law:
  \begin{equation}
    \rho_\phi(r) \sim A^2 m_\phi^2\,\frac{e^{-2m_\phi r}}{r^2},
    \qquad r \gg m_\phi^{-1}.
  \end{equation}
  This ensures the total mass $M_\phi(\infty)$ is finite.
\end{enumerate}

The intermediate regime \eqref{eq:rho-r2} is key for dark-matter-like halos: a $1/r^2$ density profile leads to a linearly growing enclosed mass and a flat rotation curve.

%===========================================================
\section{Mass profile and rotation curve}
%===========================================================

\subsection{Enclosed mass}

The mass enclosed within radius $r$ is
\begin{equation}
  M_\phi(r) = 4\pi\int_0^r dr'\,r'^2\rho_\phi(r').
\end{equation}

In the intermediate regime where $\rho_\phi(r)\approx A^2 m_\phi^2/r^2$,
\begin{align}
  M_\phi(r)
  &\approx 4\pi A^2 m_\phi^2 \int_{r_{\rm core}}^r \frac{dr'}{r'^2}\,r'^2 \\
  &= 4\pi A^2 m_\phi^2 (r - r_{\rm core}) \\
  &\approx 4\pi A^2 m_\phi^2 r, \qquad r \gg r_{\rm core}.
  \label{eq:Mphi-linear}
\end{align}
We have neglected the contribution from the core region, which adds a constant offset to $M_\phi(r)$ but does not affect the $r$-dependence at large $r$.

Thus, for $r_{\rm core} \ll r \ll m_\phi^{-1}$,
\begin{equation}
  M_\phi(r) \propto r.
  \label{eq:M-propto-r}
\end{equation}

\subsection{Circular velocity}

In the weak-field, quasi-Newtonian limit, the circular velocity $v_c(r)$ for a test particle on a circular orbit of radius $r$ satisfies
\begin{equation}
  v_c^2(r) = \frac{G_0 M_{\rm tot}(r)}{r},
\end{equation}
where $M_{\rm tot}(r)$ is the total mass (ordinary + defect) inside $r$. If the defect contribution dominates in the halo region, $M_{\rm tot}(r)\approx M_\phi(r)$ there.

Using \eqref{eq:Mphi-linear},
\begin{align}
  v_c^2(r)
  &\approx \frac{G_0}{r}\,4\pi A^2 m_\phi^2 r \\
  &= 4\pi G_0 A^2 m_\phi^2.
\end{align}
Remarkably, the factors of $r$ cancel, giving
\begin{equation}
  v_c(r) \approx v_{\rm flat}
  \equiv \sqrt{4\pi G_0}\,A m_\phi,
  \qquad r_{\rm core} \ll r \ll m_\phi^{-1}.
  \label{eq:vflat}
\end{equation}
The rotation curve is approximately \emph{flat} over this radial range.

\subsection{Three-zone picture}

Putting everything together, a spherical defect in the broken phase produces a rotation curve with three characteristic zones:

\begin{enumerate}
  \item \textbf{Inner core} ($r \lesssim r_{\rm core}$): \\
  $\rho_\phi \approx {\rm const}$, $M_\phi(r) \propto r^3$, $v_c(r) \propto r$.

  \item \textbf{Intermediate halo} ($r_{\rm core} \ll r \ll m_\phi^{-1}$): \\
  $\rho_\phi(r) \propto r^{-2}$, $M_\phi(r) \propto r$, $v_c(r) \approx {\rm const}$.

  \item \textbf{Outer region} ($r \gg m_\phi^{-1}$): \\
  $\rho_\phi(r)$ decays exponentially, $M_\phi(r)$ approaches a constant $M_\phi(\infty)$, $v_c(r) \propto r^{-1/2}$ (Keplerian).
\end{enumerate}

This three-zone structure resembles observed galactic rotation curves, with a rising inner part, a flat middle part, and a falling outer part.

%======================================================================
\section{Discussion}
\label{sec:discussion}
%======================================================================

\subsection{Summary of the defect-based halo picture}

In this note we have analysed static, spherically symmetric configurations of
the strict Z$_2$ order parameter in the broken phase and shown that
inhomogeneities in $\phi$ naturally support dark-matter--like halo profiles.
Starting from the general stress--energy tensor $T^{(\phi)}_{\mu\nu}$, we
defined the defect energy density $\rho_\phi(r)$ as the excess over the
homogeneous vacuum and studied its behaviour in regimes where the scalar field
deviates from the vacuum value $v$.

In the weak-field, quasi-Newtonian approximation, configurations in which
$\phi(r)$ interpolates away from $v$ over a large radial range can produce an
effective density profile with $\rho_\phi(r)\propto r^{-2}$, leading to
approximately flat circular velocity curves $v_c(r)\simeq \mathrm{const}$ over
that range. This realises, at the level of the scalar field alone, a standard
phenomenological requirement for dark-matter haloes, without introducing a
separate particle species.

\subsection{Validity of approximations and limitations}

The analysis is deliberately conservative and relies on several simplifying
assumptions:
\begin{itemize}
  \item \textbf{Static, spherically symmetric configurations.} We have focused
        on time-independent, spherically symmetric solutions. Real galactic
        haloes are neither perfectly static nor exactly spherical, and
        departures from this idealised picture may affect both the existence
        and stability of defect-supported profiles.
  \item \textbf{Weak-field and linearised regimes.} Most of the explicit
        calculations were performed in the weak-field limit and, in some
        regions, using linearised equations for $\phi(r)$ around the vacuum
        value $v$. The matching between these linearised outer profiles and a
        fully non-linear core solution remains to be established.
  \item \textbf{Neglect of baryons and environment.} The rôle of baryonic
        matter, feedback, and the broader cosmological environment has been
        ignored. In realistic galaxies these effects can significantly reshape
        the inner halo profile and must eventually be included.
  \item \textbf{Parameter sensitivity.} The characteristic halo scales depend
        on the same parameters $(\alpha,\beta,G_0)$ that control the homogeneous
        vacuum energy. Although this tight coupling is conceptually attractive,
        it also means that the allowed parameter space is highly constrained.
\end{itemize}

These limitations do not invalidate the basic mechanism, but they do delimit
the domain of validity of the present treatment. A full assessment of the
model's viability requires relaxing some of these assumptions and solving the
non-linear field equations numerically.

\subsection{Phenomenological implications and tests}

From a phenomenological standpoint, the defect-based halo picture suggests
several concrete signatures and tests:
\begin{itemize}
  \item \textbf{Rotation curves and scaling relations.} The approximate
        $\rho_\phi(r)\propto r^{-2}$ behaviour over an extended radial interval
        should reproduce flat rotation curves in spirals. The model can be
        tested by confronting its predicted profiles with observed velocity
        data and scaling relations (e.g.\ Tully--Fisher-type relations).
  \item \textbf{Core structure.} The inner behaviour of $\phi(r)$ determines
        whether haloes are cored or cuspy. Numerical solutions of the
        Einstein--scalar system with realistic boundary conditions will
        provide predictions for the central profiles, to be compared with
        data from dwarf galaxies and low-surface-brightness systems.
  \item \textbf{Scaling with cosmological parameters.} Because the same
        parameters that fix $\Lambda_{\rm eff}$ also control the defect sector,
        any successful fit to galactic haloes must be consistent with the
        background cosmology. This linkage provides a non-trivial cross-check
        of the unified picture.
  \item \textbf{Non-sphericity and substructure.} Deviations from spherical
        symmetry, as well as the presence of subhaloes, tidal features and
        environmental effects, may leave characteristic imprints in lensing
        maps and kinematic data. These features offer additional channels to
        test whether defect-supported haloes can reproduce the observed
        complexity of galactic and cluster-scale dark matter.
\end{itemize}

In summary, the present analysis demonstrates that within the strict Z$_2$
framework, inhomogeneous configurations of a single scalar order parameter
provide a plausible mechanism for generating dark-matter--like haloes. Whether
this mechanism can satisfy the full set of observational constraints, while
remaining consistent with the homogeneous cosmology and other sectors of the
PhaseGeometry programme, is the subject of ongoing and future work.


%=========================================================
\section*{Next steps and Core Package structure}
%=========================================================

This note is part of the PhaseGeometry Z$_2$ Core Package v2.5 (strict).
Together with the other notes listed below it defines a minimal binary-phase
framework for dark energy, dark-matter--like haloes, black holes and quantum
branching in a single Z$_2$ medium. The Core Package is archived as a bundled
record under Zenodo DOI\,10.5281/zenodo.17807433.

The current structure of the Core Package is:
\begin{itemize}
  \item \textbf{Foundations I -- Technical Passport: Minimal Binary Phase Model for Dark Sector and Gravity}\\
        Defines the strict Z$_2$ action $S[g,\phi]$, potential $V(\phi)$, phase
        structure and basic ontology of the binary phase medium.
  \item \textbf{Foundations II -- Lambda from Broken Phase}\\
        Derives how the homogeneous broken phase $\phi \simeq \pm v$ acts as an
        effective cosmological constant
        $\Lambda_{\rm eff} = 8\pi G_0 V(v)$.
  \item \textbf{Foundations III -- Dark Matter from Defects}\\
        Develops the defect/inhomogeneous sector and defines the excess energy
        density $\rho_\phi(r)$ as a dark-matter--like component.
  \item \textbf{Foundations IV -- Phase medium, observer and branching histories}\\
        Interprets the Z$_2$ field as a phase medium hosting observers and
        classical branches in an Everett--Zurek picture.
  \item \textbf{Foundations V -- Decoherence and Quantum Darwinism in a binary phase medium}\\
        Implements decoherence and Quantum Darwinism explicitly in finite Z$_2$
        chains, with redundant records stored in environmental fragments.
  \item \textbf{Phenomenology I -- Background cosmology and the age of the Universe}\\
        Confronts the homogeneous broken phase with FRW background evolution,
        distance--redshift relations and cosmic age constraints.
  \item \textbf{Phenomenology II -- Dark-matter--like haloes from defects}\\
        Studies static halo profiles supported by Z$_2$ defects and compares
        them with rotation-curve phenomenology.
  \item \textbf{Phenomenology III -- Static black holes with binary phase hair}\\
        Embeds black holes into the same binary phase medium and explores weak
        scalar hair and defect-like shells in the strong-gravity regime.
\end{itemize}

Across these layers, the same strict Z$_2$ order parameter $\phi$ underlies:
(i) the effective cosmological constant from the homogeneous broken phase,
(ii) dark-matter--like haloes from defects and inhomogeneities,
(iii) phase-dressed black holes, and
(iv) quantum branching and classical records realised as patterns in the
binary medium.

The present note should be read as one component of this unified picture. It is
designed to be technically self-contained, but its full meaning emerges when
combined with the other items in the PhaseGeometry Z$_2$ Core Package
v2.5 (strict).



\end{document}